\chapter{Getting Started}

\section{Insertion sort}

\begin{exercise}{1}
    Using Figure 2.2 as a model, illustrate the operation of I\textsc{nsertion-}S\textsc{ort} on an array initially containing the sequence $\langle 31, 41, 59, 26,41, 58\rangle$.
\end{exercise}

\begin{solution}

    We show the steps of the algorithm below, underlining the element being processed in each iteration.
    The element is shown as already positioned in its correct place, implying that the elements that follow it (up to and including position $i$ for the $i$-th iteration) have been shifted rightwards:

    \[\langle \underline{31}, 41, 59, 26,41, 58\rangle \rightarrow \langle 31, \underline{41}, 59, 26,41, 58\rangle \rightarrow  \langle 31, 41, \underline{59}, 26,41, 58\rangle \rightarrow\]
    \[\langle \underline{26}, 31, 41, 59, 41, 58\rangle \rightarrow
    \langle 26, 31, 41, \underline{41}, 59, 58\rangle \rightarrow
    \langle 26, 31, 41, 41, \underline{58}, 59\rangle\]
\end{solution}

\begin{exercise}{2}
    Consider the procedure S\textsc{um-}A\textsc{rray}:

    \begin{codebox}
        \Procname{$\proc{Sum-Array}(A, n)$}
        \li $sum \gets 0$
        \li \For $i \gets 1$ \To $n$
        \li  \Do $sum \gets sum + A[i]$
        \End
        \li \Return $sum$
    \end{codebox}

    It computes the sum of the $n$ numbers in aray $A[1:n]$.
    State a loop invariant for this procedure, and use its initialization, maintenance, and termination properties to show that the S\textsc{um-}A\textsc{rray} procedure returns the sum of the numbers in $A[1:n]$.
\end{exercise}

\begin{solution}

    The loop invariant is: at the start of the $i$-th iteration, the variable $sum$ contains precisely the sum of $A$'s elements at indices $1, 2, \ldots, i - 1$.
    \begin{itemize}
        \item \textbf{Initialization}: At the start of iteration 1, line 1 has caused $sum$ to contain 0, which equals the sum of $A$'s elements for the empty collection of indices.
        \item \textbf{Maintenance}: Suppose the property holds before iteration $i$, which means $sum = \sum_{k=1}^{i - 1} A[k]$.
        Then, the execution of iteration $i$ means that line 3 now causes $sum$ to be incremented by $A[i]$, and thus $\sum_{k=1}^{i} A[k]$.
        But this means precisely that the invariant is true before iteration $i + 1$ as well.
        \item \textbf{Termination}: The loop trivially terminates due to running for a constant number of iterations $n$.
        Furthermore, we can think of its end as the start of an ``imaginary'' iteration $n + 1$.
        Then, due to the invariant, it will hold that $sum = \sum_{k=1}^{n} A[n]$, which is the sum of all elements in the array.
    \end{itemize}
\end{solution}
