\chapter{Completeness}

\section{Completeness}

\begin{exercise}{1}
    If $A \subset B \subset M$ and \(B\) is totally bounded, then \(A\) is totally bounded.
\end{exercise}

\begin{solution}
    
    Pick any $\epsilon > 0$.
    By Lemma 7.1, because $B$ is totally bounded, there exist finitely many sets $B_1, \ldots, B_n \subset B$ such that $\text{diam}(B_i) < \epsilon$ for all $i$ and $B \subset \bigcup_{i=1}^{n} B_i$.
    Let then $A_i = B_i \cap A$, each of which is clearly a subset of $A$.
    It holds of course that $\text{diam}(A_i) \leq \text{diam}(B_i) < \epsilon$.
    Furthermore, because $A \subset B \subset \bigcup_{i=1}^{n} B_i$, any $x \in A$ must belong in at least one $B_i$, and hence also in the corresponding $A_i$, which means that $A \subset \bigcup_{i=1}^{n} A_i$.
    By all of the above and Lemma 7.1, we conclude that $A$ is totally bounded.
\end{solution}

\begin{exercise}{2}
    Show that a subset $A \subset \mathbb{R}$ is totally bounded if and only if it is bounded.
    In particular, if $I$ is a closed, bounded, interval in $\mathbb{R}$ and $\epsilon > 0$, show that $I$ can be covered by finitely many closed subintervals $J_1, \ldots, J_n$, each of length at most $\epsilon$.
\end{exercise}

\begin{solution}

    We recall example 7.2 (a): a totally bounded set $A$ is necessarily bounded.
    To show this, pick $\epsilon = 1$ and obtain $x_1, \ldots, x_n$ such that $A \subset \bigcup_{i=1}^{n} B_1(x_i)$.
    Set $M = \max_{i, j} d(x_i, x_j)$ and pick any two $x, y \in A$, for which it must hold that $x \in B_i, y \in B_j$ for some $i, j$.
    By the triangle inequality:

    \[d(x, y) \leq d(x, x_i) + d(x_i, x_j) + d(x_j, y) < 1 + M + 1 = M + 2\]

    This of course shows that $A$ is bounded.
    
    Conversely, if $A \subset \mathbb{R}$ is bounded, then $\text{diam} A < R$, which in $\mathbb{R}$ is equivalent to $A \subset [-R, R]$.
    For any $\epsilon > 0$, set $N = \lceil \frac{2R}{\epsilon} \rceil$ and subdivide $[-R, R]$ into $N$ closed subintervals of length $\epsilon, [-R, a_1], [a_1, a_2], \ldots [a_{N-1}, R]$.
    Then observe that any element of $A$ must belong in one of these intervals, and thus is at most $\epsilon$-far from some $a_i$.
    Since there are finitely many $a_i$, we conclude that $A$ is totally bounded.
\end{solution}

\begin{exercise}{4}
    Show that $A$ is totally bounded if and only if $A$ can be covered by finitely many \textit{closed} sets of diameter at most $\epsilon$ for every $\epsilon > 0$.
\end{exercise}

\begin{solution}
    
    $\implies$: Suppose first that $A$ is totally bounded and pick any $\epsilon > 0$.
    Then there exist finitely many $x_1, \ldots, x_n$ such that $A \subset \bigcup_{i=1}^{n} B_{\epsilon/4}(x_i)$.
    As in the book, we may assume that $A \cap B_{\epsilon/4}(x_i) \neq \emptyset$ for all $i$, and thus for each $x_i$ find $a_i \in A, a_i \in B_{\epsilon/4}(x_i)$.
    Now, pick any $y \in B_{\epsilon/4}(x_i)$, and by the triangle inequality:

    \[d(y, a_i) \leq d(y, x_i) + d(x_i, a_i) \leq \epsilon/4 + \epsilon/4 \leq \epsilon/2\]

    This shows that all elements of $B_{\epsilon/4}(x_i)$ belong in the \textit{closed} ball of radius $\epsilon/2$ around $a_i$.
    More specifically then, all elements of $A$ that were covered by $B_{\epsilon/4}(x_i)$ are also covered by this closed ball.
    We conclude that the union of these $n$ closed balls ---each of which of course has diameter $\epsilon$--- covers $A$.

    $\impliedby$: Conversely, suppose that for any $\epsilon > 0$, $A$ can be covered by finitely many closed sets of diameter at most $\epsilon$.
    For any $\epsilon > 0$, use this hypothesis on $\epsilon/2$ to obtain finitely many closed sets $S_1, \ldots, S_n$ such that $A \subset \bigcup_{i=1}^{n} S_i$ and $\text{diam}(S_i) \leq \epsilon/2$ for all $i$.
    Once again, we may safely assume that $A \cap S_i \neq \emptyset$ for all $i$, and thus for every $i$ find an $a_i \in A, a_i \in S_i$.
    For any $y \in S_i$, it holds that $d(y, a_i) \leq \text{diam}(S_i) \leq \epsilon/2 < \epsilon$, and so $y \in B_{\epsilon}(a_i)$.
    This means that all elements of $A$ that were covered by $S_i$ are also covered by $B_{\epsilon}(a_i)$, meaning that $A \subset \bigcup_{i=1}^{n} B_{\epsilon}(a_i)$, which is of course the definition of total boundedness for $A$.
\end{solution}

\begin{exercise}{5}
    Prove that $A$ is totally bounded if and only if $\overline{A}$ is totally bounded.
\end{exercise}

\begin{solution}
    
   $\implies$: Suppose $A$ is totally bounded and pick any $\epsilon > 0$.
   Find $x_1, \ldots, x_n$ such that $A \subset \bigcup_{i=1}^{n} B_{\epsilon/2} (x_i)$, which is always possible due to $A$ being totally bounded.
   For any $z \in \overline{A}$, we have the following.
   If $z \in A$, we have that $z$ is inside at least one of these $n$ balls of radius $\epsilon/2$, and thus also inside the corresponding ball of radius $\epsilon$.
   If $z \notin A$, by the definition of the closure it must be that $y_i \rightarrow z$ for some $(y_i) \subset A$.
   Then there exists $N > 0$ such that for all $i > N, d(y_i, z) < \epsilon/2$.
   Pick any of these $y_i$, and find a corresponding $x_j$ such that $y_i \in B_{\epsilon/2}(x_j)$.
   By the triangle inequality, $d(y_i, x_j) \leq d(y_i, z) + d(z, x_j) < \epsilon/2 + \epsilon/2 = \epsilon$, which means that $y_i \in B_{\epsilon}(x_j)$.
   We've therefore shown total boundedness for $\overline{A}$.

   $\impliedby$: In the other direction, recall that $A \subset \overline{A}$, so by Exercise 1 if $\overline{A}$ is totally bounded, $A$ is as well.
\end{solution}

\begin{exercise}{9}
    Give an example of a closed bounded subset of $l_{\infty}$ that is not totally bounded.
\end{exercise}

\begin{solution}
    
    Consider the sequence formed in the following way:


\end{solution}