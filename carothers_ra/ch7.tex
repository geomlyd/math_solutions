\chapter{Completeness}

\section{Completeness}

\begin{exercise}{1}
    If $A \subset B \subset M$ and \(B\) is totally bounded, then \(A\) is totally bounded.
\end{exercise}

\begin{solution}
    
    Pick any $\epsilon > 0$.
    By Lemma 7.1, because $B$ is totally bounded, there exist finitely many sets $B_1, \ldots, B_n \subset B$ such that $\text{diam}(B_i) < \epsilon$ for all $i$ and $B \subset \bigcup_{i=1}^{n} B_i$.
    Let then $A_i = B_i \cap A$, each of which is clearly a subset of $A$.
    It holds of course that $\text{diam}(A_i) \leq \text{diam}(B_i) < \epsilon$.
    Furthermore, because $A \subset B \subset \bigcup_{i=1}^{n} B_i$, any $x \in A$ must belong in at least one $B_i$, and hence also in the corresponding $A_i$, which means that $A \subset \bigcup_{i=1}^{n} A_i$.
    By all of the above and Lemma 7.1, we conclude that $A$ is totally bounded.
\end{solution}

\begin{exercise}{2}
    Show that a subset $A \subset \mathbb{R}$ is totally bounded if and only if it is bounded.
    In particular, if $I$ is a closed, bounded, interval in $\mathbb{R}$ and $\epsilon > 0$, show that $I$ can be covered by finitely many closed subintervals $J_1, \ldots, J_n$, each of length at most $\epsilon$.
\end{exercise}

\begin{solution}

    We recall example 7.2 (a): a totally bounded set $A$ is necessarily bounded.
    To show this, pick $\epsilon = 1$ and obtain $x_1, \ldots, x_n$ such that $A \subset \bigcup_{i=1}^{n} B_1(x_i)$.
    Set $M = \max_{i, j} d(x_i, x_j)$ and pick any two $x, y \in A$, for which it must hold that $x \in B_i, y \in B_j$ for some $i, j$.
    By the triangle inequality:

    \[d(x, y) \leq d(x, x_i) + d(x_i, x_j) + d(x_j, y) < 1 + M + 1 = M + 2\]

    This of course shows that $A$ is bounded.
    
    Conversely, if $A \subset \mathbb{R}$ is bounded, then $\text{diam} A < R$, which in $\mathbb{R}$ is equivalent to $A \subset [-R, R]$.
    For any $\epsilon > 0$, set $N = \lceil \frac{2R}{\epsilon} \rceil$ and subdivide $[-R, R]$ into $N$ closed subintervals of length $\epsilon, [-R, a_1], [a_1, a_2], \ldots [a_{N-1}, R]$.
    Then observe that any element of $A$ must belong in one of these intervals, and thus is at most $\epsilon$-far from some $a_i$.
    Since there are finitely many $a_i$, we conclude that $A$ is totally bounded.
\end{solution}

\begin{exercise}{4}
    Show that $A$ is totally bounded if and only if $A$ can be covered by finitely many \textit{closed} sets of diameter at most $\epsilon$ for every $\epsilon > 0$.
\end{exercise}

\begin{solution}
    
    $\implies$: Suppose first that $A$ is totally bounded and pick any $\epsilon > 0$.
    Then there exist finitely many $x_1, \ldots, x_n$ such that $A \subset \bigcup_{i=1}^{n} B_{\epsilon/4}(x_i)$.
    As in the book, we may assume that $A \cap B_{\epsilon/4}(x_i) \neq \emptyset$ for all $i$, and thus for each $x_i$ find $a_i \in A, a_i \in B_{\epsilon/4}(x_i)$.
    Now, pick any $y \in B_{\epsilon/4}(x_i)$, and by the triangle inequality:

    \[d(y, a_i) \leq d(y, x_i) + d(x_i, a_i) \leq \epsilon/4 + \epsilon/4 \leq \epsilon/2\]

    This shows that all elements of $B_{\epsilon/4}(x_i)$ belong in the \textit{closed} ball of radius $\epsilon/2$ around $a_i$.
    More specifically then, all elements of $A$ that were covered by $B_{\epsilon/4}(x_i)$ are also covered by this closed ball.
    We conclude that the union of these $n$ closed balls ---each of which of course has diameter $\epsilon$--- covers $A$.

    $\impliedby$: Conversely, suppose that for any $\epsilon > 0$, $A$ can be covered by finitely many closed sets of diameter at most $\epsilon$.
    For any $\epsilon > 0$, use this hypothesis on $\epsilon/2$ to obtain finitely many closed sets $S_1, \ldots, S_n$ such that $A \subset \bigcup_{i=1}^{n} S_i$ and $\text{diam}(S_i) \leq \epsilon/2$ for all $i$.
    Once again, we may safely assume that $A \cap S_i \neq \emptyset$ for all $i$, and thus for every $i$ find an $a_i \in A, a_i \in S_i$.
    For any $y \in S_i$, it holds that $d(y, a_i) \leq \text{diam}(S_i) \leq \epsilon/2 < \epsilon$, and so $y \in B_{\epsilon}(a_i)$.
    This means that all elements of $A$ that were covered by $S_i$ are also covered by $B_{\epsilon}(a_i)$, meaning that $A \subset \bigcup_{i=1}^{n} B_{\epsilon}(a_i)$, which is of course the definition of total boundedness for $A$.
\end{solution}

\begin{exercise}{5}
    Prove that $A$ is totally bounded if and only if $\overline{A}$ is totally bounded.
\end{exercise}

\begin{solution}
    
   $\implies$: Suppose $A$ is totally bounded and pick any $\epsilon > 0$.
   Find $x_1, \ldots, x_n$ such that $A \subset \bigcup_{i=1}^{n} B_{\epsilon/2} (x_i)$, which is always possible due to $A$ being totally bounded.
   For any $z \in \overline{A}$, we have the following.
   If $z \in A$, we have that $z$ is inside at least one of these $n$ balls of radius $\epsilon/2$, and thus also inside the corresponding ball of radius $\epsilon$.
   If $z \notin A$, by the definition of the closure it must be that $y_i \rightarrow z$ for some $(y_i) \subset A$.
   Then there exists $N > 0$ such that for all $i > N, d(y_i, z) < \epsilon/2$.
   Pick any of these $y_i$, and find a corresponding $x_j$ such that $y_i \in B_{\epsilon/2}(x_j)$.
   By the triangle inequality, $d(y_i, x_j) \leq d(y_i, z) + d(z, x_j) < \epsilon/2 + \epsilon/2 = \epsilon$, which means that $y_i \in B_{\epsilon}(x_j)$.
   We've therefore shown total boundedness for $\overline{A}$.

   $\impliedby$: In the other direction, recall that $A \subset \overline{A}$, so by Exercise 1 if $\overline{A}$ is totally bounded, $A$ is as well.
\end{solution}

\begin{exercise}{9}
    Give an example of a closed bounded subset of $l_{\infty}$ that is not totally bounded.
\end{exercise}

\begin{solution}
    
    Consider the set corresponding to the sequence examined in Example 7.2 (d):
    \[S = \{(1, 0, \ldots), (0, 1, 0, \ldots), (0, 0, 1, \ldots), \ldots\} \subset l_{\infty}\]

    It's easy to see that the set is bounded, since all of its elements have a $l_\infty$ norm of 1.
    At the same time, it is closed, since the only convergent sequences it contains are eventually constant ones.
    However, the observation given in that example shows that it is not totally bounded: any two of its elements are at a distance of 1, and hence no finite number of balls of radius less than 1 can cover it.

\end{solution}

\begin{exercise}{10}
    Prove that a totally bounded metric space $M$ is separable. [Hint: For each $n$, let $D_n$ be a finite $(1/n)$-net for $M$.
    Show that $D = \bigcup_{n=1}^{\infty} D_n$ is a countable dense set.]
\end{exercise}

\begin{solution}
    
    Suppose $M$ is totally bounded, and form the construction indicated in the hint.
    Namely, for each $n$, collect the \textit{finite}, due to total boundedness, $x_1, x_2, \ldots, x_{k_n}$ needed to cover $M$ with balls of radius $1/n$.
    Call this set of $k_n$ elements $D_n$, and let $D = \bigcup_{n=1}^{\infty} D_n$, which is countable as a countable union of countable sets.
    Now, select any $y \in M$ and for each $i$, select an element $x_i \in D_i$ such that $d(y, x_i) < 1/i$, which is guaranteed to exist.
    Notice that this sequence gets arbitrarily close to $y$, and hence $D$ is also dense, showing that $M$ is separable.
\end{solution}

\begin{exercise}{11}
    Prove that $H^{\infty}$ is totally bounded (see Exercises 3.10 and 4.48).
\end{exercise}

\begin{solution}
    
    Recall the methodology we followed in Exercise 4.48.
    In particular, recall the observation that given any $\epsilon > 0$, we can find $N$ such that for any $x \in H^{\infty}$ and any sequence $r = (r_1, \ldots, r_N, 0, \ldots)$ it holds that under the metric of $H^{\infty}$:

    \[d(x, r) < \sum_{n=1}^{N} \lvert x_n - r_n \rvert + \frac{\epsilon}{2}\]

    If we can show that for any $\epsilon > 0$ we can select \textit{finitely} many sequences of the form $r = (r_1, \ldots, r_N, 0, 0, \ldots)$ such that for any $x \in H^{\infty}$, it is the case that for at least one of them it holds that $\sum_{n=1}^{N} \lvert x_n - r_{i,n} \rvert < \epsilon/2$, we will have shown that $d(x, r_i) < \epsilon$, and thus that $H^{\infty}$ is totally bounded.
    Let then $K = \lceil \frac{4N}{\epsilon} \rceil$, and consider subdividing the interval $[-1, 1]$ in $K$ equal subintervals, each of length $\frac{2}{K}$.
    Let $r_1, r_2, \ldots, r_K$ be the endpoints of these subintervals.
    Then, for any given sequence $(x_n)$, and for any of its terms $x_i$, it must hold that for some $r_j$:

    \[\lvert x_i - r_j \rvert < \frac{2}{K} < \frac{2}{\frac{4N}{\epsilon}} < \frac{\epsilon}{2N}\]

    We form $r$ by selecting such an $r_j$ for each $i = 1, 2, \ldots, N$, and then we have that:

    \[d(x, r) < \sum_{n=1}^{N} \frac{\epsilon}{2N} + \frac{\epsilon}{2} = \epsilon\]

    To summarize, for any $\epsilon > 0$, pick first $N$ as described previously, then select $r_1, \ldots, r_K$ as above (where $K$ is a function of $N, \epsilon$), and then set $R = \{(\rho_1, \rho_2, \ldots, \rho_N, 0, 0, \ldots), \rho_i \in \{r_1, r_2, \ldots, r_K\}\}$, which is a finite set such that for any $x \in H^{\infty}, d(x, r) < \epsilon$ for some $r \in R$.
    Therefore, $H^{\infty}$ is totally bounded.
\end{solution}

\section{Completeness}

\begin{exercise}{12}
    Let $(A, d)$ be a subset of an arbitrary metric space \((M, d)\).
    If \((A, d)\) is complete, show that \(A\) is closed in \(M\).
\end{exercise}

\begin{solution}
    
    Suppose $(x_n) \subset A$ is a sequence that converges to $x$.
    By Exercise 36 of Ch. 3, we know that $(x_n)$ is Cauchy as a convergent sequence.
    Because $A$ is complete, this means that $(x_n)$ must converge to a point inside $A$, which of course shows that $x \in A$, meaning that $A$ is indeed closed in $M$.
\end{solution}

\begin{exercise}{16}
    Prove that $\mathbb{R}^n$ is complete under any of the norms $\lvert \lvert \cdot \rvert \rvert_1, \lvert \lvert \cdot \rvert \rvert_2, \lvert \lvert \cdot \rvert \rvert_{\infty}$.
    [This is interesting because completeness is not usually preserved by the mere equivalence of \textit{metrics}.
    Here we use the fact that all of the metrics involved are generated by \textit{norms}.
    Specifically, we need the norms in question to be equivalent as functions: $\lvert \lvert \cdot \rvert \rvert_{\infty} \leq \lvert \lvert \cdot \rvert \rvert_2 \leq \lvert \lvert \cdot \rvert \rvert_1 \leq n \lvert \lvert \cdot \rvert \rvert_{\infty}$.
    As we will see later, any two norms on $\mathbb{R}^n$ are comparable in this way.]
\end{exercise}

\begin{solution}

    Suppose $(x_i)$ is a Cauchy sequence in $\mathbb{R}^n$ under $\lvert \lvert \cdot \rvert \rvert_1$.
    Then for every $\epsilon > 0$, there exists $N > 0$ such that for every $i, j \geq N$ it holds that:

    \[\lvert \lvert x_i - x_j \rvert \rvert_1 < \epsilon \implies \sum_{k=1}^{n}\lvert x_{i,k} - x_{j, k} \rvert < \epsilon\]

    This of course means that each of the coordinate sequences is Cauchy in $\mathbb{R}$, and so it converges to some limit, which we call $l_k$ for the $k$-th coordinate.
    Then we already know that $(x_i)$ converges to $(l_1, l_2, \ldots, l_n)$, which is clearly in $\mathbb{R}^n$ and so $\mathbb{R}^n$ is complete under $\lvert \lvert \cdot \rvert \rvert_1$.
    Note also that the equivalence of the metrics induced by these norms shows that $x_i \rightarrow_{\infty} (l_1, l_2, \ldots, l_n), x_i \rightarrow_{2} (l_1, l_2, \ldots, l_n)$ as well.
    What we furthermore have is that, as stated in the hint, $\lvert \lvert x_i - x_j \rvert \rvert_{\infty} \leq \lvert \lvert x_i - x_j \rvert \rvert_1 < \epsilon$, so any sequence that is Cauchy under $\lvert \lvert \cdot \rvert \rvert_1$ is also Cauchy under $\lvert \lvert \cdot \rvert \rvert_{\infty}$.
    Similarly, because $\lvert \lvert x_i - x_j \rvert \rvert_1 \leq n \lvert \lvert x_i - x_j \rvert \rvert_{\infty}$, any Cauchy sequence under $\lvert \lvert \cdot \rvert \rvert_{\infty}$ is also Cauchy under $\lvert \lvert \cdot \rvert \rvert_1$., and so the two sets of Cauchy sequences are the identical, and the argument presented above for $\lvert \lvert \cdot \rvert \rvert_1$ suffices to show that $\mathbb{R}^n$ is complete under $\lvert \lvert \cdot \rvert \rvert_{\infty}$ as well.
    By a similar argument we can see that $\mathbb{R}^n$ is complete under $\lvert \lvert \cdot \rvert \rvert_2$ as well.
\end{solution}

\begin{exercise}{17}
    Given metric spaces $M, N$, show that $M \times N$ is complete if and only if both $M, N$ are complete.
\end{exercise}

\begin{solution}
    
    We will use $d_1((x_1, y_1), (x_2, y_2)) = d_M(x_1, x_2) + d_N(y_1, y_2)$ as the product metric on $M \times N$ (from Exercise 46 of Ch. 3 we know that we have three choices that define equivalent metrics).
    Suppose first that $M, N$ are complete.
    Pick any Cauchy sequence $((x_i, y_i)) \in M \times N$.
    Then for any $\epsilon > 0$ there exists $K > 0$ such that for $k, l \geq K$ it holds that:

    \[d_1((x_k, y_k), (x_l, y_l)) < \epsilon \implies d_M(x_k, x_l) + d_N(y_k, y_l) < \epsilon\]
    This of course shows that $d_M(x_k, x_l) < \epsilon, d_N(y_k, y_l) < \epsilon$ for $k, l \geq K$, i.e., that $(x_i), (y_i)$ are Cauchy in $M, N$ respectively.
    By the completeness of $M, N$, we have that $x_i \rightarrow x \in M, y_i \rightarrow y \in N$, and then Exercise 46 Ch. 3 guarantees that $(x_i, y_i) \rightarrow (x, y)$, which is of course an element of $M \times N$, and so we've shown completeness for $M \times N$.

    Conversely, assume $M \times N$ is complete, and pick any Cauchy sequence $(x_i) \in M$.
    For a non-trivial problem, we assume that $N \neq \emptyset$, and so there exists at least one $y \in N$.
    Form the sequence $(x_i, y) \subset M \times N$, and pick any $\epsilon > 0$.
    We then have that there exists $K > 0$ such that for $k, l \geq K$: 

    \[ d_M(x_k, x_l) < \epsilon \implies d_M(x_k, x_l) + 0 < \epsilon \implies d_M(x_k, x_l) + d_N(y, y) < \epsilon \implies d_1((x_k, y), (x_l, y)) < \epsilon\]

    This shows that $(x_i, y)$ is Cauchy in $M \times N$, and by the completeness of $M \times N$ we obtain that $(x_i, y) \rightarrow (x, y')$ for some $x \in M, y' \in N$.
    Because convergence in the product space implies convergence in each of $M, N$, we have that $x_i \rightarrow x, y \rightarrow y'$ ($y' = y$ due to the fact that this sequence is constant).
    We've therefore shown that any Cauchy sequence in $M$ converges to an element of $M$, which means that $M$ is complete.
    Exchanging the roles of $M, N$ shows that the same holds for $N$.

\end{solution}

\begin{exercise}{18}
    Fill in the details of the proofs that $l_1, l_{\infty}$ are complete.
\end{exercise}

\begin{solution}
    
    The two proofs are done in much the same way as the proof that $l_2$ is complete.
    We begin with $l_1$.
    Consider a Cauchy sequence (of sequences) $(f_n) \subset l_1$, where we write $f_n = (f_n(k))_{k=1}^{\infty}$.
    Fix any $k \in \mathbb{N}$ and consider the sequence $(f_n(k))$.
    Select any $\epsilon > 0$ and observe that because $(f_n)$ is Cauchy, there exists $N > 0$ such that for any $n, m \geq N$ we have that $\lvert \lvert f_n - f_m \rvert \rvert _{1} < \epsilon$.
    Thus, it also holds that:

    \[\lvert f_n(k) - f_m(k) \rvert \leq \lvert \lvert f_n - f_m \rvert \rvert _{1} < \epsilon,\]

    which shows that the sequence $(f_1(k), f_2(k), \ldots)$ is Cauchy in $\mathbb{R}$ and thus converges.
    Call its limit $f(k)$ and set $f = (f(1), f(2), \ldots)$.
    We now show that $f \in l_1$.
    Fix any $N \in \mathbb{N}$, and consider the ``prefix'' $(f(1), f(2), \ldots, f(N))$.
    We have that:

    \[\sum_{n=1}^{N} \lvert f(n) \rvert = \sum_{n=1}^{N} \lvert \lim_{m \rightarrow \infty} f_m(n) \rvert = \lim_{m \rightarrow \infty} \sum_{n=1}^{N} \lvert f_m(n) \rvert\]

    Notice now that the RHS here is bounded, since $f_m \in l_1$ for any $m$.
    Therefore, for any $N, \sum_{n=1}^{N} \lvert f(n) \rvert \leq B$, and since limits preserve non-strict inequalities, $\lvert \lvert f \rvert \rvert _{1} \leq B \implies f \in l_1$.
    Finally, we need to show that $f_n \rightarrow f$ in $l_1$.
    Again, fix $N$.
    Select $m_0$ such that for $m, l \geq m_0, \lvert \lvert f_m - f_l \rvert \rvert _{1} < \epsilon$ and observe that if $m \geq m_0$:

    \[\sum_{n=1}^{N} \lvert f_m(n) - f(n) \rvert = \sum_{n=1}^{N} \lvert \lim_{k \rightarrow \infty} f_k(n) - f_m(n) \rvert = \lim_{k \rightarrow \infty} \sum_{n=1}^{N} \lvert f_k(n) - f_m(n) \rvert \leq \lim_{k \rightarrow \infty} \lvert \lvert f_k - f_m \rvert \rvert _{1} \leq \epsilon\]

    Since this holds for any $N$, we have that indeed $f_n \rightarrow f$, and we've thus shown that $l_1$ is complete (any Cauchy sequence in it converges to a point in it).


\end{solution}

\begin{exercise}{20}
    If $(x_n), (y_n)$ are Cauchy in $(M, d)$ show that $(d(x_n, y_n))_{n=1}^{\infty}$ is Cauchy in $\mathbb{R}$.
\end{exercise}

\begin{solution}
    
    Pick any $\epsilon > 0$.
    We have first that:
    
    \[\lvert d(x_n, y_n) - d(x_m, y_m) \rvert = \lvert d(x_n, y_n) - d(x_m, y_n) + d(x_m, y_n) - d(x_m, y_m) \rvert \leq \]
    \[\lvert d(x_n, y_n) - d(x_m, y_n) \rvert + \lvert d(x_m, y_n) - d(x_m, y_m) \rvert \leq d(x_n, x_m) + d(y_n, y_m),\]

    where we've used the triangle inequality in $\mathbb{R}$ and Exercise 2 of Ch. 3 (reverse triangle inequality in a metric space).
    Due to $(x_n), (y_n)$ being Cauchy, we can find $N, M > 0$ such that whenever $n, m \geq N, d(x_n, x_m) < \epsilon/2$ and whenever $n, m \geq M, d(y_n, y_m) < \epsilon/2$.
    Set $K = \max\{N, M\}$ and then we conclude that for $n, m \geq K, d(x_n, x_m) < \epsilon/2, d(y_n, y_m) < \epsilon/2$, which by the above inequality shows that $\lvert d(x_n, y_n) - d(x_m, y_m) \rvert < \epsilon$, i.e.\, that $(d(x_n, y_n))_{n=1}^{\infty}$ is Cauchy in $\mathbb{R}$.
\end{solution}