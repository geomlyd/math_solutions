\chapter{Open Sets and Closed Sets}

\begin{exercise}{1}
    Show that an ``open rectangle'' $(a, b) \times (c, d)$ is an open set in $\mathbb{R}^2$.
    More generally, if $A, B$ are open in $\mathbb{R}$, show that $A \times B$ is open in $\mathbb{R}^2$.
    If $A, B$ are closed in $\mathbb{R}$, show that $A \times B$ is closed in $\mathbb{R}^2$.
\end{exercise}

\begin{solution}
    
    Call the open rectangle $S$, in which case its defining property is $S = \{(x, y) \in \mathbb{R}^2 : a < x < b, c < y < d\}$.
    Let $p = (x, y) \in S$, and set $r = \min\{x - a, b - x, c - y, d - y\}$.
    Now let $p' = (x', y')$ be any point in $B_r(p)$.
    This means that $\lvert \lvert p' - p \rvert \rvert_2 < r$.
    More specifically, this translates to the following inequalities:

    $$\lvert x' - x \rvert < x - a \implies x' - x > a - x \implies a < x'$$
    $$\lvert x' - x \rvert < b - x \implies x' - x < b - x \implies x' < b$$    
    $$\lvert y' - y \rvert < y - c \implies y' - y > c - y \implies y' > c$$
    $$\lvert y' - y \rvert < d - y \implies y' - y > d - y \implies y' < d$$

    These four properties mean that $p' \in S$, and thus we have proved that $S$ is indeed open.

    For the sake of generality, for the remaining exercise we will work with $A, B$ being subsets of two general metric spaces $(M, d), (N, \rho)$, respectively, and then the general result will imply what is asked for here.
    First of all, we select $d_{\infty}$ as the metric of $M \times N$, with the definition given in exercise 46 of Chapter 3.
    Suppose first that $A, B$ are open, and pick $(a, b) \in A \times B$.
    There exist then open balls $B_{\epsilon_1}^{d}(a) \subset A, B_{\epsilon_2}^{\rho}(b) \subset B$.
    Pick $\epsilon = \min\{\epsilon_1, \epsilon_2\}$, and consider any $(x, y) \in B_{\epsilon}^{d_{\infty}}((a, b))$.
    We have that:
    
    $$d_{\infty}((x, y), (a, b)) < \epsilon \implies \max\{d(x, a), \rho(y, b)\} < \min\{\epsilon_1, \epsilon_2\}$$

    Notice that this implies that, more specifically, $d(x, a) < \epsilon_1, \rho(y, b) < \epsilon_2$, which means $x \in B_{\epsilon_1}^{d}(a) \subset A, y \in B_{\epsilon_2}^{\rho}(b) \subset B$.
    In other words, $x \in A, b \in B$, and so $(x, y) \in A \times B$, which shows that $B_{\epsilon}^{d_{\infty}}(a, b) \subset A \times B$, and thus the Cartesian product is also open.

    Now suppose that $A, B$ are closed, which implies $M \setminus A, N \setminus B$ are open.
    Consider rewriting $A \times B$ as follows:

    $$A \times B = \{(x, y) \in M \times N \lvert x \in A, y \in B\} = (M \times N) \setminus \{(x, y) \in M \times N \lvert x \notin A \text{ or } y \notin B\}$$
    $$ = (M \times N) \setminus ((A^{c} \times N) \cup (M \times B^{c}))$$

    Now, from the immediately preceding proof, $A^{c} \times N, M \times B^{c}$ are open as Cartesian products of open sets (a metric space is always open).
    Their union is thus also open, and then the complement of this union, which we showed equals $A \times B$, is thus closed.
\end{solution}

\newpage

\begin{exercise}{3}
    Some authors say that two metrics $d, \rho$ on a set $M$ are equivalent if they generate the same open sets.
    Prove this.
    (Recall that we defined equivalence to mean that $d, \rho$ generate the same convergent sequences.)
\end{exercise}

\begin{solution}
    
    First suppose that two metrics $d, \rho$ generate the same open sets.
    Pick any sequence $(x_n) \in M$ such that $d(x_n, x) \rightarrow 0$, and pick $\epsilon > 0$.
    The ball $B_{\epsilon}^{\rho}(x)$ is open under $\rho$, and must thus also be open (although not necessarily a ball) under $d$.
    By convergence of $x_n \rightarrow x$ under $d$, we have that $(x_n)$ is eventually in $B_{\epsilon}^{\rho}(x)$, which of course means that we have showed that $\rho(x_n, x) \rightarrow 0$ as well.
    Exchanging the roles of $d, \rho$ completes the proof that they are equivalent under the sequence-based definition.

    Conversely, assume that $d, \rho$ generate the same convergent sequences.
    Suppose $S$ is an open set under $d$.
    If $S$ is the empty set, it is also clearly open under $\rho$.
    Otherwise, suppose for the sake of contradiction that $S$ is not open under $\rho$, which means that there exists $x \in S$ such that for every $\epsilon > 0$, $B_{\epsilon}^{\rho}(x) \not\subset S$.
    More specifically, this means that for a sequence of $\epsilon_n = \frac{1}{n}$, we can always find $x_n \notin S$ such that $x_n \in B_{\epsilon_n}^{\rho}(x)$.
    But then it becomes clear that $\rho(x_n, x) \rightarrow 0$, which means that $d(x_n, x) \rightarrow 0$ as well.
    Therefore we have constructed a sequence that has an infinite number of terms that do not belong in $S$, and at the same time converges to $x \in S$ under $d$.
    Because $S$ is open under $d$, this is a contradiction, and we therefore conclude that $S$ must also be open under $\rho$.
    Exchanging the roles of the two metrics completes the proof in this direction, thus showing that the two definitions are equivalent.
\end{solution}

\begin{exercise}{5}
    Let $f: \mathbb{R} \rightarrow \mathbb{R}$ be continuous.
    Show that $\{x: f(x) > 0\}$ is an open subset of $\mathbb{R}$, and that $\{x: f(x) = 0\}$ is a closed subset of $\mathbb{R}$.
\end{exercise}

\begin{solution}
    
    Let $S = \{x: f(x) > 0\}$.
    Pick any $x \in S$, in which case $f(x) > 0$.
    Let $\epsilon = f(x)$, and then because $f$ is continous, there must exist $\delta > 0$ such that:

    $$\lvert y - x \rvert < \delta \implies \lvert f(y) - f(x) \rvert < \epsilon$$

    The second inequality implies $-\epsilon < f(y) - f(x) \implies f(y) > f(x) - f(x) = 0$.
    Notice then that for every $y \in B_{\delta}(x)$ we have $f(y) > 0$, and thus that $y \in S$, which shows that $S$ is open.

    Now let $T = \{x: f(x) = 0\}$.
    Notice that $T = \mathbb{R} \setminus (\{x: f(x) > 0\} \cup \{x: f(x) < 0\})$.
    It is very easy to show that the above also holds for ``less than zero'', thus rendering the union here a union of open sets, and then $T$ a closed set as a complement of an open set.

\end{solution}

\begin{exercise}{7}
    Show that every open set in $\mathbb{R}$ is the union of countably many open intervals with $\textit{rational}$ endpoints.
    Use this to show that the collection $\mathcal{U}$ of all open subsets of $\mathbb{R}$ has the same cardinality as $\mathbb{R}$ itself.
\end{exercise}

\begin{solution}
    
    We know already that an open set $S \subset \mathbb{R}$ can be written as $S = \cup_{n=1}^{\infty}(a_n, b_n)$, where the intervals are all disjoint, and perhaps unbounded.
    Consider the following procedure.
    We take the interval $(a_n, b_n)$, where in general the endpoints are real numbers.
    We know that between them there will always be a rational number $q_n$.
    In fact, $b_n$ can be approached by a (increasing) sequence of rational numbers starting at $q_n$ and getting infinitely close to it, and the same holds for $a_n$ (only difference being the fact that this will be a decreasing sequence).
    These are \textit{countably infinite} in number: indeed, each sequence is equivalent to the natural numbers, since these directly form its indices, and of course $\mathbb{N} \cup \mathbb{N} \sim \mathbb{N}$ (``two sequences are equivalent to one sequence'').

    We can thus replace $(a_n, b_n)$ with a union of the form $\cup_{k=1}^{\infty} (q_k, r_k)$ where the endpoints are rationals.
    But then we have that $S$ is a countable union of countable sets, which we know (Theorem 2.6)  is also countable, and thus $S$ is indeed a countable union of intervals with rational endpoints.

    Now let $f: \mathbb{R} \setminus \{0\} \rightarrow \mathcal{U}, f(x) = (0, x)$ if $x > 0$, and $f(x) = (x, 0)$ otherwise. 
    This function is clearly an injection from the reals except zero to the set of open sets.
    Recall also (exercise 17, ch. 2) that $\mathbb{R} \sim \mathbb{R} \setminus \{0\}$, and thus we can easily see that there exists an injection $h_1$ from $\mathbb{R}$ to $\mathcal{U}$.
    Additionally, let $g((q_n)) = \cup_{n=1}^{\infty} (q_n, q_{n+1})$ be a function mapping sequences of rationals to open sets.
    We've shown that \textit{every} open set can be written in this form, meaning that $g$ is a surjection.
    But then for every open interval we can find at least one sequence of rationals that via $g$ (since it's a function) corresponds to it and to no other open interval.
    This shows the existence of an injection $h_2: \mathcal{U} \rightarrow \mathbb{Q}^{\mathbb{N}}$.
    Putting together the existence of $h_1, h_2$ and the fact that $\mathbb{Q}^{\mathbb{N}} \sim \mathbb{R}$, we apply the Cantor-Schröder-Bernstein theorem to see that $\mathcal{U}$ and $\mathbb{R}$ are in fact equivalent.
\end{solution}

\begin{exercise}{8}
    Show that every open interval (and hence every open set) in $\mathbb{R}$ is a countable union of closed intervals and that every closed interval in $\mathbb{R}$ is a countable intersection of open intervals.
\end{exercise}

\begin{solution}
    
    Consider any interval of the form $(a, b)$, first where $a, b \in \mathbb{R}$, and consider the union:

    $$U = \bigcup_{n=1}^{\infty} \Biggl[\frac{a+b}{2} - \sum_{i=2}^{n}(b-a)\frac{1}{2^i}, \frac{a+b}{2} + \sum_{i=2}^{n}(b-a)\frac{1}{2^i} \Biggr]$$

    For the sake of brevity, we will omit some details, but one can see that as $n \rightarrow \infty$, the terms inside the sums converge to $(b-a)/2$.
    But this means that for any given $x \in (a, b)$, it suffices to check whether $x < (a+b)/2$ or $x > (a+b)/2$, and then pick $n$ accordingly to make the left or right endpoint of the $n$-th term of the union become smaller or greater than $x$ respectively.
    This is always possible due to the sums converging to $(b-a)/2$, which in turn makes the endpoints converge to $a, b$.
    The union thus includes all points in $(a, b)$, and furthermore it is clearly countable.

    If it is the case that $a = -\infty$ and $b = \infty$, we propose the union $U = \cup_{n=1}^{\infty} [-2^n, 2^n]$, and we can thus clearly always find $n$ big enough such that any $x \in \mathbb{R}$ belongs in one such interval.
    The ``intermediate'' case of one of $a, b$ being infinity and the other a real number would be handled as a mixture of the above.

    Now consider instead an interval of the form $[a, b], a, b \in \mathbb{R}$.
    Consider the intersection:

    $$I = \bigcap_{n=1}^{\infty} \Biggl[ a - \frac{1}{2^n}, b + \frac{1}{2^n}\Biggr]$$

    We claim that this (countable) intersection equals $[a, b]$.
    First, it is easy to see that any $x \in [a,b]$ is contained in every interval, since $b + \frac{1}{2^n} > b, a - \frac{1}{2^n} < a$ for all $n$.
    Second, we claim that no $y \notin [a, b]$ can be contained in $I$.
    Indeed, if $y < a$, because $a - \frac{1}{2^n}$ gets arbitrarily close to $a$ from the left, we can always find $n$ sufficiently large such that $y < a - \frac{1}{2^n}$, and a similar argument works for $ y > b$, showing that $y$ will not be contained in the intervals corresponding to $k \geq n$, and thus neither will it be contained in $I$.
    This concludes the proof.
\end{solution}

\begin{exercise}{10}
    Given $y = (y_n) \in H^{\infty}, N \in \mathbb{N}, \epsilon > 0$, show that $\{x = (x_n) \in H^{\infty}: \lvert x_k - y_k \rvert < \epsilon, k =1, \ldots, N\}$ is open in $H^{\infty}$ (see Exercise 3.10).
\end{exercise}

\begin{solution}
    
    Let $S = \{x = (x_n) \in H^{\infty}: \lvert x_k - y_k \rvert < \epsilon, k =1, \ldots, N\}$, and suppose $x \in S$.
    Suppose also, by way of contradiction, that there exists a sequence $z^{(k)} \rightarrow x$ with an infinite number of its elements not in $S$.
    This means that there must exist infinite $k$ such that $\lvert z^{(k)}_n - y_n \rvert \geq \epsilon$ for at least one $n = 1, 2, \ldots, N$.
    We first make the observation that $z^{(k)} \rightarrow x$ implies coordinate-wise convergence for each $n$, in other words that $z^{(k)} \rightarrow x_n$ for all n.
    Indeed, given a particular $n$, pick any $\epsilon > 0$ and set $\epsilon' = 2^{-n} \epsilon$.
    Then there must exist $K > 0$ such that for $k \geq K$ (recall the metric for $H^{\infty}$ from exercise 3.10):

    $$d(z^{(k)}, x) < \epsilon' \implies \sum_{m=1}^{\infty} 2^{-m} \lvert z^{(k)}_m - x_m \rvert < \epsilon'$$

    But then it is the case that $2^{-n} \lvert z^{(k)}_n - x_n \rvert < 2^{-n}\epsilon \implies \lvert z^{(k)}_n - x_n \rvert < \epsilon$ for all $k \geq K$, which is the definition of convergence for the $n$-th coordinate.
    Now pick an arbitrary $n$, and consider the following:

    $$\lvert z_n^{(k)} - y_n \rvert \leq \lvert z_n^{(k)} - x_n \rvert + \lvert x_n - y_n \rvert $$

    The second term here is independent of $z$, so let it equal $c_n < \epsilon$ (by the definition of $S$).
    Set $\epsilon_n = \epsilon - c_n > 0$, and notice that there must exist $K > 0$ such that for $k \geq K, \lvert z_n^{k} - x_n \rvert < \epsilon_n$.
    Then:

    $$\lvert z_n^{(k)} - y_n \rvert < \epsilon_n + c_n = \epsilon - c_n + c_n = \epsilon$$

    The implication here is that after some $K$, this $n$ can no longer ``play the part'' of keeping $z^{(k)} \notin S$ by satisfying $\lvert z_n^{(k)} - y_n \rvert \geq \epsilon$.
    This is true for any $n = 1, 2, \ldots, N$, so we can find a maximum such $K$ such that for $k \geq K$ the above holds for \textit{all} $n$.
    However, the existence of infinite $k$ such that $z^{(k)} \notin S$ directly contradicts this, and thus $S$ must indeed be open.

\end{solution}

\begin{exercise}{11}
    Let $e^{(k)} = (0, \ldots, 0, 1, 0, \ldots)$, where the $k$-th entry is 1 and the rest are 0s.
    Show that $\{e^{(k)}: k \geq 1\}$ is closed as a subset of $l_1$.
\end{exercise}

\begin{solution}
    
    Let $x = (x_1, x_2, \ldots)$ be any element of $S = \{e^{(k)}: k \geq 1\}$.
    We will show that if, for every $\epsilon > 0$, it is the case that $B_{\epsilon}(x) \cap S \neq \emptyset$ (under $l_1$), then $x \in S$, which we know is one of the three equivalent ways of defining closed sets.
    What this statement means is that for every $\epsilon > 0$, there must exist a sequence $y = (0, 0, \ldots, 1, 0, \ldots)$ with precisely one (say, the $i$-th) element equal to 1 and the rest equal to 0, and at the same time $\lvert \lvert (x_1, x_2, \ldots, x_i, \ldots) - (0, 0, \ldots, 1, 0, \ldots) \rvert \rvert_1 < \epsilon \implies \lvert \lvert (x_1, x_2, \ldots 1 - x_i, x_{i+1}, \ldots) \rvert \rvert_1 < \epsilon$.
    Suppose then that $x \notin S$.
    One possibility is that $x$ has at least two non-zero elements, $x_j, x_k$.
    Notice then that no matter the position of $i$, it holds that $\lvert \lvert (x_1, x_2, \ldots 1 - x_i, x_{i+1}, \ldots) \rvert \rvert_1 \geq \min\{\lvert x_j \rvert, \lvert x_k \rvert\}$: indeed, $y$ can at most zero out one of the two non-zero elements of $x$. 
    But this means that for sufficiently small $\epsilon$ the statement above will not hold, a contradiction.
    Therefore $x$ has at most one non-zero element.
    If $x$ is the zero sequence, it is clear that the LHS above will always have a norm of 1, and thus we again arrive at a contradiction.
    The only possibility is thus that $x = (0, \ldots, a, 0, \ldots), a \neq 0$.
    Suppose that $a \neq 1$, and observe that for $\epsilon < 1$ it must be the case that the $y$ we seek has $y_i = 1$, otherwise $\lvert \lvert (0, \ldots, 1, \ldots, a, \ldots) \rvert \rvert_1 \geq 1$.
    Therefore, it must be the case that $\lvert \lvert (0, \ldots 1 - a, 0, \ldots) \rvert \rvert_1 < \epsilon$ for all $\epsilon < 1$, which enforces $\lvert 1 -a \rvert < \epsilon$, and of course this can only be true for $a = 1$, thus showing that $x \in S$.
\end{solution}

\begin{exercise}{13}
    Show that $c_0$ is a closed subset of $l_{\infty}$.
    [Hint: If $(x^{(n)})$ is a sequence of sequences in $c_0$ converging to $x \in l_{\infty}$, note that $\lvert x_k \rvert \leq \lvert x_k - x_k^{(n)} \rvert + \lvert x_k^{(n)} \rvert$ and now choose $n$ so that $\lvert x_k - x_k^{(n)} \rvert$ is small \textit{independent} of $k$.]
\end{exercise}

\begin{solution}
    
    Suppose that $(x^{(n)}) \in c_0$ is a sequence of sequences converging to $x$.
    We need to show that $x \in c_0$, in other words that $x_k \rightarrow 0$.
    We follow the hint by observing that, for any given term $x^{(n)}$ of the sequence of sequences, via the triangle inequality we have that:

    $$\lvert x_k \rvert = \lvert x_k - x_{k}^{(n)} + x_{k}^{(n)} \rvert \leq \lvert x_k - x_k^{(n)} \rvert + \lvert x_k^{(n)} \rvert$$

    Pick any $\epsilon > 0$.
    Because $x^{(n)} \rightarrow x$, there exists $N > 0$ such that for $n \geq N$ we have that $d_{\infty}(x^{(n)}, x) < \epsilon \implies \sup_{k} \lvert x_k^{(n)} - x_k \rvert < \epsilon/2$.
    This guarantees that $\lvert x_k^{(n)} - x_k \rvert < \epsilon/2$ for \textit{any} $k$.
    In addition, we know that for any $n, x^{(n)} \in c_0 \implies x_k^{n} \rightarrow 0$, and so there must exist $K \geq 0$ such that $\lvert x_k^{n} \rvert < \epsilon/2$ for $k \geq K$.
    By using any $n \geq N$ in the inequality stated above, we see that for $k \geq K$ we have that $\lvert x_k \rvert < \epsilon$, which means that $x \rightarrow 0$, and thus that $x \in c_0$.
\end{solution}