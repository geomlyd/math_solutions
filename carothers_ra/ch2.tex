\chapter{Countable and Uncountable Sets}

\section{Equivalence and Cardinality}

\begin{exercise}{3}
    Given finitely many countable sets $A_1, \ldots, A_n$, show that $A_1 \cup \ldots \cup A_n$ and $A_1 \times \ldots \times A_n$ are countable sets.
\end{exercise}

\begin{solution}

    We will use induction on $n$:
    \begin{itemize}
        \item Base case, for $n = 2$: Consider two countable sets, $A_1, A_2$. If both of them are finite, then their union and Cartesian product are also finite, and thus trivially countable.
        
        If exactly one is finite, say $A_2 = \{a_1', a_2', \ldots, a_n'\}$, then observe that $A_1 \cup A_2$ contains at most all elements of $A_1$ and all elements of $A_2$, and possibly fewer if their intersection is not empty.
        In any case, suppose $\lvert A_1 \cap A_2 \rvert = m$, and name $f$ the bijection from $A_1$ to $\mathbb{N}$.
        Then let $S = A_2 \setminus A_1 = \{a_{k_1}', \ldots a_{k_{n-m}}'\}$, and $f'(a_{k_1}') = 1, \ldots f'(a_{k_{n-m}}') = n - m$, in which case $A_1 \cup A_2 = A_1 \cup S$, and $A_1, S$ have an empty intersection.
        Let $g: A_1 \cup S \rightarrow \mathbb{N}$ be such that:
        $$g(a) = \begin{cases} 
            f(a) + (n - m) &, a \in A_1 \\
            f'(a) &, a \in S
            \end{cases}$$
        Because $f$ is a bijection, one can clearly see that $g$ is surjective.
        Furthermore, $g$ is one-to-one because $f, f'$ are one-to-one and because the two ``branches'' of $g$ have no overlapping values ($\min\{f(a) + (n-m)\} = n - m + 1> n - m =\max\{f'(a)\}$).
        Therefore $g$ is a bijection from $A_1 \cup A_2$ to $\mathbb{N}$, which means precisely that the union is countable.

        Now for the case where both $A_1, A_2$ are infinite, there exist bijections $f_1 : A_1 \rightarrow \mathbb{N}, f_2 : A_2 \rightarrow \mathbb{N}$. 
        These impose orders $f_1(a_1) = 1, f_1(a_2) = 2, \ldots$ for $a_i' \in A_1$ and $f_2(a_1') = 1, f_2(a_2') = 2$ for $a_i \in A_2$.
        One can again set $S = A_2 \setminus A_1$, and $f_2$ can again be used to extract an order for the elements $s_1, s_2, \ldots$ of $S$. 
        If $S$ is finite, the problem reduces to the case above.
        If $S$ is also infinite, we know that $S$ is also countable, with $f_2': S \rightarrow \mathbb{N}$ the corresponding bijection.
        Additionally, $A_1 \cup A_2 = A_1 \cup S = \{a_1, s_1, a_2, s_2, \ldots\}$ ands $A_1 \cap S = \emptyset$. 
        Use the orders imposed by $f_1, f_2'$ to sort the elements of $A_1$ in the order $a_1, a_2, \ldots$ and the elements of $S$ in the order $s_1, s_2, \ldots$. 
        Then define $g:A_1 \cup A_2 \rightarrow \mathbb{N}$ as:
        $$g(a) = \begin{cases} 2i, & a = a_i \in A_1 \\
            2i + 1, & a = s_i \in S
            \end{cases}$$
        For $n \in \mathbb{N}$, if $n$ is even the equation $n = g(a)$ has a unique solution for $a = a_n$, due to the bijectivity of $f_1$.
        If $n$ is odd, the equation $n = g(a)$ has a unique solution for $a = s_n$, due to the bijectivity of $f_2'$. In any case, $A_1 \cup A_2$ has been shown to be equivalent to $\mathbb{N}$.

        For the Cartesian product, the case where both $A_1, A_2$ are finite is again trivial. 
        If $A_1$ is infinite and $A_2$ finite, then:
        $$A_1 = \{a_1, a_2, \ldots\}, A_2 = \{a_1', a_2', \ldots, a_n'\}$$
        Let then $g: A_1 \times A_2 \rightarrow \mathbb{N}$:
        $$g(a_i, a_j') = n(i-1) + (j - 1), \ j = 1, 2, \ldots, n$$
        Observe that the fact that $A_1 \sim \mathbb{N}$  and $j$ only takes a finite number of values makes $g$ a surjection. Indeed, for $x = k\cdot n + l, l = 0, 1, \ldots, n - 1$ (here we use the division algorithm for integers), one has but to set $i = 1 + k$ (possible due to $A_1 \sim \mathbb{N}$ and $j = l + 1$  (always possible due to the range of values $l$ can achieve) to obtain $g(a_i, a_j') = x$.\

        To show that $g$ is injective, suppose $g(a_i, a_j') = g(a_k, a_l')$, and we then have that:
        $$n(i - 1) + (j - 1) = n(k - 1) + (l - 1) \implies n(i - k) = l - j$$
        This implies that the RHS is a multiple of $n$, which, because $1 \leq l, j \leq n$ is only possible if $l = j$. But then we also have that $i = k$, thus that $(a_i, a_j') = (a_k, a_l')$, i.e.\ that $g$ is injective.

        $g$ is therefore a bijection, and thus $A_1 \times A_2 \sim \mathbb{N}$.

        If both $A_1, A_2$ are infinite, then consider the function $g: A_1 \times A_2 \rightarrow \mathbb{N}$:
        $$g(a_i, a_j') = 2^i(2j - 1)$$
        , which, because of the fact that $A_1 \sim \mathbb{N}, A_2 \sim \mathbb{N}$ (thus $i, j$ take all natural numbers as values), and by a proof completely analogous to $\mathbb{N} \times \mathbb{N} \rightarrow \mathbb{N}$ can be shown to be a bijection, thus proving that $A_1 \times A_2 \sim \mathbb{N}$.
        \item Inductive step: If this holds for $n = k \geq 2$, then observe that $A_1 \cup \ldots \cup A_{k+1} = (A_1 \cup \ldots \cup A_k) \cup A_{k+1}$, and we can therefore apply the inductive hypothesis to the sets $A_1, \ldots A_k$, and the base case to the union of those with $A_k$ to obtain the statement for $n = k+1$ as well.
        The same argument applies to the Cartesian product as well, thus concluding the proof.
    \end{itemize}

\end{solution}

\begin{exercise}{5}
    Prove that a set is infinite if and only if it is equivalent to a proper subset of itself.

    [Hint: If $A$ is infinite and $x \in A$, show that $A$ is equivalent to $A \setminus \{x\}.$]
\end{exercise}

\begin{solution}

    $\implies$: Suppose $A$ is infinite.
    Then $A$ contains an element $x$, and the set $S = A \setminus \{x\}$ must also be infinite.
    By exercise 4, the set $S$ contains a countably infinite subset $S'$.
    We then have that:
    $$S = S' \cup (S \setminus S'), A = S' \cup (S \setminus S') \cup \{x\}$$
    , where all of the sets used in the unions are disjoint.
    By exercise 3, it holds that $S' \cup \{x\} \sim S'$, and this yields a corresponding bijection $g: S' \cup \{x\} \rightarrow S'$.
    Consider then the following function $f: A \rightarrow S$:
    $$f(z) = \begin{cases}
        z,& z \in S \setminus S' \\
        g(z),& z \in S' \cup \{x\}
    \end{cases}$$
    , which, by the above observations regarding the bijectivity of $g$ and the disjointness of the sets $S', S \setminus S', \{x\}$, means that $f$ is a bijection as well, proving that $A \sim S$.
    
    $\impliedby$: Now suppose that a set $A$ is equivalent to $S$, $S$ being a proper subset of $A$. 
    By contradiction, suppose $A$ is not infinite and that it contains $n$ elements.
    Then $S$ must contain at most $n - 1$ elements.
    By the pigeonhole principle, there cannot exist a bijection from $A$ to $S$, which means that $A \sim S$ cannot be true, a contradiction.
    Therefore $A$ must be infinite.
\end{solution}


\begin{exercise}{8}
    Show that $(0, 1)$ is equivalent to $[0, 1]$ and to $\mathbb{R}$.
\end{exercise}

\begin{solution}

    First, observe that $[0, 1] = \{0\} \cup \{1\} \cup (0, 1)$, and that $[0, 1], [0, 1), (0, 1)$ are all infinite sets.
    By exercise 5, we have that $[0, 1) \sim (0, 1)$ (by explicitly picking $x = 0$ and $S = (0, 1)$.
    Similarly, we also have that $[0, 1] \sim [0, 1)$ (by explicitly picking $x = 1$ and $S = [0, 1)$).
    The transitivity property of the equivalence relation ``is equivalent to'' (exercise 1) yields then that $[0, 1] \sim (0, 1)$.

    For $(0, 1) \sim \mathbb{R}$, consider the function $f: \mathbb{R} \rightarrow (0, 1), f(x) = \frac{1}{1 + e^{-x}}$, which is known to be both injective and surjective, thus sufficing to show that $(0, 1) \sim \mathbb{R}$.
\end{solution}

\begin{exercise}{9}
    Show that $(0, 1)$ is equivalent to the unit square $(0, 1) \times (0, 1)$.

    [Hint: ``Interlace'' decimals, but carefully!]
\end{exercise}

\begin{solution}

    Pick any $x \in (0, 1) \times (0, 1)$. 
    Then $x = (a, b), a, b \in (0, 1)$. 
    If $a$ or $b$ can be written as a decimal ending in infinite 9's (e.g. $0.3999\ldots$), write them in the equivalent form that features finite decimals (in this example, 0.4).
    Now consider the function $f: (0, 1) \times (0, 1) \rightarrow (0, 1)$:
    $$f(0.a_1a_2a_3\ldots, 0.b_1b_2b_3\ldots) = 0.a_1b_1a_2b_2a_3b_3\ldots$$
     
    Observe first that for any $x \in (0, 1)$, $x$ can be written as $x = 0.x_1x_2x_3\ldots$, and again if this can be written as ending in infinite 9's, ``round it up''. 
    Then $f(0.x_1x_3x_5\ldots, 0.x_2x_4x_6\ldots) = x$. 
    Furthermore, suppose that $f(0.y_1y_2y_3\ldots, 0.z_1z_2z_3\ldots) = x = 0.x_1x_2x_3\ldots$. 
    By definition, it is also the case that 
    $$f(0.y_1y_2y_3\ldots, 0.z_1z_2z_3\ldots) = 0.y_1z_1y_2z_2y_3z_3\ldots$$
    Because we always choose to write decimals ending in infinite 9's as ``rounded up'', there can be no ambiguities here, in the sense that the equality of these two numbers directly implies that $y_1 = x_1, z_1 = x_2, y_2 = x_3, z_2 = x_4 \ldots$ (i.e.\, any two decimals that do not end in infinite 9's can only be equal if all corresponding digits are equal).
    These two observations yields that $f$ is bijective, thus proving the desired equivalence.
\end{solution}

\begin{exercise}{10}
    Prove that $(0, 1)$ can be put into one-to-one correspondence with the set of all \textit{functions} $f: \mathbb{N} \rightarrow \{0, 1\}$.
\end{exercise}

\begin{solution}

    To begin, we observe that a function from $\mathbb{N}$ to $\{0, 1\}$ can be thought of a set of the form:
    $$\{(0, v_0), (1, v_1), (2, v_2), \ldots \}$$,
    where each $v_i$ is the value of the function on $i$, and as such can be either 0 or 1.
    We now observe that the set of all such functions can be written as $S \cup \overline{S}$, where $S = \{f \in \{0, 1\}^{\mathbb{N}} \lvert \exists k \in \mathbb{N} \cup \{-1\}, f(i) = 1, i > k\}$ (in other words, a function either becomes 1 ``forever'' after some point, or it does not).
    Note that from now on we will use the binary system in this exercise.

    Observe that all functions that belong in $S$ are uniquely characterized by a ``finite-length prefix'' of values.
    This means that we can list all of them: start with a prefix of length 0, list all possible $f$ such that in the above definition $k = -1$, then list all $f$ with a prefix of length 1, etc.
    Since each length $N$ yields only a finite number of functions (each corresponds to a binary number with $N$ digits such that the last digit is not 1), we conclude that $S$ is in fact countable, and that there exists a bijection $g: S \rightarrow \mathbb{N}$.
    Define now the following mapping $F: \{0, 1\}^{\mathbb{N}} \rightarrow (0, 1) \cup \mathbb{N}$:
    $$F(f) = \begin{cases}
        0.v_0v_1v_2\ldots, & f \in \overline{S} \\
        g(f), & f \in S
    \end{cases}$$
    By the bijectivity of $g$, $F$ achieves all natural numbers.
    Furthermore, for any $x \in (0, 1)$, we can always write $x = 0.b_0b_1\ldots$ such that it does not end in infinite ones, and then we can define an $f: \mathbb{N} \rightarrow \{0, 1\}$ such that $f(i) = b_i$, which means that $F(f) = x$.
    We have thus shown that $F$ is surjective.
    Additionally, we observe that due to our construction, if $f_1 \in \overline{S}, f_2 \in S, F(f_1), F(f_2)$ can never be equal: one is always in $(0, 1)$ and the other is a non-zero natural number.
    For $f_1, f_2 \in \overline{S}, F(f_1) = F(f_2)$ either implies that all corresponding digits are equal, in which case clearly $f_1 = f_2$, or that one of them ends in infinite ones, which is impossible since $f_1, f_2 \in \overline{S}$.
    Finally, for $f_1, f_2 \in S$, the bijectivity of $g$ implies that $f_1 = f_2$.

    This shows that $F$ is also injective, which means it is a bijection from $\{0, 1\}^{\mathbb{N}}$ to $(0, 1) \cup \mathbb{N}$.
    Now , $\mathbb{N}$ is clearly countable, and from exercise 6 we then have that $(0, 1) \cup \mathbb{N} \sim (0, 1)$, which by exercise 1 also gives us our desired result, $\{0, 1\}^{\mathbb{N}} \sim (0, 1)$.

\end{solution}

\begin{exercise}{13}
    Show that $\mathbb{N}$ contains infinitely many pairwise disjoint infinite subsets.
\end{exercise}

\begin{solution}

    We know that there exist infinitely many primes, $p_1, p_2, \ldots$. 
    Consider then forming the sets $S_i = \{p_i^k \lvert k \in \mathbb{N} \setminus \{0\}\}$.
    These are clearly infinitely many, and each of them is infinite.
    Suppose that two of them, $S_i, S_j$ had a non-empty intersection.
    Then there exists $x \in \mathbb{N}$ such that $x = p_i^k = p_j^l, k, l \geq 1$.
    But then this number has two different prime factorizations, which is a contradiction.
    Therefore all of these sets are pairwise disjoint.
\end{solution}

\begin{exercise}{15}
    Show that any collection of pairwise disjoint, nonempty open intervals in $\mathbb{R}$ is at most countable. [Hint: Each one contains a rational!]
\end{exercise}

\begin{solution}

    Suppose that there exists an uncountable collection of pairwise disjoint and nonempty intervals of the form $(a, b), a, b \in \mathbb{R}$.
    As indicated in the hint, each such interval must contain a rational, regardless of whether $a, b$ are rationals or not.
    Furthermore, because the intervals are all disjoint, no two such rationals can be equal.
    This means that there exist at least as many rationals as intervals, therefore an uncountable number of them, which we know is a contradiction.
\end{solution}


\section{The Cantor Set}
    
\begin{exercise}{21}
    Show that any ternary decimal of the form $0.a_1 a_2 \ldots a_n 11$ (base 3), i.e., any finite-length decimal ending in two (or more) 1s is \textit{not} an element of $\Delta$.
\end{exercise}

\begin{solution}

    For this problem we will be using the following formalization of the Cantor set $\mathcal{C}$:
    $$\mathcal{C} = [0, 1] \setminus \bigcup_{n=0}^{\infty} \bigcup_{k=0}^{3^n - 1} \Bigl ( \frac{3k + 1}{3^{n+1}}, \frac{3k + 2}{3^{n+1}} \Bigr )$$

    We now consider a ternary decimal $x$ of the form $x = 0.a_1 \ldots a_n 1 1$, where $n$ may be zero, in which case the ``prefix'' has zero length. 
    Consider then what $x$ equals:
    $$x = \frac{a_1}{3} + \frac{a_2}{3^2} + \ldots +\frac{a_n}{3^n} + \frac{1}{3^{n+1}} + \frac{1}{3^{n+2}} = \frac{a_1 3^{n+1} + a_2 3^n + \ldots + a_n3^2 + 3 + 1}{3^{n+2}}$$
    $$= \frac{1}{3^{n+1}}(a_1 3^n + a_2 3^{n-1} + \ldots + a_n 3 + \frac{4}{3}) = \frac{1}{3^{n+1}}(3(a_1 3^{n-1} + a_2 3^{n-2} + \ldots + a_n) + \frac{4}{3})$$
    Notice now that if $y = a_1 3^{n-1} + a_2 3^{n-2} + \ldots + a_n$, $y$ is an integer that is at least 0 and at most $6\frac{3^n - 1}{2} = 3(3^n - 1)$ (by using the geometric progression sum).
    But then this range in which $y$ can move means that $x$ is precisely equal to \textit{one} of the left endpoints in the $n$-th union in the definition of $\mathcal{{C}}$ plus $\frac{4}{3}$, which, because $1 < \frac{4}{3} < 2$, means that $x$ will always lie \textit{inside} one of the intervals that comprise this union, which means of course that it lies outside of the Cantor set.
\end{solution}

\begin{exercise}{22}
    Show that $\Delta$ contains no (nonempty) open intervals.
    In particular, show that if $x, y \in \Delta$, then there is some $z \in [0, 1] \setminus \Delta$ with $x < z < y$. 
    (It follows from this that $\Delta$ is \textit{nowhere dense}, which is another way of saying that $\Delta$ is ``small''.)
\end{exercise}

\begin{solution}

    Let $x, y \in \Delta$.
    Following the notation of the book, we name $I_k$ the union of closed intervals that comprises the $k$-th ``level'' of the ``tree'' that creates the Cantor set.
    Then recall that $\Delta = \bigcap_{k=0}^{\infty} I_k$.
    Since $x, y \in \Delta$, it holds that $x, y \in I_k$ for all $k$.
    We now claim that there exists $k$ such that $x, y$ belong in two disjoint intervals that comprise the union $I_k$.
    This will of course imply that there exists a ``removed middle third'' interval between $x, y$ at the $k$-th level, any of whose elements fulfill the condition $z \notin \Delta$.
    We do this by contradiction. 
    Namely, suppose that $x, y$ belong in the same closed interval at every level of the ``tree'', and call those intervals $I_1, I_2, \ldots$.
    By construction, these are nested intervals, and we know that $I_n$'s length goes to zero as $n \rightarrow \infty$.
    But then the nested interval theorem dictates that the infinite intersection of $I_1, I_2, \ldots$ contains precisely one element, which means $x = y$, a contradiction.
\end{solution}

\begin{exercise}{23}
    The endpoints of $\Delta$ are those points in $\Delta$ having a finite-length base 3 decimal expansion (not necessarily in the proper form), that is, all of the points in $\Delta$ of the form $a/3^n$ for some integers $n$ and $0 \leq a \leq 3^n$.
    Show that the endpoints of $\Delta$ other than 0 and 1 can be written as $0.a_1 a_2 \ldots a_{n+1}$ (base 3) where each $a_k$ is 0 or 2, except $a_{n+1}$, which is either 1 or 2.
    That is, the discarded ``middle third'' intervals are of the form $(0.a_1 a_2 \ldots a_n 1, 0.a_1 a_2 \ldots a_n 2)$ where both entires are points of $\Delta$ written in base 3.
\end{exercise}

\begin{solution}
    
    We will apply induction on $n$, which corresponds to the ``level'' of the ``tree'' of the Cantor set.
    More specifically, we will prove the following.
    \begin{itemize}
        \item If $x \neq 0$ is a left endpoint of the Cantor set, that is, the the intersection forming the Cantor set contains it in the $n$-th ``tree-level union '' in some interval $[x, y]$, then it can be written in the form $0.a_1 \ldots a_n 2$ where $a_i \in \{0, 2\}, i =1, \ldots, n$.
        \item If $x \neq 1$ is a right endpoint of the Cantor set, that is, the intersection forming the Cantor set contains it in the $n$-th ``tree-level union'' in some interval $[y, x]$, then it can be written in the form $0.a_1 \ldots a_n 1$ where $a_i \in \{0, 2\}, i =1, \ldots, n$.
    \end{itemize}
    As stated, we will use induction on $n$ starting at 0, which corresponds to the first level of the tree.
    \begin{itemize}
        \item \textbf{Base case}: if $n = 0$, then the corresponding union is $I_1 = [0, 0.1] \cup [0.2, 1]$ (written in ternary).
        Obviously, the only non-zero left endpoint is 0.2, which satisfies the claim stated above.
        Similarly, the only right endpoint not equal to 1 is 0.1, which also satisfies the claim.
        \item \textbf{Inductive hypothesis}: Suppose that for $n = k, k \geq 0$ it holds that $I_n = [0, x_1] \cup [x_2, x_3] \cup \ldots \cup [x_M, 1]$, where each left and right endpoint fulfill the conditions stated above (note that $M = 2^{n + 1}$).
        \item \textbf{Inductive step}: Observe then that the intervals comprising $I_{n+1}$ are formed by taking each interval $[x, y]$ comprising $I_n$ and replacing it by $[x, x + 0.\underbrace{0 \ldots 0}_{n + 1} 1] \cup [y - \underbrace{0.0\ldots 0}_{n + 1} 1, y]$.
        Therefore, new right endpoints are formed by taking either 0 or a previous left endpoint and adding $0.\underbrace{0 \ldots 0}_{n + 1} 1$.
        Clearly, adding such a number to 0 results in a right endpoint fulfilling the claim stated in the beginning.
        In the other case, we have an endpoint of the form $0.a_1 \ldots a_n 2 + 0.\underbrace{0 \ldots 0}_{n + 1} 1 = 0.a_1 \ldots a_n 2 1$, which based on the inductive hypothesis also fulfills the initial claim.
        Similarly, new left endpoints are formed by taking either 1 or previous right endpoints and subtracting $0.\underbrace{0 \ldots 0}_{n + 1} 1$.
        Subtracting this quantity from 1 results in $0.\underbrace{2 \ldots 2}_{n + 1} 2$, which fulfills the claim for a left endpoint.
        On the other hand, subtracting it from a previous right endpoint results in an endpoint of the form $0.a_1 \ldots a_n 1 - 0.\underbrace{0 \ldots 0}_{n + 1} 1 = 0.a_1 \ldots a_n 0 2$, which based on the inductive hypothesis also satisfies the claim for left endpoints.
    \end{itemize}
    This concludes the proof that endpoints of $\Delta$ have a ternary decimal expansion described in the exercise.
\end{solution}

\begin{exercise}{26}
    Let $f: \Delta \rightarrow [0,1]$ be the Cantor function and let $x, y \in \Delta$ with $x < y$.
    Show that $f(x) \leq f(y)$.
    If $f(x) = f(y)$, show that $x$ has two distinct ternary decimal representations.
    Finally, show that $f(x) = f(y)$ if and only if $x, y$ are ``consecutive'' endpoints of the form $x = 0.a_1 a_1 \ldots a_{n} 1$ and $y = a_1 a_2 \ldots a_n 2$ (base 3).
\end{exercise}

\begin{solution}
    
    Let $x = 0.a_1 a_2 \ldots$ and $y = 0.b_1 b_2 \ldots$.
    Because $x, y \in \Delta$, we know that we can write $x, y$ such that each of $a_i, b_i$ is either 0 or 1.
    Now, because $x < y$ it has to be the case that for some $i \geq 1, a_i < b_i$ and $a_k = b_k$ for $k < i$.
    This can be seen by making a ``digit-wise'' comparison: the first decimal digit in which $x, y$ differ determines which one is larger, since after that point the decimal digits get multiplied by smaller powers of 3 and the smaller number can never ``catch up''.
    The only possible exception would be if $x = 0.a_1 \ldots a_i 0 0 \ldots$ and $y = 0.a_1 \ldots (a_i - 1) 2 2 \ldots$, but we know that this cannot be true since it would require $a_i = 1$ or $a_i - 1 = 1$, which is impossible due to the way we've written $x, y$.
    Now consider the decimal strings $x' = 0.(\frac{a_1}{2}) (\frac{a_2}{2}) \ldots, y' = 0.(\frac{b_1}{2}) ( \frac{b_2}{2}) \ldots$.
    What the Cantor function does is interpret these as binary instead of ternary numbers, and set then $f(x) = x'_{\text{bin}}, f(y) = y'_{\text{bin}}$.
    But then $\frac{a_k}{2} = \frac{b_k}{2}$ for $k < i, \frac{a_i}{2} < \frac{b_i}{2}$.
    By this we have established that it is impossible that $f(x) > f(y)$, thus $f(x) \leq f(y)$.
    If it is the case that $f(x) = f(y)$, then for $k > i, \frac{a_k}{2} = 1, \frac{b_k}{2} = 0$ and $\frac{a_i}{2} = 0, \frac{b_i}{2} = 1$.
    Going back to the ternary numbers $x, y$, this implies that $x$ is of the form:
    $$x = 0.a_1 \ldots a_{i-1} 0 2 2 2 \ldots,$$
    which can also be written as $x = 0.a_1 \ldots a_{i-1} 1$.
    Furthermore, it implies that $y$ is of the form:
    $$y = 0.a_1 \ldots a_{i-1} 2 0 0 0 \ldots,$$
    which completes the proof that a) $x$ has two distinct ternary decimal expansions, and that whenever $f(x) = f(y), x, y$ are endpoints of ``discarded'' middle third intervals.
    For the other direction of the last equivalence, if $x, y$ are endpoints of ``discarded'' middle third intervals, write both of them such that they only contain 0s and 2s in ternary, apply $f$ and observe that $f(x)$ ends in infinite 1s, $f(y)$ in 0s, and for some $i, f(x)_i 0, f(y)_i = 1$ and $f(x)_k = f(y)_k, k < i$, which of course means that $f(x) = f(y)$.
\end{solution}
\begin{exercise}{29}
    Prove that the extended Cantor function $f: [0, 1] \rightarrow [0, 1]$ (as defined above) is increasing. [Hint: consider cases.]
\end{exercise}

\begin{solution}
    
    For any two $x, y \in [0, 1]$, with $x < y$, we have the following:
    \begin{itemize}
        \item If $x, y \in \Delta$, we have shown in exercise 23 that $f(x) \leq f(y)$.
        \item If $x \in \Delta, y \notin \Delta$, then  $f(y) = \sup\{f(z), z \leq y, z \in \Delta\}$.
        This means that $f(x) \in \{f(z), z \leq y, z \in \Delta\}$, and thus it must be the case that $f(x) \leq f(y)$.
        \item If $x, y \notin \Delta$, then $f(x) = \sup\{f(z), z \leq x, z \in \Delta\}$ and $f(y) = \sup\{f(z), z \leq y, z \in \Delta\}$.
        But then since $x < y$, the set over which the second supremum is computed is a superset of the set over which the first supremum is computed, meaning that $f(x) \leq f(y)$.
    \end{itemize}
\end{solution}

\section{Monotone Functions}

\begin{exercise}{34}
    Let $D = \{x_1, x_2, \ldots\}$, and let $\epsilon_n > 0$ with $\sum_{n=1}^{\infty} \epsilon_n < \infty$.
    Define $f(x) = \sum_{x_n \leq x} \epsilon_n$ (as above).
    Check the following:
    
    (i) $f$ is discontinuous at the points of $D$

    (ii) $f$ is right-continuous everywhere

    (iii) $f$ is continuous at each point $x \in \mathbb{R} \setminus D$

    How might this construction be modified to yield a \textit{strictly} increasing function with these same properties?
\end{exercise}

\begin{solution}

    (i) Notice that we've already seen in the main text that for $x_k \in D, f(x_k-) = f(x_k) - \epsilon_k$, that is, the left limit of $f$ at each point of $D$ can never equal its value on the respective point, since $\epsilon_k > 0$.
    Therefore $f$ is not continuous at the points of $D$.

    (ii) Again, in the main text we've seen that $f$ is right-continuous at the points of $D$.
    It suffices therefore to check that for $x \in \mathbb{R} \setminus D, f(x+) = f(x)$.
    Expanding this, we have:
    $$f(x+) = \lim_{y \rightarrow x+} f(y)$$
    Pick any $\epsilon > 0$.
    We need to show then that there exists $\delta > 0$ such that whenever $y - x < \delta$ it holds that $f(y) - f(x) < \epsilon$ (note that the fact that $f$ is increasing and the fact that we are taking a right limit allows us to omit absolute values).
    We have that:
    $$f(y) - f(x) = \sum_{\{n: x_n \leq y\}} \epsilon_n - \sum_{\{n: x_n \leq x\}} \epsilon_n = \sum_{\{n: x < x_n \leq y\}} \epsilon_n$$
    As has already been observed in the main text, since the series corresponding to $\epsilon_n$ converges, it is the case that $\sum_{n=N}^{\infty} \epsilon_n = 0$ as $N \rightarrow \infty$.
    This means that given our chosen $\epsilon > 0$, we can find an $N > 0$ such that $\sum_{n=N}^{\infty} \epsilon_n < \epsilon$.
    Therefore, after excluding a \textit{finite} number of terms from the series, namely $\{\epsilon_1, \epsilon_2, \ldots, \epsilon_{N-1}\}$ we can make the corresponding remaining infinite sum arbitrarily small.
    Because the set is finite, we can then pick $\delta > 0$ such that none of these terms are in the interval $(x, y]$: simply find the closest term to $x$ and pick $y$ to be closer to $x$ than that. 
    Then by necessity:
    $$\sum_{\{n: x < x_n \leq y\}} \epsilon_n \leq \sum_{n=N}^{\infty} \epsilon_n < \epsilon,$$
    which means precisely that $f(y) - f(x) < \epsilon$, thus completing the proof that $f$ is right-continuous everywhere.

    (iii) Due to part (ii) we only need to prove that $f$ is left-continuous at $x \in \mathbb{R} \setminus D$.
    That is, we need to prove that for $x \in \mathbb{R} \setminus D$:
    $$\lim_{y \rightarrow x-} f(y) = f(x)$$
    Pick any $\epsilon >0 $.
    We need to find $\delta > 0$ such that whenever $x - y < \delta$:
    $$f(x) - f(y) < \epsilon \implies \sum_{\{n: x_n \leq x\}} \epsilon_n - \sum_{\{n: x_n \leq y\}} \epsilon_n < \epsilon \implies \sum_{\{n : y < x_n \leq x\}} \epsilon_n < \epsilon$$
    This means that we can now use the exact same observation as in (ii) to bring $y$ sufficiently close to $x$ such that the LHS sum is strictly smaller than $\epsilon$, thus proving that $f$ is indeed left-continuous at $x$.
    Note that the important difference from the case $x \in D$ examined in the book is that here $x_n \leq x$ does not introduce a fixed term in the sum, which would of course impose a lower bound on it.
    In contrast, when $x \in D$ it is not guaranteed that we can always exclude the finite set of terms obtained in (ii) since one of them \textit{may be} $x$ \textit{itself}.

    Now for the question posed in the exercise, consider the function $g(x) = f(x) + x$.
    Because the identity function is continuous everywhere, $g$ will maintain all properties of $f$ shown in the exercise: it will be right-continuous everywhere as a sum of right-continuous functions, it will be continuous at $\mathbb{R} \setminus D$ and at the points of $D$ it will be discontinuous (since otherwise $f(x) = g(x) - x$ would be continuous as a difference of continuous functions).
    In addition, for any two $x_1 < x_2$, we have that $f(x_1) \leq f(x_2)$, and therefore clearly $g(x_1) < g(x_2)$, meaning that $g$ is strictly increasing.
\end{solution}

\begin{exercise}{35}
    Let $f: [a, b] \rightarrow \mathbb{R}$ be increasing, and let $(x_n)$ be an enumeration of the discontinuities of $f$.
    For each $n$, let $a_n = f(x_n) - f(x_n-)$ and $b_n = f(x_n+) - f(x_n)$ be the left and right ``jumps'' in the graph of $f$, where $a_n = 0$ if $x_n = a$ and $b_n = 0$ if $x_n = b$.
    Show that $\sum_{n=1}^{\infty} a_n \leq f(b) - f(a)$ and $\sum_{n=1}^{\infty} b_n \leq f(b) - f(a)$.
\end{exercise}

\begin{solution}
    
    We will first use exercise 33 (without the absolute values since we have an \textit{increasing} function) to obtain an intermediate result about the first $n$ terms of the sums that appear in this exercise.
    Namely, applying exercise 33 on the points $(x_n)$ we have that:
    $$\sum_{i=1}^{n} f(x_i+) - f(x_i-) \leq f(b) - f(a) \implies \sum_{i=1}^{n} f(x_i+) - f(x_i) + f(x_i) - f(x_i-) \leq f(b) - f(a)$$
    $$\implies \sum_{i=1}^{n} f(x_i+) - f(x_i) + \sum_{i=1}^{n} f(x_i) - f(x_i-) \leq f(b) - f(a) \implies \sum_{i=1}^{n} b_i + \sum_{i=1}^{n} a_i \leq f(b) - f(a)$$

    Since we are dealing with non-negative quantities from this we can conclude that $$\sum_{i=1}^{n} b_i \leq f(b) - f(a), \sum_{i=1}^{n} a_i \leq f(b) - f(a)$$

    Observe therefore that if the left or right discontinuities are finite in number, this result proves what is needed in the exercise (for $a_i$ or $b_i$ respectively).
    We now examine the case where the left discontinuities are infinite in number.
    Observe that each $a_n$ is non-negative, and therefore the sequence of partial sums is non-decreasing, while we've already shown that it is bounded.
    We thus know that it converges, and since limits maintain non-strict inequalities, we obtain that $\sum_{n=1}^{\infty} a_n \leq f(b) - f(a)$.
    The exact same reasoning can also be applied to right discontinuities.
\end{solution}

\begin{exercise}{36}
    In the notation of Exercise 35, define $h(x) = \sum_{\{n: x_n \leq x\}} a_n + \sum_{\{n: x_n < x\}} b_n$.
    Show that $h$ is increasing and that $g = f - h$ is both continuous and increasing.
    Thus, each increasing function $f$ can be written as the sum of a continuous increasing function $g$ and a ``pure jump'' function $h$.
\end{exercise}

\begin{solution}
    
    We begin by showing that $h$ is increasing.
    Consider $y_1, y_2 \in [a, b], y_1 < y_2$.
    Then:
    $$h(y_2) - h(y_1) = \sum_{\{n: x_n \leq y_2\}} a_n + \sum_{\{n: x_n < y_2\}} b_n - \sum_{\{n: x_n \leq y_1\}} a_n - \sum_{\{n: x_n < y_1\}} b_n$$
    $$ = \sum_{\{n: y_1 < x_n \leq y_2\}} a_n + \sum_{\{n: y_1 \leq x_n < y_2\}} b_n \geq 0,$$
    since each term of these sums is non-negative (recall that $f$ is increasing).
    Therefore $h$ is increasing.
    Now we examine the continuity of $g$, for which we'll have to examine the left and right limits at the points of discontinuity of $f$ and at the points where $f$ is continuous.

    First let's consider $x_i \in \{x_1, x_2, \ldots\}$, i.e.\ a point of discontinuity of $f$:
    \begin{itemize}
        \item For $y \rightarrow x_i+$, we are interested in the following limit:
        $$\lim_{y \rightarrow x_i+} g(y) = \lim_{y \rightarrow x_i+} (f(y) - h(y)) = \lim_{y \rightarrow x_i+} \Bigl(f(y) - \sum_{\{n: x_n \leq y\}} a_n - \sum_{\{n: x_n < y\}}b_n\Bigr)$$
        Now notice that in this case it is always true that $y > x_i$, leading us to decompose the sums as follows:
        $$f(y) - \sum_{\{n: x_n \leq x_i\}}a_n - \sum_{\{n: x_i < x_n \leq y\}} a_n - \sum_{\{n: x_n < x_i\}} b_n - \sum_{\{n: x_i \leq x_n < y\}} b_n$$
        $$ = f(y) - h(x_i) - \Bigl(\sum_{\{n: x_i < x_n \leq y\}} a_n + \sum_{\{n: x_i \leq x_n < y\}} b_n\Bigr)$$
        Now we observe that since we saw in exercise 35 that the series of the non-negative $(a_n)$ converges, the sum of $a_n$ inside the parenthesis can be made to go to zero as $y \rightarrow x_i+$: the argument here is the same as in exercise 34, where we show that we need but to exclude a finite number of terms of the series.
        The same applies to the sum of $b_n$, except that it tends to $b_i$ instead (due to the ``less than or equal to $x_i$'' in its sum, where $x_i$ does correspond to a term of the sequence).
        Consequently, the limit of this expression as $y \rightarrow x_i+$ must be:
        $$f(x_i+) - h(x_i) - 0 - b_i = f(x_i+) - h(x_i) - f(x_i+) + f(x_i) = f(x_i) - h(x_i),$$
        where the first term comes from the definition of the right limit of $f$ at $x_i$, the second term is constant and the third and fourth were explained above.
        Therefore we conclude that $\lim_{y \rightarrow x_i+} g(y) = f(x_i) - h(x_i) = g(x_i)$.
        \item For $y \rightarrow x_i-$, the proof will be highly similar:
        $$\lim_{y \rightarrow x_i-} g(y) = \lim_{y \rightarrow x_i-} \Bigl( f(y) - \sum_{\{n: x_n \leq y\}} a_n - \sum_{\{n: x_n < y\}} b_n \Bigr)$$
        The decomposition now looks like:
        $$f(y) - \sum_{\{n: x_n \leq x_i\}}a_n + \sum_{\{n: y < x_n \leq x_i\}} a_n - \sum_{\{n: x_n < x_i\}} b_n + \sum_{\{n: y \leq x_n < x_i\}} b_n$$
        $$ = f(y) - h(x_i) + \Bigl(\sum_{\{n: y < x_n \leq x_i\}} a_n + \sum_{\{n: y \leq x_n < x_i\}} b_n\Bigr)$$
        A similar argument as above yields that the first sum here tends to $a_i$ and the second to zero.
        Therefore the desired limit will equal:
        $$f(x_i-) - h(x_i) + a_i = f(x_i-) - h(x_i) + f(x_i) - f(x_i-) = f(x_i) - h(x_i),$$
        which is again the desired quantity.
    \end{itemize}
    Now we consider $x \notin \{x_1, x_2, \ldots\}$:
    \begin{itemize}
        \item For $y \rightarrow x+$, we are interested in:
        $$\lim_{y \rightarrow x+}g(y) = \lim_{y \rightarrow x} (f(y) - h(y))$$
        Carrying out the calculations will lead to an expression identical to the first bullet above, except that since $x \notin \{x_1, x_2, \ldots\}$ all sums that involve $a_n, b_n$ will now go to zero (unlike before, they never contain a ``fixed'' term corresponding to $x_i$).
        Therefore we'll eventually arrive at $\lim_{y \rightarrow x+} g(y) = f(x+) - h(x) = f(x) - h(x)$, since $f$ is now continuous at $x$.
        \item For $y \rightarrow x-$, everything is exactly symmetrical to the above, and therefore once more $\lim_{y \rightarrow x-} g(y) = f(x) - h(x)$.
    \end{itemize}
    This completes the proof that $g$ is in fact continuous in all of $[a, b]$.

    All that remains is to show that $g$ is increasing as well.
    Take $y_1, y_2 \in [a, b]$ with $y_1 < y_2$.
    We want to show that $g(y_1) \leq g(y_2)$.
    We have that:
    $$g(y_2) - g(y_1) = f(y_2) - h(y_2) - f(y_1) + h(y_1) = f(y_2) - f(y_1) - (h(y_2) - h(y_1))$$
    We examine $h(y_2) - h(y_1)$:
    $$h(y_2) - h(y_1) = \sum_{\{n: x_n \leq y_2\}} a_n + \sum_{\{n: x_n < y_2\}} b_n - \sum_{\{n: x_n \leq y_1\}} a_n - \sum_{\{n: x_n < y_1\}} b_n$$
    $$= \sum_{\{n: y_1 < x_n \leq y_2\}} a_n + \sum_{\{n: y_1 \leq x_n < y_2\}} b_n$$
    Notice that every term that appears in this sum corresponds to a $x_n \in [y_1, y_2]$.
    Recall also from exercise 35 that we had $\sum_{i=1}^{n} b_i + \sum_{i = 1}^{n} a_i \leq f(b) - f(a)$ and that taking the limit of each of the two sums was well-defined, which means that this inequality as well holds at the limit.
    The only detail that may potentially be problematic here is that we may also be summing over the terms $f(y_1+) - f(y_1), f(y_2) - f(y_2-)$ in case one or both endpoints $y_1, y_2$ are also discontinuities.
    However, it can be shown fairly easily that with a small modification to 33 the inequality given there can also be shown to hold for $n$ points in the closed interval [a, b] instead of just the open (one just has to take a finite number of $x_1, \ldots, x_n$ and consider $a < a' < x_1, x_n < b' < b$).
    Then 35 can be applied to generalize this to countably infinite points which may be the case here.
    
    All of this results in the fact that the sum above is at most $f(y_2) - f(y_1)$ (since these are the endpoints of the relevant interval), and therefore also $g(y_2) \geq g(y_1)$, which shows that $g$ is increasing.
\end{solution}