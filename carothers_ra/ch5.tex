\chapter{Open Sets and Closed Sets}

\section{Continuous Functions}

\begin{exercise}{4}
    Show that $\mathcal{X}_\Delta : \mathbb{R} \rightarrow \mathbb{R}$, the characteristic function of the Cantor set, is discontinuous at each point of $\Delta$.
\end{exercise}

\begin{solution}
    
    We will do this by contradiction.
    Suppose $\epsilon = 1/2$ and $x \in \Delta$.
    If $\mathcal{X}_\Delta$ is continuous at $x$, then there exists $\delta > 0$ such that whenever $\lvert y - x \rvert < \delta$ it holds that $\lvert \mathcal{X}_\Delta(y) -\mathcal{X}_\Delta(x) \rvert < \epsilon$.
    Pick any such $y$.
    By definition of the characteristic function, this implies that $\mathcal{X}_\Delta(y) = 1$, and so that $y \in \Delta$.
    But then by exercise 22 of Chapter 2 (page 29 on the book), we know that for our $x, y \in \Delta$, where either $x < y$ or $y < x$, there exists $x < z < y$ or $y < z < x$ respectively such that $z \notin \Delta$.
    The consequence of this is that $\lvert z - x \rvert < \delta$ but $\mathcal{X}_\Delta(z) = 0$, and so $\lvert \mathcal{X}_\Delta(z) - \mathcal{X}_\Delta(x) \rvert = 1 > 1/2$, a contradiction.
    Therefore $\mathcal{X}_\Delta$ is not continuous at $x$.
\end{solution}

\begin{exercise}{7}
    (a) If $f: M \rightarrow \mathbb{R}$ is continuous and $a \in \mathbb{R}$, show that the sets $\{x: f(x) > a\}$ and $\{x: f(x) < a\}$ are open subsets of $M$.

    (b) Conversely, if the sets $\{x: f(x) > a\}$ and $\{x: f(x) < a\}$ are open for every $a \in \mathbb{R}$, show that $f$ is continuous.

    (c) Show that $f$ is continuous even if we assume only that the sets $\{x: f(x) > a\}$ and $\{x: f(x) < a\}$ are open for every \textit{rational} $a$.
\end{exercise}

\begin{solution}
    
    (a) For any $a \in \mathbb{R}$, consider the set $S = \{y \in \mathbb{R} \mid y > a\}$.
    This is clearly seen to be an open set.
    Then, theorem 5.1 (equivalent definitions of continuity) guarantees that $f^{-1}(S)$ is open in $M$.
    This set equals $f^{-1}(S) = \{x \in M \mid f(x) \in S\} = \{x \in M \mid f(x) > a\}$.
    The same holds if ``$>$'' is replaced by ``$<$'', thus completing the proof.

    (b) Suppose $N$ is any open subset of $\mathbb{R}$.
    If we can show that $f^{-1}(N)$ is open in $M$, then we will have equivalently shown that $f$ is continuous.
    We know (theorem 4.6) that $N$ can be written as a countable union of disjoint open intervals:

    \[N = \bigcup_{n=1}^{\infty} I_n, I_n = (a_n, b_n),\]

    where $I_n \cap I_m = \emptyset$ for $n \neq m$.
    Then we have that $f^{-1}(N) = \bigcup_{n=1}^{\infty} f^{-1}(I_n)$.
    Note that in general the boundaries of these intervals may equal the two infinities.    
    First, in the special case where this union is simply the interval $(-\infty, \infty)$ (i.e., the chosen $N$ equals $\mathbb{R}$), we can immediately conclude that $f^{-1}(N) = M$: if this was not the case, then there would exist some $x \in M$ such that $f(x) \notin N = \mathbb{R}$, which is clearly impossible.
    Note that, trivially, $f^{-1}(N)$ is then open (example 4.1 (a), page 51).
    In every other case, no interval $I_n$ is of this form, and $N$ is a union of \textit{at least one} interval $(a_n, b_n)$ where not both of the endpoints are infinities.
    We are now interested in $f^{-1}(I_n)$.
    If both $a_n, b_n$ are real numbers, write $I_n = (-\infty, b_n) \cap (a_n, \infty)$.
    If not, proceed with $I_n = (-\infty, b_n)$ or $I_n = (a_n, \infty)$.
    In the last two cases, one easily sees that $f^{-1}(N) = \{x \in M: f(x) < b_n\}, f^{-1}(N) = \{x \in M: f(x) > a_n\}$ respectively, both of which are open by our hypothesis.
    In the first case, $f^{-1}(N)$ equals a finite intersection of open sets (by utilizing what we proved just now), and as such is again open.
    Then we conclude that $f^{-1}(N) = \bigcup_{n=1}^{\infty} f^{-1}(I_n)$ is also open as an arbitrary union of open sets, thus also obtaining that $f$ is continuous.

    (c) The proof here relies on exercise 7 of chapter 4 (page 55), which states that every open set in $\mathbb{R}$ can be written as the union of countably many open intervals with rational endpoints.
    Since disjointness was never necessary in (b), everything else follows in exactly the same way, because we know that $\{x: f(x) > a\}, \{x: f(x) < a\}$ are open for rational $a$, and every interval we will encounter here will be of this form.
    Therefore, $f$ is again continuous.
\end{solution}

\begin{exercise}{8}
    Let $f: \mathbb{R} \rightarrow \mathbb{R}$ be continuous.

    (a) If $f(0) > 0$, show that $f(x) > 0$ for all $x$ in some interval $(-a, a)$.

    (b) If $f(x) \geq 0$ for every rational $x$, show that $f(x) \geq 0$ for all real $x$.
    Will this result hold with ``$\geq 0$'' replaced by ``$> 0$''?
    Explain.
\end{exercise}

\begin{solution}
    
    (a) Suppose $f(x) = c > 0$, and set $\epsilon = c/2 > 0$.
    Due to the continuity of $f$, there exists $\delta > 0$ such that whenever $\lvert x - 0 \rvert < \delta$, it holds that $\lvert f(x) - f(0) \rvert < \epsilon$.
    This implies that $-c/2 < f(x) - c < c/2 \implies c/2 < f(x) < 3c/2$.
    But then setting $a = \delta$ means that for $x \in (-a, a)$ it holds that $f(x) > 0$.

    (b) Pick any $y \in \mathbb{R}$.
    We know that we can always find a sequence $(x_n) \subset \mathbb{Q}$ such that $x_n \rightarrow y$.
    By the continuity of $f$, it must hold that $f(x_n) \rightarrow f(y)$.
    Suppose now that $f(y) < 0$, and set $\epsilon = -f(y)/2$.
    Then we can find $N > 0$ such that whenever $n \geq N$ it holds that:

    \[\lvert f(x_n) - f(y) \rvert < \epsilon \implies f(y)/2 < f(x_n) - f(y) < -f(y)/2 \implies f(x_n) < f(y)/2 < 0,\]

    which is a contradiction.
    Therefore it must hold that $f(y) \geq 0$ for every $y \in \mathbb{R}$.
    Now for the second part, consider the function $f(x) = (x - \sqrt{2})^2$.
    Clearly, $f$ is continuous and strictly positive for all $x \in \mathbb{R} \setminus \{\sqrt{2}\}$, which means that it is more specifically continuous for all $x \in \mathbb{Q}$.
    However, $f(\sqrt{2}) = 0$, which is not strictly positive, and thus the assertion that ``$f(x) > 0$ for every rational $x$ implies that $f(x) > 0$ for every real $x$'' fails.
\end{solution}

\begin{exercise}{9}
    Let $A \subset M$.
    Show that $f: (A, d) \rightarrow (N, \rho)$ is continuous at $a \in A$ if and only if, given $\epsilon > 0$, there is a $\delta > 0$ such that $\rho(f(x), f(a)) < \epsilon$ whenever $d(x, a) < \delta$ \textit{and} $x \in A$.
    We paraphrase this statement by saying that ``$f$ has a point of continuity relative to $A$''.
\end{exercise}

\begin{solution}
    
    Recall the definition of continuity in a metric space $M$: $f: (M, d) \rightarrow (N, \rho)$ is continuous at $a \in M$ if for every $\epsilon > 0$ there is a $\delta > 0$ such that whenever $d(x, y) < \delta$ it holds that $\rho(f(x), f(y)) < \epsilon$.
    We know that if $M$ is a metric space, $A \subset M$ is also a metric space.
    Therefore, the definition of continuity for functions $f: (A, d) \rightarrow (N, \rho)$ states that $f$ is continuous at $a \in A$ precisely when, for every $\epsilon > 0$, there exists $\delta > 0$ such that $\rho(f(x), f(a)) < \epsilon$ whenever $d(x, a) < \delta$ for $x$ inside the metric space, which in this case is $A$.
\end{solution}

\begin{exercise}{10}
    Let $A = (0, 1] \cup \{2\}$, considered as a subset of $\mathbb{R}$.
    Show that every function $f: A \rightarrow \mathbb{R}$ is continuous, relative to $A$, at 2.
\end{exercise}

\begin{solution}
    
    Pick any $\epsilon > 0$, and set e.g. $\delta = 1/3$.
    Observe then that $B_{1/3}^{A}(2) = \{2\}$.
    Therefore, for every $x \in A$ such that $\lvert x - 2 \rvert < 1/3$, it trivially holds that $\lvert f(x) - f(2) \rvert < \epsilon$, and so $f$ is continuous at 2 relative to $A$.
\end{solution}

\begin{exercise}{14}
    A continuous function on $\mathbb{R}$ is completely determined by its values on $\mathbb{Q}$.
    Use this to ``count'' the continuous functions $f: \mathbb{R} \rightarrow \mathbb{R}$.
\end{exercise}

\begin{solution}
    
    First of all, to see that a continuous function is indeed determined by its values on $\mathbb{Q}$, suppose that $f, g: \mathbb{R} \rightarrow \mathbb{R}$ are continuous and $f(x) = g(x)$ for all $x \in \mathbb{Q}$.
    Then if $x \in \mathbb{R} \setminus \mathbb{Q}$, we know that there exists a sequence $(x_n) \subset \mathbb{Q}$ such that $x_n \rightarrow x$.
    It is the case that $f(x_n) = g(x_n)$ for all $n$, and by the continuity of $f, g$ we have that $f(x) = \lim_{n \rightarrow \infty} f(x_n) = \lim_{n \rightarrow \infty} g(x_n) = g(x)$, therefore $f = g$.

    We begin our proof by noticing that the above observation implies that there exists an injection $F: C^0 \rightarrow \mathbb{R}^{\mathbb{Q}}$.

    Furthermore, there exists a trivial injection $G$ from $\mathbb{R}$ to $C^0$: map any real number $a$ to the clearly continuous $f(x) = a$.

    Therefore, if we can now show that $\mathbb{R}^{\mathbb{Q}} \sim \mathbb{R}$, the existence of $F$ guarantees the existence of an injection from the set of continuous functions to $\mathbb{R}$ as well.
    This injection coupled with $G$ will allow us to apply the Cantor-Schröder-Bernstein theorem to obtain that in fact $C^0 \sim \mathbb{R}$.
    In showing $\mathbb{R}^{\mathbb{Q}} \sim \mathbb{R}$, the following lemma will be useful: 
    
    If $A \sim B$, and $C$ is countable, then $A^{C} \sim B^{C}$.
    Indeed, we can construct a bijection $F: A^{C} \rightarrow B^{C}$ in the following manner.
    First, let $h$ be a bijection from $A$ to $B$, which we know exists.
    If $f \in A^{C}$, we have $f(c_1) = a_1, f(c_2) = a_2, \ldots$.
    We obtain $b_1, b_2, \ldots$ by $h(a_i) = b_i$, and we define $g \in B^{C}$ as $g(c_1) = b_1, g(c_2) = b_2, \ldots$.
    We then define $F(f) = g$.
    To prove injectivity, suppose $F(f_1) = F(f_2)$ for $f_1 \neq f_2$.
    This means that there exists $c_n \in C$ such that $f_1(c_n) \neq f_2(c_n)$, and so $h(f_1(c_n)) = F(f_1)(c_n) \neq h(f_2(c_n)) = F(f_2)(c_n)$, due to the injectivity of $h$.
    But then $F(f_1) \neq F(f_2)$, a contradiction.
    For surjectivity, pick $g \in B^{C}$, which is completely determined by the values $g(c_1), g(c_2), \ldots$.
    Let then $a_n \in A, a_n = h^{-1}(g(c_n))$, which are well-defined since $h$ is a bijection.
    Then by defining $f \in A^{C}$ such that $f(c_n) = a_n$ we have $F(f) = g$, and thus $F$ is also a surjection, and therefore a bijection.

    Notice now the following.
    We know $\mathbb{R} \sim [0, 1]$, and by the lemma above we then have $\mathbb{R}^\mathbb{Q} \sim [0, 1]^\mathbb{Q}$ since $\mathbb{Q}$ is countable.
    For the remaining proof, we can thus work with $[0, 1]^\mathbb{Q}$.
    
    Pick any $f \in [0, 1]^{\mathbb{Q}}$, and let $f(q_n) = a_n \in [0, 1]$, where we write $a_n$ in binary form, and if it can be written as both ending in infinite 1s and in infinite 0s, we choose the latter.
    In a manner similar to Cantor's diagonal argument, we construct the following list:

    \[f(q_1) = a_{11}.a_{12}a_{13}\ldots\]
    \[f(q_2) = a_{21}.a_{22}a_{23} \ldots\]
    \[\vdots\]

    Define then $b_f = 0.a_{11}a_{21}a_{12}a_{31}a_{22}a_{13}\ldots \in [0, 1]$.
    We claim that the function $F: [0, 1]^{\mathbb{Q}} \rightarrow [0, 1]$ defined by $F(f) = b_f$ is an injection.
    Indeed, if $b_f = b_g$ for two such $f, g$, if $b_f, b_g$ are equal on all corresponding digits, then $f = g$ trivially.
    If not, then one of them, say $b_f$, ends in infinite ones, and the other in infinite zeros.
    However, this would imply that the construction above yields a $b_f$ ending in infinite ones, and thus that $f(q_i)$ end in infinite ones, which is impossible due to how we chose to construct these $b$.
    Therefore $F$ is injective.
    We now also find an injection $g: [0, 1] \rightarrow [0, 1]^{\mathbb{Q}}$.
    For $a \in [0, 1]$, write $a = a_1.a_2 a_3 \ldots$ in binary form, again choosing infinite zeros over infinite ones as a postfix whenever necessary.
    We then define $g(a) = f$ such that $f(q_i) = 0.0\ldots 0 a_i$, where the prefix is of length $i - 1$ (for $f(q_1)$, we have $f(q_1) = a_1$).
    Notice again that this is easily seen to be an injection, since $g(a) = g(b)$ implies that all corresponding digits of $a, b$ are equal.
    By the Cantor-Schröder-Bernstein theorem, we then have that $[0, 1]^{\mathbb{Q}} \sim [0, 1]$, and thus that $\mathbb{R}^\mathbb{Q} \sim \mathbb{R}$, i.e. there are ``as many'' continuous functions as real numbers.

\end{solution}

\begin{exercise}{17}
    Let $f, g: (M, d) \rightarrow (N, \rho)$ be continuous, and let $D$ be a dense subset of $M$.
    If $f(x) = g(x)$ for all $x \in D$, show that $f(x) = g(x)$ for all $x \in M$.
    If $f$ is onto, show that $f(D)$ is dense in $N$.
\end{exercise}

\begin{solution}
    
    Suppose that there exists $x \in M \setminus D$ such that $f(x) \neq g(x)$, which means that $\rho(f(x), g(x)) = \epsilon > 0$.
    By the denseness of $D$, there exists a sequence $(x_n) \subset D, x_n \rightarrow x$.
    Because $f, g$ are both continuous, it holds that $f(x_n) \rightarrow f(x)$ and $g(x_n) \rightarrow g(x)$.
    Therefore, there exist $N_1, N_2 > 0$ such that for $n \geq \max\{N_1, N_2\}$ it holds that $\rho(f(x_n), f(x)) < \frac{\epsilon}{2}, \rho(g(x_n), g(x)) < \frac{\epsilon}{2}$.
    Notice, furthermore, that $f(x_n) = g(x_n)$ since the two functions are equal on $D$.
    Then, by the triangle inequality we obtain that:

    \[\epsilon = \rho(f(x), g(x)) \leq \rho(f(x), f(x_n)) + \rho(f(x_n), g(x)) = \rho(f(x), f(x_n)) + \rho(g(x_n), g(x)) < \frac{\epsilon}{2} + \frac{\epsilon}{2}\]
    \[\implies \epsilon < \epsilon,\]

    which is a contradiction.
    Therefore $f(x) = g(x)$ on all of $M$.

    Now suppose that $f$ is onto, and pick any $y \in N$.
    Then there exists $x \in M$ such that $f(x) = y$.
    Furthermore, by the denseness of $D$ there exists $(x_n) \subset D$ such that $x_n \rightarrow x$.
    The continuity of $f$ then implies $f(x_n) \rightarrow f(x) = y$.
    But $f(x_n)$ is a sequence entirely in $f(D)$, which means that $f(D)$ is dense in $N$, since $y$ was picked arbitrarily in $N$.
\end{solution}

\begin{exercise}{18}
   Let $f: (M, d) \rightarrow (N, \rho)$ be continuous, and let $A$ be a separable subset of $M$.
   Prove that $f(A)$ is separable. 
\end{exercise}

\begin{solution}
    
    We need to find a countable dense subset of $f(A)$.
    Since $A$ is separable, it countains a countable and dense subset $S \subset A, S = \{a_1, a_2, \ldots\}$.
    Pick any element $y \in f(A)$, which means there exists $a \in A$ such that $y = f(a)$.
    Since $S$ is dense, there exists a sequence $(a_n) \subset A$ such that $a_n \rightarrow a$, and because $f$ is continuous, $f(a_n) \rightarrow f(a)$.
    Notice that this holds for any $y \in f(A)$.
    The consequence is that if we form the set $S' = \{f(a_1), f(a_2), \ldots\}$, which is clearly countable because $S$ is countable, we are always able to find a sequence in $s'$ converging to any element of $A$.
    Therefore $f(A)$ is separable.
\end{solution}

\begin{exercise}{19}
    A function $f: \mathbb{R} \rightarrow \mathbb{R}$ is said to satisfy a \textbf{Lipschitz condition} if there is a constant $K< \infty$ such that $\lvert f(x) - f(y) \rvert \leq K \lvert x - y \rvert$ for all $x, y \in \mathbb{R}$.
    More economically, we may say that $f$ is Lipschitz (or Lipschitz with constant $K$ if a particular constant seems to matter).
    Show that $\sin x$ is Lipschitz with constant $K = 1$.
    Prove that a Lipschitz function is (uniformly) continuous.
\end{exercise}

\begin{solution}
    
    Pick any two $x, y \in \mathbb{R}$, and let $h = x - y$.
    Then we have that:

    \[\lvert f(x) - f(y) \rvert = \lvert \sin(x) - \sin(y) \rvert = \lvert \sin(y + h) - \sin(y) \rvert = \Bigl\lvert 2 \sin(\frac{h}{2})\cos(\frac{y + y + h}{2}) \Bigr\rvert\]
    \[\leq \Bigl\lvert 2\sin(\frac{h}{2}) \Bigr\rvert \leq 2 \Bigl\lvert \frac{h}{2} \Bigr\rvert \leq \lvert h \rvert = \lvert x - y \rvert,\]

    where we used the identity $\sin(a) - \sin(b) = 2\sin((a - b)/2)\cos((a + b)/2)$, the fact that $\lvert \cos(a) \rvert \leq 1$ for any $a$, and the inequality $\lvert \sin(a) \rvert \leq \lvert a \rvert$ for any $a$.
    We've thus shown that $\sin$ is Lipschitz continuous with $K = 1$.
    Now we want to show that any Lipschitz continuous function (with constant $K$) is uniformly continuous.
    Pick any $\epsilon$, and set $\delta = \frac{\epsilon}{K}$.
    Then observe that, whenever $\lvert x - y \rvert < \delta$:

    \[\lvert f(x) - f(y) \rvert \leq K \lvert x - y \rvert < K \delta = K \frac{\epsilon}{K} = \epsilon,\]

    which is precisely the definition of uniform continuity for $f$.
\end{solution}

\begin{exercise}{20}
    If $d$ is a metric on $M$, show that $\lvert d(x, z) - d(y, z) \rvert \leq d(x, y)$ and conclude that the function $f(x) = d(x, z)$ is continuous on $M$ for any fixed $z \in M$.
    This says that $d(x, y)$ is \textit{separately continuous} - continuous in each variable separately.
\end{exercise}

\begin{solution}
    
    We have already shown this inequality in exercise 2 of chapter 3.
    What we are then asked to show is that for any \textit{fixed} $z \in M$, and for any $x \in M$, the function $f(x) = d(x, z)$ is continuous at $x$.
    To do this, pick any $\epsilon > 0$.
    We need to find $\delta > 0$ such that whenever $d(x, y) < \delta$ it holds that $\lvert f(y) - f(x) \rvert < \epsilon$ (note that the metric space containing the codomain is simply $\mathbb{R}$).
    Observe that:

    \[\lvert f(y) - f(x) \rvert = \lvert d(y, z) - d(x, z) \rvert \leq d(x, y)\]

    Therefore, if we set $\delta = \epsilon/2$, we obtain that $\lvert f(y) - f(x) \rvert \leq \epsilon/2 < \epsilon$, and thus that $f$ is continuous on $M$.
\end{solution}

\begin{exercise}{23}
    Define $S: c_0 \rightarrow c_0$ by $S(x_1, x_2, \ldots) = (0, x_1, x_2, \ldots)$.
    That is, $S$ shifts the entries forward and puts 0 in the empty slot.
    Show that $S$ is an isometry (into).
\end{exercise}

\begin{solution}
    
    We need to show that for any two sequences $(x_n), (y_n) \in c_0, d(S((x_n)), S((y_n))) = d((x_n), (y_n))$, where $d((x_n), (y_n)) = \sup_{n} \lvert x_n - y_n \rvert$.
    We notice that:

    \[d(S((x_n)), S((y_n))) = \sup_{n} \{ \lvert 0 - 0 \rvert, \lvert x_1 - y_1 \rvert, \lvert x_2 - y_2 \rvert, \ldots\} = \sup_{n} \{ \lvert x_1 - y_1 \rvert , \lvert x_2 - y_2 \rvert, \ldots \},\]

    since every term here is non-negative.
    This last quantity equals precisely $d((x_n), (y_n))$, which completes the proofthat $S$ is an isometry.
\end{solution}

\begin{exercise}{24}
    Let $V$ be a normed vector space.
    If $y \in V$ is fixed, show that the maps $a \mapsto a y$ from $\mathbb{R}$ to $V$ and $x \mapsto x + y$ from $V$ to $V$ are continuous.
\end{exercise}

\begin{solution}
    
    Let $g: \mathbb{R} \rightarrow V, g(a) = ay$ for a fixed $y \in V$.
    Note that $g$ is obviously continuous if $y = 0$, so from now on assume $y \neq 0$.
    We need to show that for any $a \in \mathbb{R}$, it holds that for any $\epsilon > 0$, there exists $\delta > 0$ such that whenever $\lvert b - a \rvert < \delta$, it holds that $\lvert \lvert g(b) - g(a) \rvert \rvert < \epsilon$ (for the usual metric on $V$).
    By the properties of normed vector spaces, we have that:

    \[\lvert \lvert g(b) - g(a) \rvert \rvert = \lvert \lvert by - ba \rvert \rvert = \lvert b - a \rvert \cdot \lvert \lvert y \rvert \rvert \]

    For a given $\epsilon > 0$, set $\delta = \frac{\epsilon}{\lvert \lvert y \rvert \rvert}$.
    Then, whenever $\lvert b - a \rvert < \delta$, the equality above shows that $\lvert \lvert g(b) - g(a) \rvert \rvert < \delta \lvert \lvert y \rvert \rvert = \epsilon$, i.e., that $g$ is indeed continuous.

    Now let instead $g: V \rightarrow V$ be $g(x) = x + y$ for a fixed $y \in V$.
    For any $x, z \in V$ we have that:

    \[\lvert \lvert g(z) - g(x) \rvert \rvert = \lvert \lvert (z + y) - (x + y) \rvert \rvert = \lvert \lvert z - x \rvert \rvert\]

    Therefore, for any given $\epsilon > 0$, if we set $\delta = \epsilon$, whenever $\lvert \lvert z - x \rvert \rvert < \delta$, the equality above tells us that $\lvert \lvert g(z) - g(x) \rvert \rvert < \delta = \epsilon$, and so that $g$ is continuous on any $x \in V$.
\end{solution}

\begin{exercise}{25}
    A function $f: (M, d) \rightarrow (N, \rho)$ is called \textbf{Lipschitz} if there is a constant $K$ such that $\rho(f(x), f(y)) \leq K d(x, y)$ for all $x, y \in M$.
    Prove that a Lipschitz mapping is continuous.
\end{exercise}

\begin{solution}
    
    Suppose first that $K = 0$.
    Then for any two $x, y \in M$ it holds that $\rho(f(x), f(y)) \leq 0$.
    By the properties of metrics, we know that this means that $\rho(f(x), f(y)) = 0$, and this in turn means that $f(x) = f(y)$ for all $x, y$, thus $f$ is constant, and trivially continuous.

    If $K \neq 0$, pick any $x \in M$ and any $\epsilon > 0$.
    Set $\delta = \frac{\epsilon}{K}$, and observe that whenever $d(x, y) < \delta$, we have that:

    \[\rho(f(x), f(y)) \leq K d(x, y) < K \delta = K \frac{\epsilon}{K} = \epsilon,\]

    which is precisely the definition of continuity for $f$.
\end{solution}

\begin{exercise}{26}
    Provide the answer to a question raised in Chapter Three by showing that integration is continuous.
    specifically, show that the map $L(f) = \int_{a}^{b} f(t) dt$ is Lipschitz with constant $K = b - a$ for $f \in C[a, b]$.
\end{exercise}

\begin{solution}
    
    We recall first the Frechet metric for continuous functions on $[a, b]: d(f, g) = \max_{a \leq t \leq b} \lvert f(t) - g(t) \rvert$.
    The question raised in the beginning of Chapter Three was whether integration is continuous, which, in other words, would mean that the quantities $L(f), L(g)$ are ``close'' whenever $f, g$ are close.
    To answer this, we observe that:

    \[\lvert L(f) - L(g) \rvert = \Biggl\lvert \int_{a}^{b} f(t) dt - \int_{a}^{b} g(t) dt \Biggr\rvert = \Biggl\lvert \int_{a}^{b} (f(t) - g(t)) dt \Biggr\rvert\]

    Now we call the mean value theorem for integration: there exists $c \in (a, b)$ such that $\int_{a}^{b}(f(t) - g(t))dt = (f(c) - g(c))(b - a)$.
    Therefore:

    \[\lvert L(f) - L(g) \rvert = \lvert (f(c) - g(c))(b - a) \rvert = (b - a) \lvert f(c) - g(c) \rvert \leq (b - a) \max_{a \leq t \leq b} \lvert f(t) - g(t) \rvert = (b - a)d(f, g)\]
    
    We've therefore shown that $\lvert L(f) - L(g) \rvert \leq (b - a)d(f, g)$, which means precisely that $L$ is Lipschitz with $K = b - a$.
\end{solution}

\begin{exercise}{28}
    Define $g: l_2 \rightarrow \mathbb{R}$ by $g(x) = \sum_{n=1}^{\infty} x_n/n$.
    Is $g$ continuous?
\end{exercise}

\begin{solution}
    
    We need to examine whether it holds that whenever $x^{(k)} \rightarrow x$ (these are both sequences of sequences), $g(x^{(k)}) \rightarrow g(x)$.
    We have the following:

    \[ \lvert g(x^{(k)}) - g(x) \rvert = \lvert \sum_{n=1}^{\infty} \frac{x_n^k}{n} - \sum_{n=1}^{\infty} \frac{x_n}{n} \rvert = \lvert \sum_{n=1}^{\infty} \frac{x_n^k - x_n}{n} \rvert \]

    Consider the sequence $y = (\frac{x_1}{1}, \frac{x_2}{2}, \ldots)$, and the sequence of sequences $y^{(k)}$ such that $y_k = (\frac{x_1^k}{1}, \frac{x_2^k}{2}, \ldots)$.
    These sequences are also in $l_2$, and then from a previous result recall that they are thus in $l_1$ as well, and in fact:


\end{solution}

\begin{exercise}{30}
    Let $f: (M, d) \rightarrow (N, \rho)$.
    Prove that $f$ is continuous if and only if $f(\overline{A}) \subset \overline{f(A)}$ for every $A \subset M$ if and only if $f^{-1}(\mathring{B}) \subset (\mathring{f^{-1}(B)})$ for every $B \subset N$.
    Give an example of a continuous $f$ such that $f(\overline{A}) \neq \overline{f(A)}$ for some $A \subset M$.
\end{exercise}

\begin{solution}
    
    For ease of notation, name the three propositions:

    (a): $f$ is continuous

    (b): $f(\overline{A}) \subset \overline{f(A)}$ for every $A \subset M$

    (c): $f^{-1}(\mathring{B}) \subset (\mathring{f^{-1}(B)})$ for every $B \subset N$

    (a) $\implies$ (b): Pick any $A \in M$ and any $y \in f(\overline{A})$.
    We need to show that $y \in \overline{f(A)}$.
    Because $y \in f(\overline{A})$, we have that there exists $x \in \overline{A}$ such that $f(x) = y$.
    We recall that $x \in \overline{A}$ is equivalent to the existence of a sequence $(x_n) \subset A$ such that $x_n \rightarrow x$.
    Because $f$ is continuous, we must then have that $f(x_n) \rightarrow f(x) = y$.
    By definition of the image of a function, $f(x_n) \in f(A)$ for every $n$.
    Once again, $f(x_n) \rightarrow y$ with $(f(x_n)) \subset f(A)$ is equivalent to $y \in \overline{f(A)}$.

    (b) $\implies$ (a): For the converse, pick any $x \in M$.
    If no sequences in $M$ converge to $x$, then there exists $\delta > 0$ such that $d(x, y) > \delta$ for every $y \in M, y \neq x$.
    Now pick any $\epsilon > 0$ and observe that $d(x, y) < \delta$ is only satisfied when $y = x$, in which case of course $\rho(f(x), f(y)) = 0 < \epsilon$, so $f$ is continuous at $x$.
    In the other case, we have that there exists at least one sequence in $M$ that converges to $x$.
    By way of contradiction, assume that there $f$ is not continuous, and thus that for some $x_n \rightarrow x$ it does not hold that $f(x_n) \rightarrow f(x)$.
    By negating the definition of the limit, we obtain that there exists $\epsilon > 0$ such that for every $N > 0$ there exists $n \geq N$ such that $\rho(f(x_n), f(x)) \geq \epsilon$.
    Gather then an infinite sequence $(f(x_i))$ for which $\rho(f(x_i), f(x)) \geq \epsilon$, and notice that $x_i \rightarrow x$ since the corresponding $x_i$ form a subsequence of $(x_n)$.
    We observe now that if $A = (x_i), \overline{A} = \{x_1, x_2, \ldots, \} \cup \{x\}$: this is due to the uniqueness of the limit, as well as the fact that $x_i \rightarrow x$.
    Then $f(x) \in f(\overline{A}) \subset \overline{f(A)}$, which in turn means that there exists a sequence $(f(y_j)) \subset f(A)$ such that $f(y_j) \rightarrow f(x)$ (by the definition of the closure).
    Since $f(A) = \{f(x_1), f(x_2), \ldots\}$, we conclude that each $f(y_j)$ must equal some $f(x_i)$.
    But then $\rho(f(y_j), f(x)) \geq \epsilon$ for every $j$, and thus $f(y_j)$ cannot convert to $x$, a contradiction.
    Therefore $f$ must be continuous.

    (a) $\implies$ (c): Pick any $B \subset N$, and any $x \in f^{-1}(\mathring{B})$.
    This means that there exists $\epsilon > 0$ such that, if $y = f(x)$, then $B_{\epsilon}^{\rho}(y) \subset B$.
    Since $f$ is continuous, there exists $\delta > 0$ such that whenever $d(x', x) < \delta$ it holds that $\rho(f(x'), y) < \epsilon$.
    Therefore, for every such $x', f(x') \in B_{\epsilon}^{\rho}(y)$, which implies $f(x') \in B$.
    We thus obtain that $x' \in f^{-1}(B)$, and since this holds for every $x'$ with $d(x', x) < \delta$, we have that $B_{\delta}^{d}(x) \subset f^{-1}(B)$, which means precisely that $x \in \mathring{(f^{-1}(B))}$.

    (c) $\implies$ (a): Suppose $B \subset N$ is open.
    Then $B = \mathring{B}$ and we have that $f^{-1}(B) = f^{-1}(\mathring{B}) \subset \mathring{(f^{-1}(B))}$.
    We know of course that $\mathring{f^{-1}(B)} \subset f^{-1}(B)$, which means $f^{-1}(B) = \mathring{f^{-1}(B)}$, which is equivalent to $f^{-1}(B)$ being open.
    Since this holds for any open $B$, we have one of the conditions that are known to be equivalent to $f$ being continuous: namely, that the inverse image of any open set under $f$ is also open.

    For the requested example, consider $M = \mathbb{N}$ equipped with the discrete metric, $N = \mathbb{R}$ and $f: M \rightarrow N, f(i) = \frac{1}{i}, i \neq 0, f(0) = 1$, which we easily see is continuous.
    Then if $A = \mathbb{N}^+$, we notice that $\overline{A} = A$ and $f(\overline{A}) = \{1, 1/2, 1/3, \ldots\}$.
    However, $\overline{f(A)} = \{0, 1, 1/2, 1/3, \ldots\}$, and so $f(\overline{A}) \neq f(\overline{A})$.

\end{solution}

\begin{exercise}{31}
    Let $f: (M, d) \rightarrow (N, \rho)$.

    (a) If $M = \bigcup_{n = 1}^{\infty} U_n$, where each $U_n$ is an open set in $M$, and if $f$ is continuous on each $U_n$ relative to that $U_n$, show that $f$ is continuous on $M$.

    (b) If $M = U_{n=1}^{N} E_n$, where each $E_n$ is a closed set in $M$, and if $f$ is continuous on each $E_n$ relative to that $E_n$, show that $f$ is continuous on M.

    (c) Give an example showing that $f$ can fail to be continuous on all of $M$ if, instead, we use a countably infinite union of closed sets $M = \bigcup_{n=1}^{\infty} E_n$ in (b).
\end{exercise}

\begin{solution}
    
    (a) Pick any $x \in M$.
    Then $x \in U_i$ for at least one $i$.
    Because $U_i$ is open, there exists $\delta_1$ such that $B_{\delta_1}^{d}(x) \subset U_i$.
    In addition, because $f$ is continuous on $x$ relative to $U_i$, for every $\epsilon > 0$ there exists $\delta_2 > 0$ such that whenever $d(x', x) < \delta_2$ and $x' \in U_i$ it holds that $\rho(f(x'), f(x)) < \epsilon$.
    Set $\delta = \min\{\delta_1, \delta_2\}$.
    Then, whenever $d(x', x) < \delta$ we have $x' \in U_i$, and also $d(x', x) < \delta_2$, thus $\rho(f(x'), f(x)) < \epsilon$.
    Therefore $f$ is indeed continuous on $x$ in the general sense.

    (b) Again, pick any $x \in M$.
    Then $x \in E_i$ for at least one $i$.
    If $x \in \mathring{E_i}$, an argument like the one in (a) serves to show that $f$ is continuous at $x$.
    The complication here is it may be the case that $x \in \partial E_i$, in which case for any open ball $B_{\delta}^{d}(x), B_{\delta}^{d}(x) \cap E_i^{c} \neq \emptyset$.
    Suppose then that $x \in \partial E_i$, pick any $\epsilon > 0$ and obtain a $\delta_i > 0$ from the definition of relative continuity to $E_i$. 
    Then set $S = B_{\delta_i}^{d}(x) \cap E_i^{c}$.
    Any point $x' \in S$ must belong in at least one $E_j, j \neq i$.
    For all such $E_j$, we examine whether their intersection with $S$ is non-empty (call the corresponding set of indices $I$), and if it isn't, we know that it must contain at least $x$.
    It must thus hold that $f$ is continuous on $x$ relative to each $E_j, j \in I$.
    From this we can obtain \textit{at most} $N - 1$ quantities $\delta_j > 0$ that satisfy the definition of relative continuity of $f$ on $x$ (with respect to each of these sets).
    The minimum of these quantities is thus well defined, and we call it $\delta'$.
    Notice then that $\delta = \min\{\delta_1, \delta'\}$ is such that whenever $d(x', x) < \delta$, it holds that $x'$ is in \textit{some} $E_i$, and relative continuity with respect to that $E_i$ guarantees that $\rho(f(x'), f(x)) < \epsilon$.
    Therefore $f$ is indeed continuous on $x$ in the general sense.

    (c)
    Consider first the following construction of subsets of $\mathbb{R}^2$.
    We define $l_i$ to be the line passing through the origin at an angle of $0 + \sum_{j=1}^{i - 1}\frac{\pi}{2^j}$ with the $x'x$ axis.
    Geometrically, this forms lines that start from an angle of 0 radians with the $x'x$ axis and tend towards an angle of $\pi$, but without ever attaining it.
    Let now $U_i$ be the subset of $\mathbb{R}^2$ contained inside $l_i, l_{i+1}$, the lines included, i.e. increasingly small ``pie slices''.
    Notice that each of these sets is closed.
    Now discard $U_i$ for $i$ even, which yields a countably infinite collection of sets $\mathcal{U}$ that are closed, and every pair of which only has the origin in its intersection.
    Define $M$ to be the union of all members of $U$, which is a metric space as a subset of $\mathbb{R}^2$, and let $N = \mathbb{R}$.
    Then define $f: M \rightarrow N$ as:

    \[f(x, y) =  i x + i y, (x, y) \in U_i\]

    This function is clearly continuous on each $U_i$ relative to $U_i$, since this restriction of it is linear (furthermore, it is well-defined since it is zero at the origin no matter which $U_i$ we examine in the formula).
    However, it fails to be continuous at (0, 0): notice that these are increasingly ``steep'' planes, each of which requires an increasingly small $\delta > 0$ in the definition of continuity for a given $\epsilon > 0$, and as such the argument used in (b) regarding a well-defined minimum $\delta$ fails.
\end{solution}

\begin{exercise}{34}
    Show that $d$ is continuous on $M \times M$, where $M \times M$ is supplied with ``the'' product metric (see Exercise 3.46).
    This says that $d$ is jointly continuous, that is, continuous as a function of two variables.
    [Hint: If $x_n \rightarrow x, y_n \rightarrow y$, show that $d(x_n, y_n) \rightarrow d(x, y)$.]
\end{exercise}

\begin{solution}
    
    By exercise 3.46, we can choose to use as the product metric the metric $d_1((x_n, y_n), (x, y)) = d(x_n, x) + d(y_n, y)$.
    Suppose that $x_n \rightarrow^{d} x, y_n \rightarrow^{d} y$, in which case we know by the same exercise that $(x_n, y_n) \rightarrow^{d_1} (x, y)$.
\end{solution}