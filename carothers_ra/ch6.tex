\chapter{Connectedness}

\section{Connected Sets}

\begin{exercise}{1}
    Supply the missing details in the proof of Lemma 6.3:

    Let $E$ be a subset of a metric space $M$.
    If $U$ and $V$ are disjoint open sets in $E$, then there are disjoint open sets $A$ and $B$ in $M$ such that $U = A \cap E, V = B \cap E$.
\end{exercise}

\begin{solution}
    
    The proof provided in the book arrives in full detail at the observation that for any two $x \in U, y \in V$ it is the case that $E \cap B_{\epsilon_x}(x) \cap B_{\epsilon_y}(y) = \emptyset$.
    Notice that this shows that if these two open balls intersect, then their intersection must be entirely contained in $M \setminus E$.
    If it was the case that they didn't intersect at all, then simply writing $A$ as the union of all $B_{\epsilon_x}(x)$ and $B$ as the union of all $B_{\delta_y}(y)$ would make these open as unions of open sets, and of course disjoint.
    The claim that is given in the proof is that one can in fact achieve this by substituting $\epsilon_x$ with $\epsilon_x/2$ and $\delta_y$ with $\delta_y/2$, i.e., it holds that $B_{\epsilon_x/2}(x) \cap B_{\delta_y/2}(y) = \emptyset$ for any two $x \in U, y \in V$.
    Suppose then that there exists $z \in M \setminus E$ such that $z \in B_{\epsilon_x/2}(x), z \in B_{\delta_y/2}$.
    We then have:

    \[d(x, y) \leq d(x, z) + d(z, y) < \frac{\epsilon_x}{2} + \frac{\delta_y}{2}\]

    WLOG, assume $\delta_y \leq \epsilon_x$, in which case we conclude $d(x, y) < 2\frac{\epsilon_x}{2} = \epsilon_x$.
    But then this means $y \in B_{\epsilon_x}(x)$, and of course $y \in E$.
    We originally had that $E \cap B_{\epsilon_x}(x) = U$, and so $y \in U$, which is a contradiction since $y \in V$ and $U \cap V = \emptyset$.
    Therefore $B_{\epsilon_x/2}(x) \cap B_{\delta_y/2}(y) = \emptyset$, and so indeed the open sets $A = \cup \{B_{\epsilon_x/2}(x), x \in U\}, B = \bigcup \{B_{\delta_y/2}(y), y \in V\}$ are disjoint.
\end{solution}

\begin{exercise}{5}
    If $E$ and $F$ are connected subsets of $M$ with $E \cap F \neq \emptyset$, show that $E \cup F$ is connected.
\end{exercise}

\begin{solution}
    
    Suppose for the sake of contradiction that $E \cup F$ is disconnected.
    Then there exist open sets $A, B, A \cap B = \emptyset$ such that $E \cup F \subset A \cup B, (E \cup F) \cap A \neq \emptyset, (E \cup F) \cap B \neq \emptyset$.
    By the first of these conditions, we obtain $E \subset A \cup B$ and $F \subset A \cup B$.
    By the second of these conditions, we obtain that at least one of $E \cap A \neq \emptyset$ and $F \cap A \neq \emptyset$ must be true.
    By the third of these conditions, we obtain that at least one of $E \cap B \neq \emptyset$ and $F \cap B \neq \emptyset$ must be true.
    Then we have the following possibilities:
    \begin{itemize}
        \item If $E \cap A \neq \emptyset$ and $E \cap B \neq \emptyset$ both hold, because we also have $E \subset A \cup B$, we obtain the contradiction that $E$ is disconnected.
        \item If $E \cap A \neq \emptyset$ and $E \cap B = \emptyset$, we have necessarily that $F \cap B \neq \emptyset$.
        Furthermore, because $E \subset A \cup B$, it must be the case that $E \subset A$.
        Now, if it holds that $F \cap A \neq \emptyset$, by the same reasoning as above we arrive at the contradiction that $F$ is disconnected.
        Therefore it must be the case that $F \cap A = \emptyset$, which means $F \subset B$.
        Notice then that $E \subset A, F \subset B, A \cap B = \emptyset$, but $E \cap F \neq \emptyset$ by the hypothesis.
        This is clearly a contradiction as well.
        \item The case $E \cap A = \emptyset$ and $E \cap B \neq \emptyset$ is symmetric to the above.
    \end{itemize}
    Therefore, $E \cup F$ must be connected.
\end{solution}

\begin{exercise}{6}
    More generally, if $\mathcal{C}$ is a collection of connected subsets of $M$, all having a point in common, prove that $\bigcup \mathcal{C}$ is connected.
    Use this to give another proof that $\mathbb{R}$ is connected.
\end{exercise}

\begin{solution}
    
    Suppose for the sake of contradiction that $\bigcup \mathcal{C}$ is disconnected, which means that there exist open sets $A, B, A \cap B = \emptyset$ such that $\bigcup \mathcal{C} \subset A \cup B, \bigcup \mathcal{C} \cap A \neq \emptyset, \bigcup \mathcal{C} \cap B \neq \emptyset$.
    Now, this means that there must exist $C_i, C_j \in \mathcal{C}$, such that $C_i \cap A \neq \emptyset$ and $C_j \cap B \neq \emptyset$.
    Furthermore, it must hold that $C_i \cap C_j \neq \emptyset$, since all sets forming $\bigcup \mathcal{C}$ have a common point.
    These now imply that $(C_i \cup C_j) \cap A \neq \emptyset$ and $(C_i \cup C_j) \cap B \neq \emptyset$, while it is also clear that $(C_i \cup C_j) \subset (A \cup B)$, thus yielding that $C_i \cup C_j$ is disconnected.
    However, in Exercise 5 we showed that since $C_i, C_j$ are connected and $C_i \cap C_j \neq \emptyset$, their union is also connected, which means that we arrive at a contradiction.
    Therefore $\bigcup \mathcal{C}$ is also connected.
    As an application, consider the fact that $\mathbb{R} = \bigcup_{a > 0} [-a, a]$, and that each of these intervals is connected, which means that since all of the intervals have 0 as a common point, $\mathbb{R}$ is also connected.

\end{solution}

\begin{exercise}{7}
    If every pair of points in $M$ is contained in some connected set, show that $M$ is itself connected.
\end{exercise}

\begin{solution}
    
    Fix a point $x \in M$, and observe that $M = \cup_{y \in M} (\{x\} \cup \{y\})$.
    We know that each pair $x, y$ is contained in some connected set $A_y$.
    Furthermore, because $M$ is the entire metric space, it must also hold that $M = \cup_{y \in M} A_y$ (i.e., it cannot be that some point in some $A_y$ is not contained in the metric space).
    We have thus written $M$ as a union of connected sets that have a point in common, namely, $x$.
    By exercise 6, $M$ is therefore also connected.
\end{solution}