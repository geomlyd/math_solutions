\chapter{Connectedness}

\section{Connected Sets}

\begin{exercise}{1}
    Supply the missing details in the proof of Lemma 6.3:

    Let $E$ be a subset of a metric space $M$.
    If $U$ and $V$ are disjoint open sets in $E$, then there are disjoint open sets $A$ and $B$ in $M$ such that $U = A \cap E, V = B \cap E$.
\end{exercise}

\begin{solution}
    
    The proof provided in the book arrives in full detail at the observation that for any two $x \in U, y \in V$ it is the case that $E \cap B_{\epsilon_x}(x) \cap B_{\epsilon_y}(y) = \emptyset$.
    Notice that this shows that if these two open balls intersect, then their intersection must be entirely contained in $M \setminus E$.
    If it was the case that they didn't intersect at all, then simply writing $A$ as the union of all $B_{\epsilon_x}(x)$ and $B$ as the union of all $B_{\delta_y}(y)$ would make these open as unions of open sets, and of course disjoint.
    The claim that is given in the proof is that one can in fact achieve this by substituting $\epsilon_x$ with $\epsilon_x/2$ and $\delta_y$ with $\delta_y/2$, i.e., it holds that $B_{\epsilon_x/2}(x) \cap B_{\delta_y/2}(y) = \emptyset$ for any two $x \in U, y \in V$.
    Suppose then that there exists $z \in M \setminus E$ such that $z \in B_{\epsilon_x/2}(x), z \in B_{\delta_y/2}$.
    We then have:

    \[d(x, y) \leq d(x, z) + d(z, y) < \frac{\epsilon_x}{2} + \frac{\delta_y}{2}\]

    WLOG, assume $\delta_y \leq \epsilon_x$, in which case we conclude $d(x, y) < 2\frac{\epsilon_x}{2} = \epsilon_x$.
    But then this means $y \in B_{\epsilon_x}(x)$, and of course $y \in E$.
    We originally had that $E \cap B_{\epsilon_x}(x) = U$, and so $y \in U$, which is a contradiction since $y \in V$ and $U \cap V = \emptyset$.
    Therefore $B_{\epsilon_x/2}(x) \cap B_{\delta_y/2}(y) = \emptyset$, and so indeed the open sets $A = \cup \{B_{\epsilon_x/2}(x), x \in U\}, B = \bigcup \{B_{\delta_y/2}(y), y \in V\}$ are disjoint.
\end{solution}

\begin{exercise}{2}
    Show that the only nonempty connected subsets of $\Delta$ are singletons. (We would say that $\Delta$ is \textit{totally disconnected}.)
\end{exercise}

\begin{solution}
    
    Suppose that there exists a connected subset of $\Delta$ that contains at least two points $x, y, x < y$.
    By Theorem 6.4, it must also contain the entire interval $[x, y]$, and so it must hold that $[x, y] \subset \Delta$.
    But this directly contradicts Exercise 22 of 2.2: $\Delta$ contains no nonempty open intervals, but here it must contain $(x, y)$, a contradiction.
    Therefore, the only nonempty connected subsets of $\Delta$ are singletons.
\end{solution}

\begin{exercise}{5}
    If $E$ and $F$ are connected subsets of $M$ with $E \cap F \neq \emptyset$, show that $E \cup F$ is connected.
\end{exercise}

\begin{solution}
    
    Suppose for the sake of contradiction that $E \cup F$ is disconnected.
    Then there exist open sets $A, B, A \cap B = \emptyset$ such that $E \cup F \subset A \cup B, (E \cup F) \cap A \neq \emptyset, (E \cup F) \cap B \neq \emptyset$.
    By the first of these conditions, we obtain $E \subset A \cup B$ and $F \subset A \cup B$.
    By the second of these conditions, we obtain that at least one of $E \cap A \neq \emptyset$ and $F \cap A \neq \emptyset$ must be true.
    By the third of these conditions, we obtain that at least one of $E \cap B \neq \emptyset$ and $F \cap B \neq \emptyset$ must be true.
    Then we have the following possibilities:
    \begin{itemize}
        \item If $E \cap A \neq \emptyset$ and $E \cap B \neq \emptyset$ both hold, because we also have $E \subset A \cup B$, we obtain the contradiction that $E$ is disconnected.
        \item If $E \cap A \neq \emptyset$ and $E \cap B = \emptyset$, we have necessarily that $F \cap B \neq \emptyset$.
        Furthermore, because $E \subset A \cup B$, it must be the case that $E \subset A$.
        Now, if it holds that $F \cap A \neq \emptyset$, by the same reasoning as above we arrive at the contradiction that $F$ is disconnected.
        Therefore it must be the case that $F \cap A = \emptyset$, which means $F \subset B$.
        Notice then that $E \subset A, F \subset B, A \cap B = \emptyset$, but $E \cap F \neq \emptyset$ by the hypothesis.
        This is clearly a contradiction as well.
        \item The case $E \cap A = \emptyset$ and $E \cap B \neq \emptyset$ is symmetric to the above.
    \end{itemize}
    Therefore, $E \cup F$ must be connected.
\end{solution}

\begin{exercise}{6}
    More generally, if $\mathcal{C}$ is a collection of connected subsets of $M$, all having a point in common, prove that $\bigcup \mathcal{C}$ is connected.
    Use this to give another proof that $\mathbb{R}$ is connected.
\end{exercise}

\begin{solution}
    
    Suppose for the sake of contradiction that $\bigcup \mathcal{C}$ is disconnected, which means that there exist open sets $A, B, A \cap B = \emptyset$ such that $\bigcup \mathcal{C} \subset A \cup B, \bigcup \mathcal{C} \cap A \neq \emptyset, \bigcup \mathcal{C} \cap B \neq \emptyset$.
    Now, this means that there must exist $C_i, C_j \in \mathcal{C}$, such that $C_i \cap A \neq \emptyset$ and $C_j \cap B \neq \emptyset$.
    Furthermore, it must hold that $C_i \cap C_j \neq \emptyset$, since all sets forming $\bigcup \mathcal{C}$ have a common point.
    These now imply that $(C_i \cup C_j) \cap A \neq \emptyset$ and $(C_i \cup C_j) \cap B \neq \emptyset$, while it is also clear that $(C_i \cup C_j) \subset (A \cup B)$, thus yielding that $C_i \cup C_j$ is disconnected.
    However, in Exercise 5 we showed that since $C_i, C_j$ are connected and $C_i \cap C_j \neq \emptyset$, their union is also connected, which means that we arrive at a contradiction.
    Therefore $\bigcup \mathcal{C}$ is also connected.
    As an application, consider the fact that $\mathbb{R} = \bigcup_{a > 0} [-a, a]$, and that each of these intervals is connected, which means that since all of the intervals have 0 as a common point, $\mathbb{R}$ is also connected.

\end{solution}

\begin{exercise}{7}
    If every pair of points in $M$ is contained in some connected set, show that $M$ is itself connected.
\end{exercise}

\begin{solution}
    
    Fix a point $x \in M$, and observe that $M = \cup_{y \in M} (\{x\} \cup \{y\})$.
    We know that each pair $x, y$ is contained in some connected set $A_y$.
    Furthermore, because $M$ is the entire metric space, it must also hold that $M = \cup_{y \in M} A_y$ (i.e., it cannot be that some point in some $A_y$ is not contained in the metric space).
    We have thus written $M$ as a union of connected sets that have a point in common, namely, $x$.
    By exercise 6, $M$ is therefore also connected.
\end{solution}

\begin{exercise}{9}
    If $A \subset B \subset \overline{A} \subset M$, and if $A$ is connected, show that $B$ is connected.
    In particular, $\overline{A}$ is connected.
\end{exercise}

\begin{solution}
    
    Suppose $B$ is disconnected.
    Observe first that obviously it cannot be that $B = A$.
    Because $B \subset \overline{A}$, it must then be that $B$ contains at least one $x \in \overline{A} \setminus A$, and possibly more.
    For any such $x$, there exists some sequence $(x_n) \subset A$ such that $x_n \rightarrow x$ (by the definition of the closure).
    Because $B$ is disconnected, by Lemma 6.5 there exists $f: B \rightarrow \{0, 1\}$ continuous and onto.
    Note that by the corollary noted immediately after Lemma 6.5, the restriction of $f$ to $A$ must be constant because $A$ is connected.
    This means that, WLOG, $f(a) = 0$ for each $a \in A$.
    More specifically, $f(x_n) = 0$ for all $n$.
    However, because $f$ is onto, it must be the case that $f(x) = 1$ for at least one of the $x$ described above.
    But this is a contradiction due to the fact that continuous functions preserve limits on convergent sequences.
    Therefore, $B$ must be connected.
\end{solution}

\begin{exercise}{13}
    If $f: [a, b] \rightarrow [a, b]$ is continuous, show that $f$ has a fixed point; that is, show that there is some point $x \in [a, b]$ with $f(x) = x$.
\end{exercise}

\begin{solution}

    If $f(b) = b, b$ serves as the fixed point.
    If $f(a) = a, a$ serves as the fixed point.
    If neither of these hold, then due to the range of $f$ we have that $f(b) < b \implies f(b) - b < 0, f(a) > a \implies f(a) - a > 0$.
    Let then $g: [a, b] \rightarrow \mathbb{R}, g(x) = f(x) - x$, which is a continuous function such that $g(a) > 0, g(b) < 0$.
    Therefore, by the Intermediate Value Theorem (Corollary 6.7), we obtain that $g$ assumes every value between $g(b)$ and $g(a)$, so more specifically it achieves the value of 0 for some $c \in (a, b)$, meaning that $g(c) = 0 \implies f(c) = c$.
\end{solution}

\begin{exercise}{15}
    If $f: \mathbb{R} \rightarrow \mathbb{R}$ is continuous and open, show that $f$ is strictly monotone.
\end{exercise}

\begin{solution}
    
    If $f$ is constant in any interval $(a, b)$, we trivially see that it cannot be open (the image of $(a, b)$ under $f$ is a single point, which is a closed set).
    From now on we therefore assume that it is not constant in any interval.
    Suppose that it is not strictly monotone, which means that there exist $a, b, c \in \mathbb{R}, a < b < c$ such that (assuming WLOG a direction for the inequalities) $f(a) < f(b)$ but $f(b) > f(c)$.
    The important observation here is that $f$, as a continuous function, achieves a minimum value $m$ and a maximum value $M$ in the interval $[a, c]$, and that $M$ cannot be achieved at either $a$ or $c$ (since $f(b) > f(a), f(b) > f(c)$).
    Therefore, when we consider the open interval $(a, b)$, it must be the case that $f((a, b))$ is an interval that is either of the form $[m, M]$, or of the form $(m, M]$.
    In both cases, this is not an open set, which contradicts the assumption that $f$ is open.
    Therefore, $f$ is strictly monotone.
\end{solution}

\begin{exercise}{26}
    Let $f: [0, 1] \rightarrow \mathbb{R}$ be defined as $f(x) = \sin(1/x)$ for $x \neq 0$ and $f(0) = 0$.
    Show that although $f$ is not continuous, the graph of $f$ is a connected subset of $\mathbb{R}^2$.
    [Hint: Use Exercise 9.]
\end{exercise}

\begin{solution}
    
    First we show that $f$ is not continuous at 0.
    Indeed, pick $\epsilon = 1/2$ and observe that it should be the case that for some $\delta > 0$, whenever $\lvert x \rvert < \delta, \lvert f(x) \rvert < \epsilon$.
    However, for any $\delta$, if we select a sufficiently large $k \in \mathbb{N}$, we obtain that $\frac{2}{4k \pi + \pi} < \delta$ but $\sin(1/x) = \sin(\frac{4k \pi + \pi}{2}) = \sin(2k\pi + \pi/2) = 1 > \epsilon$, and so $f$ cannot be continuous at 0 (this is one way to formalize the increasingly fast oscillations of this function's graph).
    Now we show that its graph is connected in $\mathbb{R}^2$.
    We have that the set $\{x > 0\}$ is connected (as an interval).
    Furthermore, the map $x \mapsto (x, f(x)), x > 0$ is continuous (Lemma 5.8), and so the set $\{(x, f(x)), x > 0\}$ is connected as the image of a connected set under a continuous function.
    Now, the closure of this set of course contains $\{(0, 0)\} \cup \{(x, f(x)), x > 0\}$ (i.e., the graph of $f$): the sequence $((\frac{1}{n\pi}, f(\frac{1}{n\pi}))), n > 0$ converges to $(0, 0)$.
    By exercise 9, we thus have that the graph of $f$ is connected.
\end{solution}

\begin{exercise}{27}
    Let $V$ be a normed vector space, and let $x \neq y \in V$.
    Show that the map $f(t) = x + t(y - x)$ is a homeomorphism from $[0, 1]$ into $V$.
    The range of $f$ is the line segment joining $x$ and $y$, and is often written $[x, y]$ (since $f$ is a homeomorphism, the interval notation is justified).
    [Hint: That $f$ is continuous and one-to-one is easy; next, show that if $f(t_n) \rightarrow z$, then $(t_n)$ converges to some $t$ in $[0, 1]$ with $z = f(t)$.]
\end{exercise}

\begin{solution}
    
    To show that $f$ is one-to-one, we have that if $f(t_1) = f(t_2)$:

    \[f(t_1) = f(t_2) \implies x + t_1(y - x) = x + t_2(y - x) \implies (t_1 - t_2)(y - x) = 0,\]

    by the properties of vector spaces.
    Because $y \neq x, y - x$ cannot equal the zero vector, and so it must be the case that $t_1 = t_2$, thus showing that $f$ is one-to-one.
    The \textit{into} part of the statement implies that we do not need to show that $f$ is onto, since we only care about its range.
    For continuity, notice that:

    \[\lvert \lvert f(t_1) - f(t_2) \rvert \rvert = \lvert \lvert x + t_1(y - x) - x - t_2(y - x) \rvert \rvert = \lvert \lvert (t_1 - t_2)(y - x) \rvert \rvert = \lvert t_1 - t_2 \rvert \cdot \lvert \lvert y - x \rvert \rvert,\]

    and so $f$ is in fact Lipschitz with constant $\lvert \lvert y - x \rvert \rvert$, and therefore continuous.
    The only thing that remains is showing that $f^{-1}$ (defined in $f$'s range) is also continuous.
    To do this, we follow the hint: suppose $f(t_n) \rightarrow z = f(t)$.
    Then, for any given $\epsilon > 0$, set $\delta = \epsilon \lvert \lvert y - x \rvert \rvert$ and we have that for $n \geq N$ for some $N$:

    \[\lvert \lvert f(t_n) - f(t) \rvert \rvert < \delta \implies \lvert t_n - t \rvert \cdot \lvert \lvert y - x \rvert \rvert < \epsilon \lvert \lvert y - x \rvert \rvert \implies \lvert t_n - t \rvert < \epsilon,\]

    which of course shows that $t_n \rightarrow t$, which completes the proof that $f^{-1}$ is continuous, and thus $f$ is a homeomorphism into $V$.
\end{solution}

\begin{exercise}{28}
    Deduce from Exercises 7 and 27 that any normed space $V$ is connected.
\end{exercise}

\begin{solution}
    
    If we can show that any pair of vectors $x, y \in V$ is contained in some connected set, we will have shown that $V$ is connected (Exercise 7).
    The observation here is that for any two such vectors, consider the set $\{x + t(y - x), t \in [0, 1]\} \subset V$.
    This is the image $f([0, 1])$ if $f$ is defined as in Exercise 27.
    Because $[0, 1]$ is connected as an interval and because $f$ is continuous, we conclude that $f([0, 1])$ is also connected.
    This set contains $x, y$, and is connected, so by our initial observation we conclude that the normed vector space $V$ is connected.
\end{solution}