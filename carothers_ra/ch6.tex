\chapter{Connectedness}

\section{Connected Sets}

\begin{exercise}{1}
    Supply the missing details in the proof of Lemma 6.3:

    Let $E$ be a subset of a metric space $M$.
    If $U$ and $V$ are disjoint open sets in $E$, then there are disjoint open sets $A$ and $B$ in $M$ such that $U = A \cap E, V = B \cap E$.
\end{exercise}

\begin{solution}
    
    The proof provided in the book arrives in full detail at the observation that for any two $x \in U, y \in V$ it is the case that $E \cap B_{\epsilon_x}(x) \cap B_{\epsilon_y}(y) = \emptyset$.
    Notice that this shows that if these two open balls intersect, then their intersection must be entirely contained in $M \setminus E$.
    If it was the case that they didn't intersect at all, then simply writing $A$ as the union of all $B_{\epsilon_x}(x)$ and $B$ as the union of all $B_{\delta_y}(y)$ would make these open as unions of open sets, and of course disjoint.
    The claim that is given in the proof is that one can in fact achieve this by substituting $\epsilon_x$ with $\epsilon_x/2$ and $\delta_y$ with $\delta_y/2$, i.e., it holds that $B_{\epsilon_x/2}(x) \cap B_{\delta_y/2}(y) = \emptyset$ for any two $x \in U, y \in V$.
    Suppose then that there exists $z \in M \setminus E$ such that $z \in B_{\epsilon_x/2}(x), z \in B_{\delta_y/2}$.
    We then have:

    \[d(x, y) \leq d(x, z) + d(z, y) < \frac{\epsilon_x}{2} + \frac{\delta_y}{2}\]

    WLOG, assume $\delta_y \leq \epsilon_x$, in which case we conclude $d(x, y) < 2\frac{\epsilon_x}{2} = \epsilon_x$.
    But then this means $y \in B_{\epsilon_x}(x)$, and of course $y \in E$.
    We originally had that $E \cap B_{\epsilon_x}(x) = U$, and so $y \in U$, which is a contradiction since $y \in V$ and $U \cap V = \emptyset$.
    Therefore $B_{\epsilon_x/2}(x) \cap B_{\delta_y/2}(y) = \emptyset$, and so indeed the open sets $A = \cup \{B_{\epsilon_x/2}(x), x \in U\}, B = \bigcup \{B_{\delta_y/2}(y), y \in V\}$ are disjoint.
\end{solution}