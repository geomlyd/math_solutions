\chapter{Metrics and Norms}

\section{Metric Spaces}


\begin{exercise}{2}
    If $d$ is a metric on $M$, show that $\lvert d(x, z) - d(y, z) \rvert \leq d(x, y) $ for any $x, y, z \in M$.
\end{exercise}

\begin{solution}
    
    First, we apply the triangle inequality as follows:
    $$d(x, z) \leq d(x, y) + d(y ,z) \implies d(x, z) - d(y, z) \leq d(x, y)$$
    Another application of it yields:
    $$d(y, z) \leq d(y, x) + d(x, z) \implies -d(y, x) \leq d(x, z) - d(y, z) \implies -d(x, y) \leq d(x, z) - d(y,z),$$
    where we used the symmetry property of metric $d$.
    Putting the two inequalities together results in:
    $$\lvert d(x, z) - d(y, z) \rvert \leq d(x, y)$$
\end{solution}

\begin{exercise}{3}
    As it happens, some of our requirements for a metric are redundant.
    To see why this is so, let $M$ be a set and suppose $d: M \times M \rightarrow \mathbb{R}$ satisfies $d(x, y) = 0$ if and only if $x = y$, and $d(x, y) \leq d(x, z) + d(y, z)$ for all $x, y, z \in M$.
    Prove that $d$ is a metric; that is, show that $d(x, y) \geq 0$ and $d(x, y) = d(y, x)$ hold for all $x, y$.
\end{exercise}

\begin{solution}
    
    Pick any two $x, y \in M$.
    We know then that for any $z \in M$ it holds that $d(x, z) \leq d(x, y) + d(z, y)$.
    More specifically, this holds for $z = x$, in which case we obtain:
    $$d(x, x) \leq d(x, y) + d(x, y) \implies 0 \leq 2d(x, y),$$
    which means that for any two $x, y \in M, d(x, y) \geq 0$.

    For the symmetry propety, once again pick any two $x, y \in M$.
    Then $d(x, y) \leq d(x, z) + d(y, z)$ for any $z$.
    Pick then $z = x$, in which case we obtain $d(x, y) \leq d(x, x) + d(y, x) \implies d(x, y) \leq d(y, x)$.
    By exchanging the roles of $x, y$, we have that $d(y, x) \leq d(y, z) + d(x, z)$ for any $z$.
    Pick $z = y$ to observe that $d(y, x) \leq d(y, y) + d(x, y) \implies d(y, x) \leq d(x, y)$.
    But then $d(x, y) \leq d(y, x) \leq d(x, y)$, thus the only possibility is that $d(x, y) = d(y, x)$.
    Therefore $d$ is indeed a metric.
\end{solution}

\newpage

\begin{exercise}{6}
    If $d$ is any metric on $M$, show that $\rho(x, y) = \sqrt{d(x, y)}, \sigma(x, y) = d(x, y)/(1 + d(x, y))$ and $\tau(x, y) = \min\{d(x, y), 1\}$ are also metrics on $M$.
    [Hint: $\sigma(x, y) = F(d(x, y))$, where $F$ is as in exercise 5.]
\end{exercise}

\begin{solution}
    
    We will solve this as a simple application of 7.
    If $F_1: [0, \infty) \rightarrow [0, \infty), F_1(x) = \sqrt{x}$, then $\rho(x, y) = F_1(d(x, y))$.
    Then $F_1(x)/x = 1/\sqrt{x}$, which is clearly decreasing for $x> 0$, and therefore exercise 7 guarantees that $\rho$ is a metric.

    Similarly, if $F_2: [0, \infty) \rightarrow [0, \infty), F_2(x) = \frac{x}{1 + x}$, then $F_2'(x) = \frac{x + 1 - x}{(1 + x)^2} = \frac{1}{(x + 1)^2}$, which is clearly decreasing as for $x \geq 0$, and therefore $\sigma$ is also a metric.

    Lastly, if $F_3: [0, \infty) \rightarrow [0, \infty), F_3(x) = \min\{x, 1\}$, then $\tau(x, y) = F_3(d(x, y))$ and:
    $$F_3(x)/x = \begin{cases}
        1, & x \leq 1 \\
        \frac{1}{x}, & x > 1
    \end{cases},$$
    which is of course a decreasing function for $ x > 0$, and therefore $\tau$ is a metric.
\end{solution}
    
\begin{exercise}{7}
    Here is a generalization of exercises 5 and 6.
    Let $f: [0, \infty) \rightarrow [0, \infty)$ be increasing and satisfy $f(0) = 0$, and $f(x) > 0$ for all $x > 0$.
    If $f$ also satisfies $f(x + y) \leq f(x) + f(y)$ for all $x, y \geq 0$, then $f \circ d$ is a metric whenever $d$ is a metric.
    Show that each of the following conditions is sufficient to ensure that $f(x + y) \leq f(x) + f(y)$ for all $x, y \geq 0$:

    (a) $f$ has a second derivative satisfying $f'' \leq 0$

    (b) $f$ has a decreasing first derivative

    (c) $f(x)/x$ is decreasing for $x > 0$
    
    [Hint: First show that (a) $\implies$ (b) $\implies$ (c).]
\end{exercise}

\begin{solution}
    
    As indicated in the hint, we first show that (a) $\implies$ (b) $\implies$ (c).
    If, after that, we can show that (c) implies the triangle inequality for $f$, we know also that any of (a), (b) imply it as well (since they imply (c)).

    (a) $\implies$ (b): Since $f'' \leq 0$, $f$ more specifically has a first derivative on every point of $[0, \infty)$.
    From calculus 1, it's known also that this implies that $f'$ is decreasing (one would prove this via the Mean Value Theorem).

    (b) $\implies$ (c): (b) implies that $f$ is differentiable, thus $g(x) = f(x)/x, x > 0$ is also differentiable, with $g'(x) = \frac{f'(x)x - f(x)}{x^2}$.
    If we can show that the numerator here is non-positive for $x > 0$, we'll have shown that $g$ is decreasing.
    Pick any $x > 0$, and apply the Mean Value Theorem on $f$ in the interval $[0, x]$.
    This means that there exists $y \in (0, x)$ such that:
    $$f'(y) = \frac{f(x) - f(0)}{x - 0} = \frac{f(x)}{x},$$
    where we used the fact that $f(0) = 0$.
    Therefore, we have that:
    $$f'(x)x - f(x) = f'(x)x - xf'(y) = x(f'(x) - f'(y)) \leq 0,$$
    since $f'$ is decreasing and $y < x$.
    This completes the proof that (b) $\implies$ (c).

    We now need to show that the triangle inequality for $f$ follows from (c).
    Pick any two $x, y \geq 0$.
    If any of them, or both, are zero, the triangle inequality follows from the fact that $f(0) = 0$.
    If $x, y \neq 0$, then we know that $x \leq x + y, y \leq x + y$, and thus that $\frac{f(x)}{x} \geq \frac{f(x + y)}{x + y}, \frac{f(y)}{y} \geq \frac{f(x + y)}{x + y}$ (by the fact that $f(x)/x$ is decreasing). 
    We can rewrite these as:
    $$(x + y)f(x) \geq xf(x+ y), (x+ y)f(y) \geq yf(x + y),$$
    and now we can add them to obtain that:
    $$(x + y)(f(x) + f(y)) \geq f(x + y)(x + y) \implies f(x) + f(y) \geq f(x + y),$$
    which is of course what we wanted to show.
\end{solution}

\begin{exercise}{10}
    The \textit{Hilbert cube}, $H^{\infty}$, is the collection of all real sequences $x = (x_n)$ with $\lvert x_n \rvert \leq 1, n = 1, 2, \ldots$.

    (i) Show that $d(x, y) = \sum_{n=1}^{\infty} 2^{-n} \lvert x_n - y_n \rvert$ defines a metric on $H^{\infty}$.

    (ii) Given $x, y \in H^{\infty}$ and $k \in \mathbb{N}$, let $M_k = \max\{ \lvert x_1 - y_1 \rvert, \ldots, \lvert x_k - y_k \rvert\}$.
    Show that $2^{-k}M_k \leq d(x, y) \leq M_k + 2^{-k + 1}$.
\end{exercise}

\begin{solution}
    
    (i) We will make use of exercise 3, which allows us to conclude that $d$ is a metric if the following three things hold:
    \begin{itemize}
        \item First, that $d(x, y)$ is a non-negative real number for any $x, y$, which, due to the infinite sum, is non-trivial here. 
        Consider any two sequences $x, y$. We know that $\lvert x_n \rvert \leq 1, \lvert y_n \rvert \leq 1$, for all $n$, therefore the maximum value that $\lvert x_n - y_n \rvert$ achieves is 2.
        This means that $2^{-n} \lvert x_n - y_n \rvert \leq 2^{-n+1}$.
        The partial sums that correspond to $d$ are therefore non-decreasing and upper bounded by a convergent geometric series, therefore the series that defines $d$ also converges.
        \item Second, that $d(x, y) = 0$ iff $x = y$.
        If two sequences $x, y$ are equal, then $x_n = y_n$ for all $n$, thus $d(x, y) = \sum_{n=1}^{\infty} 2^{-n} \lvert x_n - x_n \rvert = 0$.
        Conversely, if $d(x, y) = 0$, then we have a non-decreasing sequence of non-negative partial sums that converges to zero.
        The only way this can happen is if each term of the sequence is zero, i.e.\ if $\lvert x_n - y_n \rvert = 0$ for all $n$, that is, if $x_n = y_n$ for all $n$.
        But this means precisely that $x = y$.
        \item Third, that the triangle inequality holds.
        Pick any three sequences $x, y, z$, and $k \in \mathbb{N}^+$.
        Then:
        $$\sum_{n = 1}^{k} 2^{-n} \lvert x_n - y_n \rvert = \sum_{n=1}^{k} 2^{-n} \lvert x_n - z_n + z_n - y_n \rvert \leq \sum_{n=1}^{k} 2^{-n} \lvert x_n - z_n \rvert + \sum_{n=1}^{k} 2^{-n} \lvert z_n - y_n \rvert,$$
        by the triangle inequality.
        Since this holds for any $k$ and all series converge (by the first item proven), we know that the inequality holds for the infinite series as well, i.e.\ that:
        $$d(x, y) \leq d(x, z) + d(y, z)$$
    \end{itemize}
    Therefore $d$ is a metric on the Hilbert cube.

    (ii) Pick any two sequences $x, y$ in the Hilbert cube and $k \in \mathbb{N}$.
    We know then by definition that $M_k \geq \lvert x_n - y_n \rvert$ for $i \leq k$.
    In addition, for $n > k$, the $n$-th term of the series that forms the metric $d$ is $2^{-n}\lvert x_n - y_n \rvert \leq 2^{-n+1}$.
    Thus:
    $$d(x, y) = \sum_{n=1}^{k} 2^{-n} \lvert x_n - y_n \rvert + \sum_{n > k}^{\infty}2^{-n}\lvert x_n - y_n \rvert \leq M_k \sum_{n=1}^{k}2^{-n} + \sum_{n=k+1}^{\infty} 2^{-n + 1}$$
    $$ = M_k(1 - 2^{-k}) + 2(2 - 1 - (1 - 2^{-k})) = M_k - 2^{-k}M_k + 2^{-k+1} \leq M_k + 2^{-k+1},$$
    where we used the fact that the geometric series with ratio $r=1/2$ sums to 2, and our second sum here was thus 2 minus the sum of the first $k$ terms minus 1 since the series starts at 1 instead of 0.
    In the last step we also use the fact that $M_k \geq 0$.
    For the other inequality, we have again that:
    $$d(x, y) = \sum_{n=1}^{k} 2^{-n} \lvert x_n - y_n \rvert + \sum_{n > k}^{\infty}2^{-n}\lvert x_n - y_n \rvert$$
    Notice that in order for $x, y$ to minimize this, it should be the case that $x_n - y_n = 0, n > k$, and that the maximum absolute difference $M_k$ is multiplied with the smallest possible quantity up to $k$, that is, with $2^{-k}$, while for all other $n < k, x_n - y_n = 0$.
    This would yield that $d(x, y) \geq M_k2^{-k} + 0 + 0 = M_k2^{-k}$.

\end{solution}

\begin{exercise}{12}
    Check that $d(f, g) = \max_{a \leq t \leq b} \lvert f(t) - g(t) \rvert$ defines a metric on $C[a, b]$, the collection of all continuous, real valued functions defined on the closed interval $[a, b]$.
\end{exercise}

\begin{solution}
    
    Note that here continuity is important to ensure that the difference of any two functions is continuous, and as such achieves a maximum value in any closed interval.
    Using exercise 3, we need to check the following:
    \begin{itemize}
        \item One, that $d(f, g) = 0$ iff $f = g$.
        Suppose first that $d(f, g) = 0$, that is, $\max_{a \leq b} \lvert f(t) - g(t) \rvert = 0$.
        If $f, g$ were not equal, there would exist $x \in [a, b]$ such that $f(x) \neq g(x)$.
        Clearly then $\max_{a \leq t \leq b} \lvert f(t) - g(t) \rvert \geq \lvert f(x) - g(x) \rvert > 0$, which is a contradiction.
        Therefore $f = g$.
        In the other direction, if $f = g$, we equivalently have that $f - g$ is the zero function on $[a, b]$, and therefore $d(f, g) = \max_{a \leq t \leq b} \lvert f(t) - g(t) \rvert = 0$.
        \item Two, that for any three $f, g, h \in C[a, b], d(f, h) \leq d(f, g) + d(g, h)$.
        We have first that $d(f, h) = \max_{a \leq t \leq b} \lvert f(t) - h(t) \rvert$.
        Suppose that this function achieves its maximum value on $t_M \in [a, b]$ (if there are more than one, pick any).
        Then $d(f, h) = \lvert f(t_M) - h_(t_M) \rvert = \lvert f(t_M) - g(t_M) + g(t_M) - h(t_M) \rvert \leq \lvert f(t_M) - g(t_M) \rvert + \lvert g(t_M) - h(t_M) \rvert$.
        By definition, it holds that $\lvert f(t_M) - g(t_M) \rvert \leq \max_{a \leq t \leq b} \lvert f(t) - g(t) \rvert = d(f, g)$, and similarly $\lvert g(t_M) - h(t_M) \rvert \leq d(g, h)$.
        We therefore obtain that $d(f, h) \leq d(f, g) + d(g, h)$.
    \end{itemize}
    Therefore $d$ is indeed a metric on $C[a, b]$.
\end{solution}

\begin{exercise}{14}
    We say that a subset $A$ of a metric space is \textbf{bounded} if there is some $x_0 \in M$ and some constant $C < \infty$ such that $d(a, x_0) \leq C$ for all $a \in A$.
    Show that a finite union of bounded sets is again bounded.
\end{exercise}

\begin{solution}
    
    Suppose we have the finite union 
    $$U = A_1 \cup A_2 \cup \ldots \cup A_n$$
    of $n$ bounded subsets of a metric space $M$.
    Let then $x_1, x_2, \ldots, x_n, C_1, \ldots, C_n$ be such that $d(a, x_i) \leq C$ for $a \in A_i$.
    Form the set $\{d(x_1, x_1), d(x_1, x_2), \ldots, d(x_n, x_1)\}$, which, crucially, has a finite number of elements, and therefore also has a maximum element, $M$.
    By the same reasoning, there exists $C$ such that $C = \max\{C_1, \ldots, C_n\}$.
    Now pick any $a \in U$, which must belong in at least one $A_i$.
    Therefore:
    $$d(a, x_1) \leq d(a, x_i) + d(x_i, x_1) \leq C_i + M \leq C + M$$
    If we thus set $x_U = x_1$ and $C_U = C + M$, we have shown precisely that $U$ is bounded.
\end{solution}

\begin{exercise}{15}
    We define the \textbf{diameter} of a nonempty subset $A$ of $M$ by $\text{diam}(A) = \sup\{d(a, b) : a, b \in A\}$.
    Show that $A$ is bounded if and only if $\text{diam}(A)$ is finite.
\end{exercise}

\begin{solution}
    
    $\implies$: Suppose first that $A$ is bounded.
    Then there exists $x_0 \in M, C \in \mathbb{R}$ such that $d(a, x_0) \leq C$ for all $a \in A$.
    Pick any two $a, b \in A$.
    We then have that $d(a, b) \leq d(a, x_0) + d(x_0, b) \leq 2C$.
    Therefore the set $\{d(a, b): a, b \in A$\} has an upper bound, namely, $2C$, and therefore also a finite least upper bound, which means precisely that $\text{diam}(A)$ is finite.

    $\impliedby$: Now suppose $\text{diam}(A) = C \in \mathbb{R}$.
    Fix an $a \in A$ (which exists since $A$ is nonempty).
    Pick any $b \in A$, in which case we have that:
    $$d(a, b) \in \{d(x, y): x, y \in A\} \implies d(a, b) \leq \text{diam}(A) = C$$
    If thus set $x_0 = a$ and $C$ the respective constant, we see that the definition of $A$ being bounded is fulfilled.
\end{solution}

\section{Normed Vector Spaces}

\begin{exercise}{16}
    Let $V$ be a vector space, and let $d$ be a metric on $V$ satisfying $d(x, y) = d(x - y, 0)$ and $d(ax, ay) = \lvert a \rvert d(x, y)$ for every $x, y \in V$ and every scalar $a$.
    Show that $\lvert \lvert x \rvert \rvert = d(x, 0)$ defines a norm on $V$ (that has $d$ as its ``usual'' metric).
    Give an example of a metric on the vector space $\mathbb{R}$ that fails to be associated with a norm in this way.
\end{exercise}

\begin{solution}
    Let us examine the properties that would make $\lvert \lvert \cdot \rvert \rvert$ one by one:
    \begin{itemize}
        \item Suppose $x \in V$.
        Then $\lvert \lvert x \rvert \rvert = d(x, 0) \geq 0$, since $d$ is a metric (and of course $\lvert \lvert x \rvert \rvert$ is well-defined as a finite real number for the same reason).
        \item Suppose $\lvert \lvert x \rvert \rvert = 0$.
        Then $d(x, 0) = 0$, which by the properties of metrics we know is true iff $x = 0$.
        Conversely, if $x = 0$, by the same argument $d(x, 0) = 0 \implies \lvert \lvert x \rvert \rvert = 0$.
        Thus $\lvert \lvert x \rvert \rvert = 0 \iff x = 0$.
        \item For any scalar $a$, we have that $\lvert \lvert a x \rvert \rvert = d(ax, 0) = d(ax, a0) = \lvert a \rvert d(x, 0) = \lvert a \rvert \lvert \lvert x \rvert \rvert$ by the exercise hypothesis.
        \item Pick any two $x, y \in V$.
        Then: 
        $$\lvert \lvert x +y \rvert \rvert = d(x + y, 0) \leq d(x + y, y) + d(y, 0) = d(x + y - y, 0) + d(y, 0) = d(x, 0) + d(y, 0) = \lvert \lvert x \rvert \rvert + \lvert \lvert y \rvert \rvert,$$
        where we used the triangle inequality property for metrics and the fact that $d(x, y) = d(x - y, 0)$.
        Therefore the triangle inequality holds for the proposed norm.
    \end{itemize}
    We have thus shown that $\lvert \lvert \cdot \rvert \rvert$ is indeed a norm, and its usual metric will of course be $d'(x, y) = \lvert \lvert x - y \rvert \rvert = d(x - y, 0) = d(x, y)$, i.e.,\ $d$.

    For the requested example, consider the metric $\sigma(x, y) = \lvert x - y \rvert/(1 + \lvert x - y \rvert)$ from exercise 6 of section 3.1.
    Notice that the ``proposed'' norm would then be $\lvert \lvert x \rvert \rvert = \sigma(x, 0) = \lvert x \rvert/(1 + \lvert x \rvert)$.
    Then for a scalar $a$:
    $$\lvert \lvert a x \rvert \rvert = \lvert a x \rvert/(1 + \lvert a x \rvert) = \lvert a \rvert \cdot \lvert x \rvert/(1 + \lvert a \rvert \cdot \lvert x \rvert),$$
    which we can see does not necessarily equal $\lvert a \rvert \cdot \lvert \lvert x \rvert \rvert$, and therefore the ``proposed'' norm is not really a norm.
    The cause for this is the fact that $\sigma(ax, ay) \neq \lvert a \rvert \sigma(x, y)$ in general, for similar reasons.
\end{solution}

\begin{exercise}{18}
    Show that $\lvert \lvert x \rvert \rvert_{\infty} \leq \lvert \lvert x \rvert \rvert_2 \leq \lvert \lvert x \rvert \rvert_1$ for any $x \in \mathbb{R}^n$.
    Also check that $\lvert \lvert x \rvert \rvert_1 \leq n \lvert \lvert x \rvert \rvert_{\infty}$ and $\lvert \lvert x \rvert \rvert_1 \leq \sqrt{n} \lvert \lvert x \rvert \rvert_2$.
\end{exercise}

\begin{solution}
    
    We have that $\lvert \lvert x \rvert \rvert_{\infty} = \max_{1 \leq i \leq n} \lvert x_i \rvert$, while $\lvert \lvert x \rvert \rvert_2 = \sqrt{\sum_{i=1}^{n} \lvert x_i \rvert^2}$.
    By definition, there must exist at least one $j \in \{1, 2, \ldots, n\}$ such that $\lvert x_j \rvert = \max_{1 \leq i \leq n} \lvert x_i \rvert$.
    In addition, the square is an increasing function for $x \geq 0$, which would mean that $(\max_{1 \leq i \leq n} \lvert x_i \rvert)^2 = \max_{1 \leq i \leq n} \lvert x_i \rvert^2$.
    We therefore have that:
    $$\sum_{i=1}^{n} \lvert \lvert x_i \rvert \rvert^2 = (\max_{1 \leq i \leq n} \lvert x_i \rvert)^2 + X,$$
    for $X \geq 0$ (the sum of the absolute values of the remaining coordinates).
    Thus $(\lvert \lvert x \rvert \rvert_2)^2 \geq (\lvert \lvert x \rvert \rvert_{\infty})^2$, and by taking square roots we have the first desired result.
    
    We also have that:
    $$(\lvert \lvert x \rvert \rvert_1)^2 = \Biggl(\sum_{i=1}^{n} \lvert x_i \rvert\Biggr)\Biggl(\sum_{i=1}^{n} \lvert x_i \rvert\Biggr) \geq \sum_{i=1}^{n} \lvert x_i \rvert^2 = (\lvert \lvert x_2 \rvert \rvert)^2,$$
    which we obtain by observing that the product results in a sum of non-negative terms over all pairs of indices $i, j \in \{1, 2, \ldots, n\}$, which of course includes all terms for which $i = j$.
    Taking square roots yields the second desired result.

    For the third result, observe that for each $i, \lvert x_i \rvert \leq \max_{1 \leq i \leq n} \lvert x_i \rvert = \lvert \lvert x \rvert \rvert_{\infty}$.
    This means of course that:
    $$\lvert \lvert x \rvert \rvert_1 = \sum_{i=1}^{n} \lvert x_i \rvert \leq n \lvert \lvert x \rvert \rvert_{\infty}$$
    
    Lastly, we consider the following ``trick'' for the last part.
    For a given $x = (x_1, x_2, \ldots, x_n)$, consider the vectors $y, z \in \mathbb{R}^{n^2}$:
    $$y = \begin{pmatrix}
        \lvert x_1 \rvert \\
        \lvert x_2 \rvert \\
        \vdots \\
        \lvert x_n \rvert \\
        \vdots \\
        \lvert x_1 \rvert \\
        \lvert x_2 \rvert \\
        \vdots \\
        \lvert x_n \rvert
    \end{pmatrix}, z = \begin{pmatrix}
        \lvert x_1 \rvert  \\
        \lvert x_1 \rvert \\
        \vdots \\
        \lvert x_1 \rvert \\
        \vdots \\
        \lvert x_n \rvert \\
        \lvert x_n \rvert\\
        \vdots \\
        \lvert x_n \rvert
    \end{pmatrix}$$
    We have that $\lvert \lvert z \rvert \rvert_2 = \lvert \lvert y \rvert \rvert_2 = \sqrt{\sum_{i=1}^{n} n\lvert x_i \rvert^2 } = \sqrt{n} \lvert \lvert x \rvert \rvert_2$,
    while $\langle y, z \rangle = \sum_{i=1}^{n} \sum_{j=1}^{n} \lvert x_i \rvert \cdot \lvert x_j \rvert = (\sum_{i=1}^{n} \lvert x_i \rvert)^2 = (\lvert \lvert x \rvert \rvert_1)^2$.
    Applying the Cauchy-Schwarz inequality on $y, z$, we then have that:
    $$\langle y, z \rangle \leq \lvert \lvert y \rvert \rvert \cdot \lvert \lvert z \rvert \rvert \implies (\lvert \lvert x \rvert \rvert_1)^2 \leq \sqrt{n} \lvert \lvert x \rvert \rvert_2 \sqrt{n} \lvert \lvert x \rvert \rvert_2,$$
    which, once more by taking square roots, gives us the last desired result.
\end{solution}

\begin{exercise}{19}
    Show that we have $\sum_{i=1}^{n} x_i y_i = \lvert \lvert x \rvert \rvert_2 \lvert \lvert y \rvert \rvert_2$ (equality in the Cauchy-Schwarz inequality) if and only if $x, y$ are \textit{proportional}, that is, if and only if $x = ay$ or $y = ax$ for some $a \geq 0$.
\end{exercise}

\begin{solution}

    $\impliedby$: Suppose $x = ay$ for some $a \geq 0$ (everything is symmetric for $y = ax$).
    Then:
    $$\lvert \lvert x \rvert \rvert_2 \lvert \lvert y \rvert \rvert_2 = \lvert \lvert a y \rvert \rvert_2 \lvert \lvert y \rvert \rvert_2 = \lvert a \rvert \cdot \lvert \lvert y \rvert \rvert_2^2,$$
    where we used the ``scalar multiplication''/positive homogeneity property of the norm.
    Furthermore:
    $$\sum_{i = 1}^{n} x_i y_i = \sum_{i=1}^{n} (a y_i) y_i = a \sum_{i=1}^{n} y_i^2 = a \lvert \lvert y \rvert \rvert_2^2,$$ 
    by the definition of the 2-norm on sequences, which yields the desired equality.

    $\implies$: First, if $y = 0$ then this is trivially true. If $y \neq 0$, it's also the case that $\lvert \lvert y \rvert \rvert_2 \neq 0$ and we have the following.
    Consider the beginning of the proof of the Cauchy-Schwarz inequality:
    $$0 \leq \lvert \lvert x + ty \rvert \rvert_2^2 = \lvert \lvert x \rvert \rvert_2^2 + 2t \langle x, y \rangle + t^2 \lvert \lvert y \rvert \rvert_2^2,$$
    for any $t \in \mathbb{R}$. 
    As we saw, the discriminant here is $\Delta = (2\langle x, y \rangle)^2 - 4 \lvert \lvert x \rvert \rvert_2^2 \lvert \lvert y \rvert \rvert_2^2 = 0$, by our hypothesis that $\sum_{i=1}^{n} x_i y_i = \lvert \lvert x \rvert \rvert_2 \lvert \lvert y \rvert \rvert_2$ (since $\langle x, y \rangle = \sum_{i=1}^{n} x_i y_i$).
    Then this means that the corresponding second-degree polynomial of $t$ has a \textit{unique} solution $a$:
    $$a = \frac{-2 \langle x, y \rangle}{2\lvert \lvert y \rvert \rvert_2^2} = -\frac{ \langle x, y \rangle}{\lvert \lvert y \rvert \rvert_2^2}$$
    By the above equation, it must also be the case that:
    $$0 = \lvert \lvert x + ay \rvert \rvert_2^2,$$
    which, by the properties of the norm, means that by necessity $x = -ay$.
    Now we only need to show that $a \leq 0$.
    Observe that the denominator of $a$ is clearly positive as the norm of $y$.
    The numerator equals, by definition, $\sum_{i=}^{n} x_i y_i = \lvert \lvert x \rvert \rvert_2 \lvert \lvert y \rvert \rvert_2$ (by the hypothesis), so it's clearly also non-negative.
    The minus sign in front of the fraction makes $a$ non-positive, which completes the proof.

\end{solution}

\begin{exercise}{21}
    Recall that we defined $l_1$ to be the collection of all absolutely summable sequences under the norm $\lvert \lvert x \rvert \rvert_1 = \sum_{n=1}^{\infty} \lvert x_n \rvert$, and we defined $l_{\infty}$ to be the collection of all bounded sequences under the norm $\lvert \lvert x \rvert \rvert_{\infty} = \sup_{n \geq 1} \lvert x_n \rvert$.
    Fill in the details showing that each of these spaces is in fact a normed vector space.
\end{exercise}

\begin{solution}
    
    For each of the two proposed norms, we need to show that they are in fact norms, and also that the sequences which converge under them form a vector space.
    We begin with showing that $l_1$ defines a norm:
    \begin{itemize}
        \item It's obvious from the definition that $\lvert \lvert x \rvert \rvert_1 \geq 0$ whenever it exists, as a sum of non-negative terms.
        \item For any scalar $a$, we have that if for a sequence $x$ the series $\sum_{n=1}^{\infty} \lvert x_n \rvert$ converges to $\lvert \lvert x \rvert \rvert _1$, then by well-known properties of limits, the series $\sum_{n=1}^{\infty} \lvert a x_n \rvert$ will converge to $\lvert a \rvert \cdot \lvert \lvert x \rvert \rvert _1$.
        \item If $\sum_{i=0}^{\infty} \lvert x_n \rvert$ converges to zero, then since all terms are strictly non-negative, it must be the case that all of them are zero, thus that $\lvert \lvert x \rvert \rvert _1 = 0$ implies $x = 0$.
        The converse is obvious.
        \item We now need to prove the triangle inequality.
        Pick $x, y \in l_1$ and any $n > 0$.
        Then, by the triangle inequality for absolute values:

        $$ \sum_{i=1}^{n} \lvert x_i + y_i \rvert \leq \sum_{i=1}^{n} \lvert x_i \rvert + \sum_{i=1}^{n} \lvert y_i \rvert$$

        Notice that as $n \rightarrow \infty$, the two sums on the RHS converge to $\lvert \lvert x \rvert \rvert _1, \lvert \lvert y \rvert \rvert _1$ respectively.
        This imposes a bound on the series corresponding to the LHS, whose partial sums are also non-decreasing.
        Hence, the LHS also converges, and in fact the inequality holds at infinity, yielding $\lvert \lvert x + y \rvert \rvert _1 \leq \lvert \lvert x \rvert \rvert _1 + \lvert \lvert y \rvert \rvert _1$.
    \end{itemize}
    Therefore, $\lvert \lvert x \rvert \rvert_1$ is a norm.
    The third point above showed that $0 \in l_1$, while the triangle inequality proof showed that $l_1$ is closed under addition.
    Lastly, the second point above showed that $l_1$ is closed under scalar multiplication.
    Therefore, we've already shown that $l_1$ is a vector space, and thus $(l_1, \lvert \lvert \cdot \rvert \rvert_1)$ is a normed vector space.

    Now we examine $\lvert \lvert \cdot \rvert \rvert _{\infty}$ in a similar fashion.
    \begin{itemize}
        \item Again, from the definition it is obvious that whenever $x$ is a bounded sequence, $\lvert \lvert x \rvert \rvert _{\infty} \geq 0$.
        \item For any scalar $a$, and any bounded sequence $x$, the sequence $(\lvert a x_i \rvert)$ will have as supremum $\lvert a \rvert \cdot \lvert \lvert x \rvert \rvert _{\infty}$. 
        If this were not the case, one would get a contradiction for $x$ by dividing with $\lvert a \rvert$ (and if $a = 0$, then the supremum of $(\lvert a x_i \rvert)$ is obviously zero).
        \item It's clear that $\lvert \lvert x \rvert \rvert _{\infty}$ is zero iff $x$ is the zero sequence, since the supremum of a set of non-negative numbers is zero iff all of them are zero.
        \item Lastly, the triangle inequality in this case arises as follows.
        Pick any $x, y \in l_{\infty}$.
        Then, for any $i$ we have that $\lvert x_i + y_i \rvert \leq \lvert x_i \rvert + \lvert y_i \rvert \leq \lvert \lvert x \rvert \rvert _{\infty} + \lvert \lvert y \rvert \rvert _{\infty}$ by the definition of the infinity norm.
        But then this sum constitutes an upper bound for $\lvert x_i + y_i \rvert$ for all $i$, which means that by definition the supremum $\sup_{i} \lvert x_i + y_i \rvert$ both exists and is such that $\lvert \lvert x + y \rvert \rvert _{\infty} \leq \lvert \lvert x \rvert \rvert _{\infty} + \lvert \lvert y \rvert \rvert _{\infty}$.
    \end{itemize}
    The above shows both that $\lvert \lvert \cdot \rvert \rvert _{\infty}$ is a norm, and that $l_{\infty}$ is a vector space, thus that $(l_{\infty}, \lvert \lvert \cdot \rvert \rvert _{\infty})$ is a normed vector space.
\end{solution}

\section{More Inequalities}

\begin{exercise}{24}
    The conclusion of Lemma 3.7 (Hölder's inequality) also holds in the case $p = 1$ and $q = \infty$.
    Why?
\end{exercise}

\begin{solution}
    
    As a reminder, we say that a sequence $y$ is in $l_{\infty}$ if $y$ is bounded above, and in that case we define $\lvert \lvert y \rvert \rvert_{\infty} = \sup_{n} \lvert y_n \rvert$. 
    Let us first formally state what Hölder's inequality would assert in this case:

    Given $x \in l_1$ and $y \in l_{\infty}$, we have $\sum_{i=1}^{\infty} \lvert x_i y_i \rvert \leq \lvert \lvert x \rvert \rvert_1 \lvert \lvert y \rvert \rvert_{\infty}$.

    Inded, pick any $n \geq 1$.
    We then have that:
    $$\sum_{i=1}^{n} \lvert x_i y_i \rvert = \sum_{i=1}^{n} \lvert x_i \rvert \cdot \lvert y_i \rvert \leq \sum_{i=1}^{n} \lvert x_i \rvert \cdot \sup_{k} \lvert y_k \rvert = \sup_{k} \lvert y_k \rvert \cdot \sum_{i=1}^{n} \lvert x_i \rvert \leq \lvert \lvert y \rvert \rvert_{\infty} \cdot \lvert \lvert x \rvert \rvert_1,$$
    where we used the definition of the supremum and the fact that $\lvert \lvert x \rvert \rvert_1$ is well-defined.
    Since the partial sums are non-decreasing and bounded, the LHS converges and the inequality (i.e., Hölder's) holds for the infinite series as well.
\end{solution}

\begin{exercise}{25}
    The same techniques can be used to show that $\lvert \lvert f \rvert \rvert_p = (\int_{0}^{1} \lvert f(t) \rvert^p dt)^{1/p}$ defines a norm on $C[0, 1]$ for any $1 < p < \infty$.
    State and prove the analogues of Lemma 3.7 and Theorem 3.8 in this case.
    (Does Lemma 3.7 still hold in this setting for $p = 1, q = \infty$?)
\end{exercise}

\begin{solution}
    
    We begin by noting that due to the absolute value, the functions being integrated are always non-negative.
    For any $t \in (0, 1)$, we can thus apply Young's inequality on $a = \frac{\lvert f(t) \rvert}{\lvert \lvert f \rvert \rvert_p}, b = \frac{\lvert g(t) \rvert}{\lvert \lvert g \rvert \rvert_q}$:

    $$\frac{\lvert f(t) \rvert}{\lvert \lvert f \rvert \rvert_p}\cdot \frac{\lvert g(t) \rvert}{\lvert \lvert g \rvert \rvert_q} \leq \frac{1}{p}\cdot \frac{\lvert f(t) \rvert^p}{\lvert \lvert f \rvert \rvert_p^p} + \frac{1}{q} \cdot \frac{\lvert g(t) \rvert^q}{\lvert \lvert g \rvert \rvert_q^{q}}$$

    Since this is true for all $t \in (0, 1)$, we can integrate both sides wrt. $t$ to obtain:
    
    $$\int_{0}^{1}\frac{\lvert f(t) \rvert}{\lvert \lvert f \rvert \rvert_p}\cdot \frac{\lvert g(t) \rvert}{\lvert \lvert g \rvert \rvert_q} dt \leq \frac{1}{p}\cdot \int_{0}^{1} \frac{\lvert f(t) \rvert^p}{\lvert \lvert f \rvert \rvert_p^p} dt + \frac{1}{q} \cdot \int_{0}^{1} \frac{\lvert g(t) \rvert^q}{\lvert \lvert g \rvert \rvert_q^{q}} dt \implies$$
    $$\frac{1}{\lvert \lvert f \rvert \rvert_p \cdot \lvert \lvert g \rvert \rvert_q} \int_{0}^{1} \lvert f(t) g(t) \rvert dt \leq \frac{1}{p}\cdot \frac{1}{\lvert \lvert f \rvert \rvert_p^p}\int_{0}^{1} \lvert f(t) \rvert^p dt + \frac{1}{q} \cdot \frac{1}{\lvert \lvert g \rvert \rvert_q^{q}} \int_{0}^{1} \lvert g(t) \rvert^q dt = \frac{1}{p} + \frac{1}{q} = 1$$

    This yields $\int_{0}^{1} \lvert f(t) g(t) \rvert dt \leq \lvert \lvert f \rvert \rvert_p \cdot \lvert \lvert g \rvert \rvert_q$, which would be the equivalent of Hölder's inequality.

    Now, Minkowski's inequality would state firstly that for any $1 < p < \infty$, if the $p$-norm exists for $f, g \in C[0, 1]$, it also exists for $f + g$, and secondly that $\lvert \lvert f + g \rvert \rvert_p \leq \lvert \lvert f \rvert \rvert_p + \lvert \lvert g \rvert \rvert_p$.

    For the analogue of Minkowski's inequality (``if $f, g \in C[0, 1]$ and $\lvert \lvert f \rvert \rvert_p, \lvert \lvert g \rvert \rvert_p$ both exist, then $\lvert \lvert f + g \rvert \rvert_p$ exists and $\lvert \lvert f + g \rvert \rvert_p \leq \lvert \lvert f \rvert \rvert_p + \lvert \lvert g \rvert \rvert_p$''), we first prove an analogue of Lemma 3.5 that allows us to immediately obtain that $\lvert \lvert f + g \rvert \rvert_p$ exists.
    More specifically, for any $t \in (0, 1)$, from Lemma 3.5 applied on $a = \lvert f(t) \rvert, b = \lvert g(t) \rvert$, we have that:

    $$(\lvert f(t) \rvert + \lvert g(t) \rvert)^p \leq 2^p(\lvert f(t) \rvert^p + \lvert g(t) \rvert^p)$$

    By the triangle inequality of the absolute value, we also have that $\lvert f(t) + g(t) \rvert^p \leq (\lvert f(t) \rvert+ \lvert g(t) \rvert)^p$ since $p > 1$.
    Therefore, since these hold for any $t \in (0, 1)$:

    $$\lvert f(t) + g(t) \rvert^p \leq 2^p(\lvert f(t) \rvert^p + \lvert g(t) \rvert^p)\implies \int_{0}^{1} \lvert f(t) + g(t) \rvert^p dt \leq 2^p(\int_{0}^{1} \lvert f(t) \rvert^p dt + \int_{0}^{1} \lvert g(t) \rvert^p dt)$$
    $$\implies \lvert \lvert f + g \rvert \rvert_p^p \leq 2^p(\lvert \lvert f \rvert \rvert_p^p + \lvert \lvert g \rvert \rvert_p^p),$$
    
    which imposes a bound on $\lvert \lvert f + g \rvert \rvert_p$, thus showing it exists.

    Now, following the proof in the book, we observe the following regarding $\lvert \lvert f \rvert \rvert_p^{p-1}$, for $1/p + 1/q = 1$:

    $$\lvert \lvert f^{p-1} \rvert \rvert_q = (\int_{0}^{1} \lvert f^{q(p-1)}(t) \rvert dt)^{1/q} = (\int_{0}^{1} \lvert f(t) \rvert^{p} dt)^{1/q} = (\int_{0}^{1} \lvert f(t) \rvert^{p} dt)^{\frac{p-1}{p}} = \lvert \lvert f \rvert \rvert_p^{p - 1}$$

    Now, for any $t \in (0, 1)$:

    $$\lvert f(t) + g(t) \rvert^p = \lvert f(t) + g(t) \rvert \cdot \lvert f(t) + g(t) \rvert^{p-1} \leq \lvert f(t) \rvert \cdot \lvert f(t) + g(t) \rvert^{p-1} + \lvert g(t) \rvert \cdot \lvert f(t) + g(t) \rvert^{p-1},$$

    which means we can integrate to obtain that:

    $$\int_{0}^{1} \lvert f(t) + g(t) \rvert^p dt \leq \int_{0}^{1} \lvert f(t) \rvert \cdot \lvert f(t) + g(t) \rvert^{p-1}dt + \int_{0}^{1} \lvert g(t) \rvert \cdot \lvert f(t) + g(t) \rvert^{p-1}dt$$

    Now, define $q$ such that $1/p + 1/q = 1$, and by Hölder's inequality and the observation above:

    $$\int_{0}^{1} \lvert f(t) + g(t) \rvert^p dt \leq \lvert \lvert f \rvert \rvert_p \cdot \lvert \lvert (f + g)^{p-1} \rvert \rvert_q + \lvert \lvert g \rvert \rvert_p \cdot \lvert \lvert (f + g)^{p-1} \rvert \rvert_q \leq \lvert \lvert f \rvert \rvert_p \cdot \lvert \lvert f + g \rvert \rvert_{p}^{p-1} + \lvert \lvert g \rvert \rvert_p \cdot \lvert \lvert f + g \rvert \rvert_p^{p-1}$$
    $$\implies \lvert \lvert f + g \rvert \rvert^p \leq \lvert \lvert f + g \rvert \rvert_p^{p-1}(\lvert \lvert f \rvert \rvert_p + \lvert \lvert g \rvert \rvert_p),$$

    from which Minkowski's inequality follows directly.
    For the case $p = 1, q = \infty$, we have that for any $t \in (0, 1)$:

    $$\lvert f(t) g(t) \rvert = \lvert f(t) \rvert \cdot \lvert g(t) \rvert \leq \lvert f(t) \rvert \cdot \max_{0 \leq t \leq 1} \lvert g(t) \rvert = \lvert f(t) \rvert \cdot \lvert \lvert g \rvert \rvert_{\infty}$$
    $$\implies \int_{0}^{1} \lvert f(t) g(t) \rvert dt \leq \lvert \lvert g \rvert \rvert_{\infty} \int_{0}^{1} \lvert f(t) \rvert dt = \lvert \lvert g \rvert \rvert_{\infty} \cdot \lvert \lvert f \rvert \rvert_{1},$$

    showing that Hölder's inequality does indeed hold again.
\end{solution}

\begin{exercise}{26}
    Given $a, b > 0$, show that $\lim_{p \rightarrow \infty}(a^p + b^p)^{1/p} = \max\{a, b\}$. [Hint: If $a < b$ and $r = a/b$ show that $(1/p)\log(1 + r^p) \rightarrow 0$ as $p \rightarrow \infty$.] What happens as $p \rightarrow 0$? as $p \rightarrow -1$? as $p \rightarrow -\infty$?
\end{exercise}

\begin{solution}
    
    First, if $a = b$, the statement is obvious: the quantity inside the limit simplifies to $(2a^p)^{1/p} = 2^{1/p}a$, and as $p \rightarrow \infty, 1/p \rightarrow 0$ thus $2^{1/p}a \rightarrow 1a = a = \max\{a, b\}$.
    Therefore we can assume from now on that $a < b$, and, as indicated in the hint, set $r = a / b < 1$.
    Since $r < 1$, we have that $\lim{_p \rightarrow \infty} r^p \rightarrow 0$ (an exponential with a base less than 1).
    Therefore, by standard limit rules, $\lim_{p \rightarrow \infty} \log(1 + r^p) = \log(1) = 0$. 
    Furthermore $\lim_{p \rightarrow \infty} \frac{1}{p} = 0$, which means $\lim_{p \rightarrow \infty} (1/p)\log(1 + r^p) = 0$.
    Now we apply the following ``trick'':

    $$(a^p + b^p)^{1/p} = e^{\log(a^p + b^p)^{1/p}} = e^{\frac{\log(a^p + b^p)}{p}} = e^{\frac{\log((rb)^p + b^p)}{p}} = e^{\frac{\log(b^p) + \log(1 + r^p)}{p}} = e^{\log(b) + \frac{\log(1 + r^p)}{p}}$$
    
    Clearly, what we showed based on the hint allows us to take limits on both sides and easily obtain that:
    $$\lim_{p \rightarrow \infty}(a^p + b^p)^{1/p} = e^{\log(b)} = b = \max\{a, b\}$$

    Now, for the case of $p \rightarrow 0$, if $a = b, \lim_{p \rightarrow 0}(a^p + b^p)^{1/p} = \lim_{p \rightarrow 0}(2a^p)^{1/p} = \lim_{p \rightarrow 0}(2^{1/p}a)$, and as $p \rightarrow 0+$ this quantity will tend to positive infinity, whereas if $p \rightarrow 0-$ it will tend to zero, thus the limit does not exist.
    If $a < b$, based on the above observation what interests us is $\lim_{p \rightarrow 0} \frac{\log(1 + r^p)}{p}$.
    In this case, $r^p \rightarrow 1$ as $p \rightarrow 0$, and thus the numerator tends to $\log(2)$, thus the entire fraction tends to positive infinity as $p \rightarrow 0+$ and negative infinity as $p \rightarrow 0-$, which means the limit does not exist.

    The expression inside the limit is continuous as a function of $p$ at -1, since $a, b > 0$.
    Therefore $\lim_{p \rightarrow -1}(a^p + b^p)^{1/p} = (\frac{1}{a} + \frac{1}{b})^{-1} = \frac{1}{\frac{1}{a} + \frac{1}{b}} = \frac{ab}{a + b}$.

    Lastly, for $p \rightarrow -\infty$, for $a = b$ we are interested in $\lim_{p \rightarrow -\infty}(2^{1/p}a)$, which is easily seen to equal $a$.
    For $a < b$, we are interested in $\lim_{p \rightarrow -\infty}\frac{\log(1 + r^p)}{p}$.
    Because $0 < r < 1$, the quantity inside the logarithm will tend to positive infinity, whereas the denominator tends to negative infinity.
    Applying L'Hospital's rule and calling this limit $L$ we have that:

    $$L = \lim_{p \rightarrow -\infty} \frac{\log(r)r^p}{1 + r^p} = \lim_{p \rightarrow -\infty} \frac{\log(r)}{1 + \frac{1}{r^p}}$$

    Here the numerator is constant, whereas since $0 < r < 1, r^p \rightarrow \infty$, and thus the denominator will tend to 1.
    This means that $L = \log(r) = \log(a/b) = \log(a) - \log(b)$.
    Going back to our original limit we would have that $\lim_{p \rightarrow -\infty}(a^p + b^p)^{1/p} = e^{\log(b) + \log(a) - \log(b)} = a = \min\{a, b\}$, which we can see was also true for $a = b$.
\end{solution}

\begin{exercise}{ - Unlisted; Arose from a discussion of exercise 26}
    a) Prove that if $1 \leq p \leq q \leq \infty$, then $l^p \subset l^q$.

    b) If $x \in l_p \subset l_q$ for $1 \leq p \leq q \leq \infty$, then $\lvert \lvert x \rvert \rvert_p \geq \lvert \lvert x \rvert \rvert_q$.

    c) If $x \in l^{p_0}$ for some $p_0$, prove that $\lim_{p \rightarrow \infty} \lvert \lvert x \rvert \rvert_p = \lvert \lvert x \rvert \rvert _\infty$.
\end{exercise}

\begin{solution}
    
    a) Let $x$ be a sequence in $l^p$.
    If $p = q$, then the statement is obvious, so we continue the proof assuming that $p \neq q$.
    In the case where $q = \infty$, we know that if, for some $1 \leq p < \infty, \lvert \lvert x \rvert \rvert _p$ exists, then the sequence of partial sums corresponding to $(\lvert x_i \rvert^p)$ must be bounded and non-decreasing.
    But then the same must hold true for the sequence $(\lvert x_i \rvert)$, which leads us to conclude that $x$ is in fact bounded, and thus has a supremum.
    This means then that $x \in l_q$, since $q = \infty$ and the infinity norm is defined as the supremum of absolute values.

    Now, in the remaining case we have that $1 \leq p < q$ and both are real numbers.
    Note that, by using e.g. the Archimidean property of $\mathbb{R}$, we can write $q = np + \epsilon$, where $n$ is a positive integer and $\epsilon > 0$.
    Consider now examining the first $m$ terms of $x$ in the following way:

    $$\sum_{i=1}^{m} \lvert x_i \rvert^q = \sum_{i=1}^{m} \lvert x_i \rvert^{np + \epsilon} = \sum_{i=1}^{m} \lvert x_i \rvert^{np} \cdot \lvert x_i \rvert^{\epsilon}$$

    Recall that we already showed that if $x \in l_p$, then $\lvert x_i \rvert$ are bounded above, and thus we can say that for every $i, \lvert x_i \rvert^\epsilon \leq S^\epsilon, S =\sup_{j} \lvert x_j \rvert$.
    Furthermore, since $n$ is a positive integer, we have that:
    
    $$\sum_{i=1}^{m} \lvert x_i \rvert^{np} \leq (\sum_{i=1}^{m} \lvert x_i \rvert^p)^n,$$
    since if one expands the power of the RHS, we get at least all terms of the LHS plus possibly more, all of which are non-negative.
    Combining these two facts, we have that:

    $$\sum_{i=1}^{m} \lvert x_i \rvert^q \leq (\sum_{i=1}^{m} \lvert x_i \rvert^p)^n S^\epsilon \leq \lvert \lvert x \rvert \rvert_p^{pn} S^{\epsilon},$$

    which we can safely conclude since the $p$-norm exists.
    But then the partial sums are bounded above for each $m$ and are non-decrasing, meaning that the LHS converges as $m \rightarrow \infty$, and this equals precisely $\lvert \lvert x \rvert \rvert_q^q$.
    Thus $x$ is indeed also in $l^q$.

    b) Consider a sequence $x = (x_1, x_2, \ldots) \in l_p \subset l_q$. 
    The statement is obvious if $p = q$, and also if $x$ is the zero sequence.
    Therefore from now on we assume $p < q, x \neq 0$, which means also $\lvert \lvert x \rvert \rvert _p > 0, \lvert \lvert x \rvert \rvert _q > 0$.
    In the case where $q = \infty$, we have that $\lvert \lvert x \rvert \rvert_q = \sup_{i} \lvert x_i \rvert$.
    For any $m > 0$, we have that:

    $$\sum_{i=1}^{m} \lvert x_i \rvert^p \geq \max_{1 \leq i \leq m} \lvert x \rvert^p$$

    If we take limits on both sides, and by thinking about the $\epsilon$-based definition of the supremum, we can see that this yields $\lvert \lvert x \rvert \rvert _p^p \geq \lvert \lvert x \rvert \rvert _\infty^p \implies \lvert \lvert x \rvert \rvert_p \geq \lvert \lvert x \rvert \rvert_\infty$.


    In the case where $p < q < \infty$, consider the following.
    Let $y$ be the sequence formed by $y_i = \frac{x_i}{\lvert \lvert x \rvert \rvert _p}$.
    Notice first that $\lvert \lvert y \rvert \rvert _p = 1$.
    Notice also that for any $i, \lvert y_i \rvert^p \leq 1$, since $\lvert \lvert y \rvert \rvert _p = 1 \implies \sum_{i=1}^{\infty} \lvert y_i \rvert^p = 1^p$, so any individual term of the series must be at most 1.
    Since $q > p \geq 1$, we have that it must be the case that $q = rp$ for some $r > 1$.
    Then, by using properties of powers, for any $i$:
    
    $$\lvert y_i \rvert^p \leq 1 \implies \lvert y_i \rvert^{rp} \leq \lvert y_i \rvert^p \implies \lvert y_i \rvert^q \leq \lvert y_i \rvert^p $$

    This in turn implies that for any $m > 0$:

    $$\sum_{i=1}^{m} \lvert y_i \rvert^q \leq \sum_{i=1}^{m} \lvert y_i \rvert^p \leq \lvert \lvert y \rvert \rvert _p^p = 1$$

    Since this holds for any $m$, it also holds at infinity, meaning that $\lvert \lvert y \rvert \rvert _q^q \leq 1 = \lvert \lvert y \rvert \rvert _p ^q$, and thus by taking $q$-roots we obtain that $\lvert \lvert y \rvert \rvert_q \leq \lvert \lvert y \rvert \rvert_p$.
    But then by using the definition of $y$:
    
    $$\lvert \lvert y \rvert \rvert_q \leq \lvert \lvert y \rvert \rvert _p \implies \Biggl\lvert \Biggl \lvert \frac{x}{\lvert \lvert x \rvert \rvert _p} \Biggr \rvert \Biggr \rvert _q \leq \Biggl \lvert \Biggl \lvert \frac{x}{\lvert \lvert x \rvert \rvert _p} \Biggr \rvert \Biggr \rvert _p \implies \lvert \lvert x \rvert \rvert _q \leq \lvert \lvert x \rvert \rvert _p,$$

    where we used the ``multiplication by scalar'' property of norms.
    This completes the proof.

    c) First of all, we observe from part (a) that since $\lvert \lvert x \rvert \rvert_{p_0}$ exists for some $p_0$, it will also be the case that $x \in l^p$ for all $p \geq p_0$, as well as that $x \in l^{\infty}$.
    We now have the following:

    $$\lim_{p \rightarrow \infty} \frac{\lvert \lvert x \rvert \rvert _p}{\lvert \lvert x \rvert \rvert _\infty} = \lim_{p \rightarrow \infty} e^{\log \Bigl (\frac{\lvert \lvert x \rvert \rvert _p}{\lvert \lvert x \rvert \rvert _{\infty}} \Bigr )}$$

    We focus on the exponent:

    $$\lim_{p \rightarrow \infty} \log \Biggl( \frac{\lvert \lvert x \rvert \rvert _p}{\lvert \lvert x \rvert \rvert _{\infty}}\Biggr) = \lim_{p \rightarrow \infty} \log \Biggl( \frac{(\sum_{i=1}^{\infty} \lvert x_i \rvert^p)^{1/p}}{(\lvert \lvert x \rvert \rvert _{\infty}^p)^{1/p}}\Biggr) = \lim_{p \rightarrow \infty} \frac{1}{p} \cdot \log \Biggl( \sum_{i=1}^{\infty} \frac{\lvert x_i \rvert^p}{\lvert \lvert x \rvert \rvert _{\infty}^{p}} \Biggr)$$

    Now we make the observation that since the series corresponding to $\lvert \lvert x \rvert \rvert_p^p$ converges, the individual terms tend to zero.
    This means that must exist at least one $j$ such that $\lvert x_j \rvert$ equals the supremum of $x$.
    This is because the only other possibility would be for $\lvert x_i \rvert$ to tend to their supremum without achieving it, but then the series would not converge.
    Therefore, the argument of the logarithm is at least 1 (since $\frac{\lvert x_j \rvert^p}{\lvert \lvert x \rvert \rvert _{\infty}^p} = 1$).
    Additionally, the existence of $\lvert \lvert x \rvert \rvert _p$ implies that the argument does not go to infinity.
    Thus, the numerator of the last limit above tends to $L > 1$ and the denominator to positive infinity, which means that the limit tends to zero.
    But then:

    $$\lim_{p \rightarrow \infty} \frac{\lvert \lvert x \rvert \rvert _p}{\lvert \lvert x \rvert \rvert _{\infty}} = e^0 = 1,$$

    which concludes the proof since $\lvert \lvert x \rvert \rvert _{\infty}$ is constant with respect to the limit variable.
\end{solution}

\begin{exercise}{ - Unlisted; Arose from a discussion of exercise 26}
    Recall that for a continuous function $f \in C([0, 1])$ we define
    
    $$ \lvert \lvert f \rvert \rvert _p = (\int_{0}^{1} \lvert f(x) \rvert^p dx)^{1/p}, \lvert \lvert f \rvert \rvert _{\infty} = \max_{x \in [0, 1]} \lvert f(x) \rvert $$
    Show that:

    a) If $f \in C[0, 1]$, then for $1 \leq p \leq q \leq \infty, \lvert \lvert f \rvert \rvert_{1} \leq \lvert \lvert f \rvert \rvert_{p} \leq \lvert \lvert f \rvert \rvert_{q} \leq \lvert \lvert f \rvert \rvert_{\infty}.$

    b) If $f \in C[0, 1]$, then $\lim_{p \rightarrow \infty} \lvert \lvert f \rvert \rvert_{p} = \lvert \lvert f \rvert \rvert_{\infty}$.
\end{exercise}

\begin{solution}
    
    a) We begin by showing that for any real number $p \geq 1, \lvert \lvert f \rvert \rvert _{p} \leq \lvert \lvert f \rvert \rvert _{\infty}$.
    We do this as follows. First, we have that:
    
    $$\lvert \lvert f \rvert \rvert_p^{p} = \int_{0}^{1} \lvert f(x) \rvert^p dx$$

    Since $p \geq 1$, the $p$-th power is a non-decreasing function, and we know that by definition, $\lvert f^p(x) \rvert \leq \max_{0 \leq x \leq 1} \lvert f^p(x) \rvert$ for any $x \in [0, 1]$. 
    By integrating both sides we have that:

    $$\int_{0}^{1} \lvert f^p(x)  \rvert dx \leq \max_{0 \leq x \leq 1} \lvert f^p(x) \rvert \implies \lvert \lvert f \rvert \rvert _{p}^{p} \leq \lvert \lvert f \rvert \rvert _{\infty}^{p}$$

    By taking $p$-roots, we have the desired inequality.
    Now consider any two real numbers $p, q \geq 1$ such that $p < q$.
    This means that there exists $r > 1$ such that $q = rp$.
    Let $g(x) = f^P(x)$ and set $r'$ to be the number satisfying $1/r + 1/r' = 1$.
    We now apply Hölder's inequality for the functions $g$ and $h(x) = 1$:

    $$\int_{0}^{1} \lvert g(x) \cdot 1 \rvert dx \leq \lvert \lvert g \rvert \rvert_{r} \cdot \lvert \lvert 1 \rvert \rvert_{r'} \implies \int_{0}^{1} \lvert f^p(x) \rvert dx \leq (\int_{0}^{1} \lvert f^{rp}(x) \rvert)^{(1/r)} \implies (\int_{0}^{1} \lvert f^p(x) \rvert dx)^{r} \leq \int_{0}^{1} \lvert f^{rp}(x) \rvert dx,$$

    where for the last step we used the fact that $r > 1$.
    Continuing:

    $$(\int_{0}^{1} \lvert f^p(x) \rvert dx)^{r} \leq \int_{0}^{1} \lvert f^{rp}(x) \rvert dx \implies (\int_{0}^{1} \lvert f^p(x) \rvert dx)^{q/p} \leq \int_{0}^{1} \lvert f^{q}(x) \rvert dx$$
    $$ \implies (\int_{0}^{1} \lvert f^p(x) \rvert dx)^{1/p} \leq (\int_{0}^{1} \lvert f^{q}(x) \rvert dx)^{1/q} \implies \lvert \lvert f \rvert \rvert _{p} \leq \lvert \lvert f \rvert \rvert _{q}$$
    
    This completes the proof of the remaining inequalities, thus establishing that for $1 \leq p \leq q \leq \infty, \lvert \lvert f \rvert \rvert _{1} \leq \lvert \lvert f \rvert \rvert _{p} \leq \lvert \lvert f \rvert \rvert _{q} \leq \lvert \lvert f \rvert \rvert _{\infty}$.

    b) From part (a) we already know that as $p \rightarrow \infty, \lvert \lvert f \rvert \rvert _p$ always exists, and in fact is bounded above by $\lvert \lvert f \rvert \rvert _{\infty}$ and is non-decreasing (since for $p \leq q, \lvert \lvert f \rvert \rvert _p \leq \lvert \lvert f \rvert \rvert _q$).
    This means that $\lim_{p \rightarrow \infty} \lvert \lvert f \rvert \rvert _p$ exists, and will equal the supremum of the set $S = \{ \lvert \lvert f \rvert \rvert _{p}, p \geq 1\}$.
    We thus only need to show that this supremum is in fact $\lvert \lvert f \rvert \rvert _{\infty}$.
    Since we already know that this constitutes an upper bound for $S$, assume that the supremum of $S$ is $M < \lvert \lvert f \rvert \rvert_{\infty}$.
    Then, by the definition of the infinity norm, there exists $x_0 \in [0, 1]$ such that $\lvert f(x_0) \rvert = \lvert \lvert f \rvert \rvert _{\infty} > M$.
    For the sake of simplicity, we shall assume that $x_0 \in (0, 1)$: as will become clear, in the ``edge cases'' the only thing that changes is that some quantities lack a factor of 2.
    Set now $\epsilon = \lvert f(x_0) \rvert - M > 0$.
    Since $f$ is continuous, there must exist $\delta > 0$ such that:
    
    $$\lvert x - x_0 \rvert < \delta \implies \lvert f(x_0) \rvert - \lvert f(x) \rvert < \epsilon,$$

    where the absolute value in the second inequality can be omitted since $\lvert f(x_0) \rvert$ is the maximum value of $\lvert f \rvert$.
    Now, this can be rewritten as $\lvert f(x) \rvert > \lvert f(x_0) \rvert - \epsilon = \lvert f(x_0) \rvert - \lvert f(x_0) \rvert + M = M$, which means that in the interval $(x_0 - \delta, x_0 + \delta), \lvert f(x) \rvert > M$.
    Therefore, for all $x$ in this interval we have that $\lvert f(x) \rvert > M \implies \lvert f^p(x) \rvert > M^p$. 
    By integrating:

    $$\int_{x_0 - \delta}^{x_0 + \delta} \lvert f^p(x) \rvert dx > 2M^p \delta \implies \Biggl(\int_{x_0 - \delta}^{x_0 + \delta} \lvert f^p(x) \rvert dx\Biggr)^{1/p} > (2\delta)^{1/p}M$$

    We now have that $\lvert \lvert f \rvert \rvert _p$ is greater than or equal to the LHS here, since $(x_0 - \delta, x_0 + \delta) \subset (0, 1)$.
    Furthermore, as $p \rightarrow \infty$, the RHS tends to M, since $\delta$ is constant.
    By taking limits, we can thus obtain that:

    $$\lim_{p \rightarrow \infty} \lvert \lvert f \rvert \rvert _p \geq M$$

    Now we observe that we can repeat the entirety of the argument above for some $N$ with $M < N < \lvert \lvert f \rvert \rvert _{\infty}$.
    But then this means also that $\lim_{p \rightarrow \infty} \lvert \lvert f \rvert \rvert _p \geq N$, and then the $\epsilon$-based limit definition would allow us to find $p$ such that $\lvert \lvert f \rvert \rvert _p > M$, which contradicts the defining property of $M$ as the supremum of all $\lvert \lvert f \rvert \rvert _p$.

    Therefore, we have arrived at a contradiction, and thus it must be the case that $\lim_{p \rightarrow \infty} \lvert \lvert f \rvert \rvert _p = \lvert \lvert f \rvert \rvert _{\infty}$.

\end{solution}

\section{Limits in Metric Spaces}

\begin{exercise}{30}
    If $A \subset B$, show that $\text{diam}(A) \leq \text{diam}(B)$.
\end{exercise}

\begin{solution}
    
    Consider the definition of $\text{diam}(A): \text{diam}(A) = \{\sup\{d(a, b): a, b \in A\}$.
    Because $A \subset B$, we have that any two $a, b \in A$ also belong in $B$.
    Therefore, the set $S_1$ over which the diameter is computed for $B$ is a superset of the set $S_2$ over which the diameter is computed for $A$.
    But then exercise 2 of Chapter 1 guarantees that $\sup S_1 \leq \sup S_2$, which means precisely that $\text{diam}(A) \leq \text{diam}(B)$.
\end{solution}

\begin{exercise}{32}
    In a normed vector space $(V, \lvert \lvert \cdot \rvert \rvert)$ show that $B_r(x) = x + B_r(0) = \{x + y: \lvert \lvert y \rvert \rvert < r\}$ and that $B_r(0) = rB_1(0) = \{r x : \lvert \lvert x \rvert \rvert <1 \}$.
\end{exercise}

\begin{solution}
    
    Let us call $S = \{x + y: \lvert \lvert y \rvert \rvert < r\}$, in which case we asked to show that $B_r(x) = S$.
    First, let $z \in B_r(x)$.
    By definition, this means that $d(x, z) < r$.
    Recall that in a normed vector space we have that $\lvert \lvert z - x \rvert \rvert = d(x, y)$ (unless a different metric is explicitly specified).
    Observe then that we can write $z = x + (z - x)$, where $\lvert \lvert z - x \rvert \rvert = d(z, x) = d(x, z) < r$. 
    By setting $y = z - x$, we obtain that $z \in S$.
    Therefore, $B_r(x) \subset S$.

    In the other direction, suppose $z \in S$, which means that there exists $y, \lvert \lvert y \rvert \rvert < r$ such that $z = x + y$.
    Then we observe that $d(x, z) = \lvert \lvert z - x \rvert \rvert = x + y - x \rvert \rvert = \lvert \lvert y \rvert \rvert < r$.
    By definition, this means that $z \in B_r(x)$, and thus that $S \subset B_r(x)$, meaning that in fact $S = B_r(x)$.

    For the second part of the exercise, set $S = \{rx : \lvert \lvert x \rvert \rvert < 1 \}$.
    First, suppose $x \in B_r(0)$, which means $ \lvert \lvert x \rvert \rvert < r$.
    By using the scalar multiplication properties of vector spaces, we can then write $x = r \cdot \frac{x}{r}$.
    Set $y = \frac{x}{r}$, in which case $\lvert \lvert y \rvert \rvert = \lvert \lvert \frac{x}{r} \rvert \rvert = \frac{1}{\lvert r \rvert} \lvert \lvert x \rvert \rvert < 1$. 
    This means that $x$ can be written in the form $ry, \lvert \lvert y \rvert \rvert < 1$, thus that $x \in S$, thus that $B_r(0) \subset S$.
    Conversely, assume $x \in S$, which means $x = ry, \lvert \lvert y \rvert \rvert < 1$.
    But then $\lvert \lvert x \rvert \rvert = \lvert \lvert r y \rvert \rvert =  \lvert r \rvert \cdot \lvert \lvert y \rvert \rvert < r$, i.e. that $x \in B_r(0)$, and thus that $S \subset B_r(0)$, completing the proof that $B_r(0) = S$.
\end{solution}

\begin{exercise}{33}
    Limits are unique. [Hint: $d(x, y) \leq d(x, x_n) + d(x_n, y)$.]
\end{exercise}

\begin{solution}
    
    Suppose that a sequence $(x_n)$ in a metric space $M$ converges to two $a, b \in M$ such that $a \neq b$.
    By the definition of metrics, we know then that it must be the case that $d(a, b) > 0$.
    Set $\epsilon = d(a, b) > 0$, in which case by the definition of the limit in a metric space there must exist $N_1, N_2$ such that $d(x_n, a) < \epsilon/4, d(x_n, b) < \epsilon/4$ for $n \geq N_1, n \geq N_2$ respectively.
    If we then pick $n > \max\{N_1, N_2\}$ we have that both of these inequalities hold for $x_n$.
    Recall the triangle inequality for metrics:

    $$d(a, b) \leq d(a, x_n) + d(x_n, b) < \frac{\epsilon}{4} + \frac{\epsilon}{4} = \frac{\epsilon}{2} < \epsilon = d(a, b)$$

    This is a clear contradiction, which means that if a sequence in a metric space has a limit, the limit has to be unique.
\end{solution}

\begin{exercise}{34}
    If $x_n \rightarrow x$ in $(M, d)$, show that $d(x_n, y) \rightarrow d(x, y)$ for any $y \in M$.
    More generally, if $x_n \rightarrow x, y_n \rightarrow y$, show that $d(x_n, y_n) \rightarrow d(x, y)$.
\end{exercise}

\begin{solution}
    
    Pick $\epsilon > 0$.
    Because $x_n r\rightarrow x$, we know that there exists $N > 0$ such that $d(x_n, x) < \epsilon$ for all $n \geq N$.
    By the triangle inequality, for $n \geq N$ we have that:
    $$d(x_n, y) \leq d(x, y) + d(x_n, x) < \epsilon + d(x, y) \implies d(x_n, y) - d(x, y) < \epsilon$$
    By exercise 2 of Chapter 3, we also have that:
    $$\lvert d(x_n, x) - d(y, x) \rvert \leq d(x_n, y) \implies -d(x_n, y) \leq d(x_n, x) - d(x, y) < \epsilon - d(x, y) $$
    $$\implies d(x_n, y) - d(x, y) > -\epsilon$$

    Combining these two inequalities we obtain that $\lvert d(x_n, y) - d(x, y) \rvert < \epsilon$ for all $n > N$, which is precisely the definition of $d(x_n, y) \rightarrow d(x, y)$.

    For the second, more general statement, pick again $\epsilon > 0$.
    Then there exist $N_1, N_2$ such that $d(x_n, x) < \frac{\epsilon}{2}, d(y_n, y) < \frac{\epsilon}{2}$ whenever $n > N_1, n > N_2$ respectively.
    Set then $N = \max\{N_1, N_2\}$.
    By the triangle inequality for $n > N$ we have that:

    $$d(x_n, y_n) \leq d(x_n, x) + d(x, y_n) \leq d(x_n, x) + d(x, y) + d(y, y_n) < d(x, y) + \frac{\epsilon}{2} + \frac{\epsilon}{2}$$
    $$\implies d(x_n, y_n) - d(x, y) < \epsilon$$
    
    $$d(x, y) \leq d(x, x_n) + d(x_n, y) \leq d(x, x_n) + d(x_n, y_n) + d(y_n, y) < \frac{\epsilon}{2} + \frac{\epsilon}{2} + d(x_n, y_n)$$
    $$\implies d(x,y) - d(x_n, y_n) < \epsilon \implies - \epsilon < d(x_n, y_n) - d(x, y)$$

    Putting together these two inequalities results in $\lvert d(x_n, y_n) - d(x,y) \rvert < \epsilon$ for $n > N$, which is precisely the definition of $d(x_n, y_n) \rightarrow d(x, y)$.
\end{solution}

\begin{exercise}{35}
    If $x_n \rightarrow x$, then $x_{n_k} \rightarrow x$ for any subsequence $(x_{n_k})$ of $(x_n)$.
\end{exercise}

\begin{solution}
    
    Since $x_n \rightarrow x$, for any given $\epsilon > 0$ we can always find $N > 0 $ such that for all $n \geq N, d(x_n, x) < \epsilon$.
    Because $(x_{n_k})$ contains infinite terms of $(x_n)$ selected in an order which maintains the order of indices, it must be the case that for $n_k \geq N$, $d(x_{n_k}, x) < \epsilon$.
    We conclude that $x_{n_k} \rightarrow x$.

\end{solution}

\begin{exercise}{36}
    A convergent sequence is Cauchy, and a Cauchy sequence is bounded (that is, the set $\{x_n : n \geq 1\}$ is bounded).
\end{exercise}

\begin{solution}
    
    We begin by showing that a convergent sequence is Cauchy.
    Pick any $\epsilon > 0$.
    Because the sequence converges, say to $x$, there exists $N > 0$ such that whenever $n \geq N$ it is the case that $d(x_n, x) < \epsilon/2$.
    For any two such $n_1, n_2 \geq N$ we then have:
    
    $$d(x_{n_1}, x_{n_2}) \leq d(x_{n_1}, x) + d(x, x_{n_2}) < \epsilon,$$
    
    which is precisely the definition of $x$ being Cauchy.

    Now assume $(x_n)$ is Cauchy.
    Pick e.g. $\epsilon = 1$, and find $N > 0$ such that for $n_1, n_2 \geq N$ it holds that $d(x_{n_1}, x_{n_2}) < \epsilon$.
    Consider the set $\{d(x_n, x_1), 1 \leq n \leq N\}$.
    This has a finite number of non-negative elements, and as such a well-defined non-negative maximum $M$.
    For any $n > N$ we then have that:

    $$d(x_n, x_1) \leq d(x_n, x_N) + d(x_N, x_1) < M + \epsilon,$$

    and therefore all elements of the sequence satisfy $d(x_n, x_1) < M + \epsilon$, i.e. they are contained in the open ball $B_{M + \epsilon}(x_1)$, which is to say that the sequence is bounded.

\end{solution}

\begin{exercise}{37}
    A Cauchy sequence with a convergent subsequence converges.
\end{exercise}

\begin{solution}
    
    Let $(x_n)$ be a Cauchy sequence with a convergent subsequence $(x_{n_k}) \rightarrow x$.
    Pick any $\epsilon > 0$.
    Because $(x_{n_k}) \rightarrow x$, there exists $N_1 > 0$ such that for $n_k \geq N_1$ it holds that $d(x_{n_k}, x) < \epsilon/2$.
    Furthermore, because $(x_n)$ is Cauchy, there exists $N_2 > 0$ such that for $i, j \geq N_2$ it holds that $d(x_{i}, x_{j}) < \epsilon/2$.
    Set $N = \max\{N_1, N_2\}$ and pick any $n > N$.
    Then, by the triangle inequality:

    $$d(x_n, x) \leq d(x_n, x_{N_1}) + d(x_{N_1}, x) < \epsilon/2 + \epsilon/2 = \epsilon$$

    But then this means that $(x_n)$ converges to $x$, which completes the proof.
\end{solution}

\begin{exercise}{39}
    If every subsequence of $(x_n)$ has a \textit{further} subsequence that converges to $x$, then $(x_n)$ converges to x.
\end{exercise}

\begin{solution}

    By way of contradiction, assume $(x_n)$ does not converge to $x$.
    Then there exists $\epsilon > 0$ such that for all $N > 0$ there exists $n \geq N$ such that $d(x_n, x) \geq \epsilon$.
    Consider constructing the following subsequence of $(x_n)$: for any $i > 0$, the $i$-th element of the subsequence equals the first $x_n$ such that $d(x_n, x) \geq \epsilon, n \geq i$ and $x_n$ has not been selected before.
    By the hypothesis above, this is always well-defined.

    As a subsequence of $(x_n)$, by the hypothesis of the exercise this subsequence will have a further subsequence converging to $x$.
    Notice, however, that by construction all elements of this ``further subsequence'' are such that $d(x_{n_{k_l}}, x) \geq \epsilon$, which contradicts convergence to $x$.
    Therefore, $(x_n)$ itself must converge to $x$.
\end{solution}

\begin{exercise}{40}
    Here is a positive result about $l_1$ that may restore your faith in intuition.
    Given any fixed element $x \in l_1$, show that the sequence $x^{(k)} = (x_1, \ldots, x_k, 0, \ldots) \in l_1$ (i.e. the first $k$ terms of $x$ followed by all 0s) converges to $x$ in $l_1$-norm.
    Show that the same holds true in $l_2$, but give an example showing that it fails (in general) in $l_{\infty}$.
\end{exercise}

\begin{solution}
    
    For any given $k$, we make the following observation:

    $$\lvert \lvert x - x^{(k)} \rvert \rvert_1 = \sum_{i=1}^{k} \lvert x_i - x_i \rvert + \sum_{i=k+1}^{\infty} \lvert x_i \rvert = \sum_{i=k+1}^{\infty} \lvert x_i \rvert$$

    Notice now that, since $x \in l_1$, the RHS here equals $\lvert \lvert x \rvert \rvert_1 - \sum_{i=1}^{k} \lvert x_i \rvert$.
    Furthermore, because the $l-1$ norm of $x$ viewed as an infinite series converges, it must be the case that by selecting $k$ large enough, the term $\sum_{i=1}^{k} \lvert x_i \rvert$ can be made to be arbitrarily close to it, which in other words means that the RHS above can be made arbitrarily small.
    Of course, this has the immediate consequence that for any $\epsilon > 0$ we can find $K > 0$ such that for $n \geq K, \lvert \lvert x - x^{(n)} \rvert \rvert_1 < \epsilon$, which means that the sequence $x^{(k)}$ converges to $x$ under the $l_1$-norm.
    
    To show that the same holds for the $l_2$-norm, we compute:

    $$\lvert \lvert x - x^{(k)} \rvert \rvert_2^2 = \sum_{i=1}^{k} \lvert x_i - x_i \rvert^2 + \sum_{i=k+1}^{\infty} \lvert x_i \rvert^2 = \sum_{i=k+1}^{\infty} \lvert x_i \rvert^2$$

    The RHS here equals $\lvert \lvert x \rvert \rvert_2^2 - \sum_{i=1}^{k} \lvert x_i \rvert^2$.
    Because the square of the $l_2$-norm of $x$ viewed as an infinite series converges, it must be the case that $\sum_{i=1}^{k} \lvert x_i \rvert^2$ gets arbitrarily close to it by picking an appropriate $k$.
    The remaining argument is also the same as for $l_1$, which means that $x^{(k)} \rightarrow x$ under $l_2$ as well.

    For the case of $l_{\infty}$, we consider the sequence formed by $x_n = 1 - \frac{1}{n}$.
    This can be easily shown to have $\lvert \lvert x \rvert \rvert_{\infty} = 1$.
    However, for $x^{(k)}$ as defined in the exercise:

    $$\lvert \lvert x - x^{(k)} \rvert \rvert_{\infty} = \sup\{0, x_{k+1}, x_{k+2}, \ldots\} = 1$$

    The reason this holds is because $x_n \rightarrow 1$, and it is thus made clear that $x^{(k)}$ cannot converge to $x$ under the $l_{\infty}$ norm.
\end{solution}

\begin{exercise}{41}
    Given $x, y \in l_2$, recall that $\langle x, y \rangle = \sum_{i=1}^{\infty} x_i y_i$.
    Show that if $x^{(k)} \rightarrow x$ and $y^{(k)} \rightarrow y$ in $l_2$, then $\langle x^{(k)}, y^{(k)} \rangle \rightarrow \langle x, y \rangle$.
\end{exercise}

\begin{solution}
    
    We begin first by verifying that if $a, b, c \in l_2$, then $\langle a + b, c \rangle = \langle a, b \rangle + \langle a, c \rangle$.
    We have that:

    $$\langle a + b, c \rangle = \sum_{i=1}^{\infty}(a_i + b_i)c_i = \sum_{i=1}^{\infty}(a_ic_i + b_ic_i)$$

    Notice that because $a, b, c$ are all in $l_2$, we can safely split this series into a sum of the two series corresponding $\langle a, c \rangle, \langle b, c \rangle$.
    Now we move on to the main part of the proof by picking any $\epsilon > 0$.
    To begin with, for any $k, x^{(k)}, y^{(k)} \in l_2$ so any inner products involving them are well-defined:

    $$\lvert \langle x, y \rangle - \langle x^{(k)}, y^{(k)} \rangle \rvert = \lvert \langle x, y \rangle - \langle x, y^{(k)} \rangle + \langle x, y^{(k)} \rangle - \langle x^{(k)}, y^{(k)} \rangle \rvert = \lvert \langle x, y - y^{(k)} \rangle + \langle x - x^{(k)}, y^{(k)} \rangle \rvert$$
    $$\leq \lvert \langle x, y - y^{(k)} \rangle \rvert + \lvert \langle x - x^{(k)}, y^{(k)} \rangle \rvert \leq \lvert \lvert x \rvert \rvert_2 \cdot \lvert \lvert y - y^{(k)} \rvert \rvert_2 + \lvert \lvert x - x^{(k)} \rvert \rvert_2 \cdot \lvert \lvert y^{(k)} \rvert \rvert_2$$

    Notice now that because $y^{(k)} \rightarrow y, x^{(k)} \rightarrow x$, we can, for this $\epsilon$, find $K > 0$ such that for all $k \geq K, \lvert \lvert x - x^{(k)} \rvert \rvert_2 < \epsilon, \lvert \lvert y - y^{(k)} \rvert \rvert_2 < \epsilon$.
    By the extension of the triangle inequality, it will then also hold that $\bigl \lvert \lvert \lvert y^{(k)} \rvert \rvert_2 - \lvert \lvert y \rvert \rvert_2 \bigr \rvert < \epsilon$.
    Note that here we are using the association of the $l_2$ norm to the corresponding metric in $l_2$.
    Coupled with these observations, the above inequality implies:

    $$\lvert \langle x, y \rangle - \langle x^{(k)}, y^{(k)} \rangle \rvert < \lvert \lvert x \rvert \rvert_2 \cdot \epsilon + \epsilon \cdot (\lvert \lvert y \rvert \rvert_2 + \epsilon)$$

    Now note that the RHS is a function of $\epsilon$ that tends to zero as $\epsilon$ tends to zero, which means (by the ``elegance is not required'' theorem from Hubbard \& Hubbard) that indeed $\langle x^{(k)}, y^{(k)} \rangle \rightarrow \langle x, y \rangle$.
\end{solution}

\begin{exercise}{42}
    Two metrics $d, \rho$ on a set $M$ are said to be \textbf{equivalent} if they generate the same convergent sequences; that is, $d(x_n, x) \rightarrow 0$ iff $\rho(x_n, x) \rightarrow 0$.
    If $d$ is any metric on $M$, show that the metrics $\rho, \sigma, \tau$ defined in Exercise 6 are all equivalent to $d$.
\end{exercise}

\begin{solution}
    
    We begin with $\rho(x, y) = \sqrt{d(x, y)}$.
    Suppose that some sequence $x_n$ converges to $x$ under $d$. 
    Then, pick any $\epsilon > 0$ and find $N > 0$ such that for $n \geq N, d(x_n, x) < \epsilon^2$.
    We have then that $\rho(x_n, x) = \sqrt{d(x_n, x)} < \sqrt{\epsilon^2} = \epsilon$, which of course shows that $\rho(x_n, x) \rightarrow 0$, i.e. that $x_n$ converges to $x$ under $\rho$.
    Conversely, if $\rho(x_n, x) \rightarrow 0$ for some $(x_n), x$, then for any given $\epsilon > 0$, pick $N > 0$ such that for $n \geq N, \rho(x_n, x) < \sqrt{\epsilon} \implies \sqrt{d(x_n, x)} < \sqrt{\epsilon}\implies d(x_n, x) < \epsilon$, which means that $d(x_n, x) \rightarrow 0$.

    Now, for $\sigma(x, y) = d(x, y)/(1 + d(x, y))$, suppose again $d(x_n, x) \rightarrow 0$ and pick first $\epsilon < 1$.
    Then there exists $N > 0$ such that for $n \geq N, d(x_n, x) < \epsilon/(1 - \epsilon) \implies d(x_n, x) - \epsilon d(x_n, x) < \epsilon \implies d(x_n, x) < \epsilon (1 + d(x_n, x)) \implies \sigma(x_n, x) < \epsilon$.
    Notice that for $\epsilon \geq 1$, it suffices to pick, say, the $N > 0$ for which the above holds when e.g. $\epsilon' = 1/2$. 
    This shows that $\sigma(x_n, x) \rightarrow 0$.
    Conversely, for $\sigma(x_n, x) \rightarrow 0$, for any given $\epsilon > 0$ there exists $N > 0$ such that for $n \geq N, \sigma(x_n, x) < \epsilon \implies d(x_n, x)/(1 + d(x_n, x)) < \epsilon \implies d(x_n, x) - \epsilon d(x_n, x) < \epsilon \implies d(x_n, x) (1 - \epsilon) < \epsilon$.
    Notice that for $\epsilon < 1$, we obtain $d(x_n, x) < \epsilon/(1 - \epsilon)$, which means that $d(x_n, x)$ is bound by a function of $\epsilon > 0$ which tends to zero as $\epsilon \rightarrow 0$.
    It furthermore holds that this function bounds $d(x_n, x)$ when $\epsilon < 1$.
    Also, for $\epsilon \geq 1$, it holds trivially that $d(x_n, x)/(1 + d(x_n, x))  < \epsilon$.
     By defining $u(\epsilon) = \epsilon/(1 - \epsilon), \epsilon < 1$ and $u(\epsilon) = 1$ for $\epsilon \geq 1$, we have a function that fulfills the conditions of the ``elegance is not required'' theorem for limits, thus showing that $d(x_n, x) \rightarrow 0$.

     Lastly, for $\tau(x, y) = \min\{d(x, y), 1\}$, if $d(x_n, x) \rightarrow 0$, then for $\epsilon > 1$ it trivially holds that $\tau(x_n, x) < \epsilon$ for $n \geq 1$, while for $\epsilon \leq 1$ we have that for sufficiently large $N > 0, d(x_n, x) < \epsilon$ and thus $\tau(x_n, x) = d(x_n, x) < \epsilon$, thus showing that $\tau(x_n, x) \rightarrow 0$.
     Conversely, if $\tau(x_n, x) \rightarrow 0$ then for any given $\epsilon > 0$, there exists $N > 0$ such that for $n \geq N, \tau(x_n, x) < \epsilon \implies \min\{d(x_n, x), 1\} < \epsilon$, thus for $\epsilon < 1$ it has to be that $d(x_n, x) < \epsilon$, which is enough to show that $d(x_n, x) \rightarrow 0$ (since for larger $\epsilon, N$ can trivially be found by using $\epsilon' < 1$).
\end{solution}

\begin{exercise}{43}
    Show that the usual metric on $\mathbb{N}$ is equivalent to the discrete metric.
    Show that any metric on a \textit{finite} set is equivalent to the discrete metric.
\end{exercise}

\begin{solution}
    
    As a reminder, for $x, y \in \mathbb{N}$ the discrete metric is $d(x, y) = 1$ iff $x \neq y$ and $d(x, y) = 0$ otherwise.
    The usual metric on $\mathbb{N}$ is $\rho(x, y) = \lvert x - y \rvert$. 
    To show that the two metrics are equivalent, suppose first that $x_n \rightarrow x$ under $d$.
    In order for this to happen, notice that for any given $\epsilon > 0$ we must be able to find $N > 0$ such that for $n \geq N, d(x_n, x) < \epsilon$.
    The definition of $d$ then dictates that in fact after some $N, x_n$ are always equal to $x$.
    Then it is clear that $\rho(x_n, x) = 0$ for $n \geq N$, meaning of course that $\rho(x_n, x) \rightarrow 0$.
    Conversely, assume $x_n \rightarrow x$ under $\rho$.
    Once again, for any given $\epsilon > 0$ we can find $N > 0$ such that for $n \geq N, \rho(x_n, x) < \epsilon$.
    Choose $\epsilon = 1/2$.
    Then $\rho(x_n, x) < 1/2 \implies \lvert x_n - x \rvert < 1/2$ means that in fact $x_n = x$, since these are natural numbers.
    Then it is of course true that $d(x_n, x) \rightarrow 0$ as well.

    For the second part of the exercise, let $M = \{x_1, x_2, \ldots, x_N\}$ be a finite set, $d$ be the discrete metric and $\rho$ be any metric on it.
    Suppose the sequence $(y_n)$ is such that 
    $d(y_n, x_i) \rightarrow$ for some $x_i \in M$.
    By the same observation as above, for $n \geq K$ for some $K$ it must hold that $y_n = x_i$.
    Then it must be the case that \textit{any} metric $\rho$, purely by being a metric, must satisfy $\rho(y_n, x_i) = 0$.
    This shows that $\rho(y_n, x_i) \rightarrow 0$.
    Conversely, if $\rho(y_n, x_i) \rightarrow 0$, then for any $\epsilon > 0$ there exists $K > 0$ such that for $n \geq K, \rho(y_n, x_i) < \epsilon$.
    Pick then $\epsilon = \min\{\rho(x_j, x_i), j = 1, 2, \ldots, N\, j \neq i \}$, which is well-defined due to $M$ being finite.
    Then the fact that $\rho(y_n, x_i)$ eventually becomes less than $\epsilon$ implies that it must be the case that after some $N, y_n = x_i$, such that the metric becomes zero.
    Consequently, it also holds that $d(y_n, x_i)\rightarrow 0$.

\end{solution}

\begin{exercise}{44}
    Show that the metrics induced by $\lvert \lvert \cdot \rvert \rvert_1, \lvert \lvert \cdot \rvert \rvert_2, \lvert \lvert \cdot \rvert \rvert_{\infty}$ on $\mathbb{R}^n$ are all equivalent.
    [Hint: see exercise 18.]
\end{exercise}

\begin{solution}
    
    Call the corresponding metrics $d_1, d_2, d_{\infty}$ respectively.
    Let $(x_n)$ be a sequence in $\mathbb{R}^n$ that converges to $x$ under $d_2$.
    Pick any $\epsilon > 0$ and find $N$ such that $d_2(x_n, x) < \epsilon$ for $n \geq N$.
    By exercise 18, we have that:

    $$d_{\infty}(x_n, x) = \lvert \lvert x_n - x \rvert \rvert_{\infty} \leq \lvert \lvert x_n - x \rvert \rvert_2 = d_2(x_n, x) < \epsilon,$$

    which shows that $d_{\infty}(x_n, x) \rightarrow 0$.
    Also by exercise 18, we have that:

    $$d_1(x_n, x) = \lvert \lvert x_n - x \rvert \rvert_1 \leq \sqrt{n} \lvert \lvert x_n - x \rvert \rvert_2 = \sqrt{n} d_2(x_n, x) < \sqrt{n} \epsilon,$$

    and we once again refer to the ``elegance is not required'' theorem which guarantees that $d_1(x_n, x) \rightarrow 0$.

    Now suppose $(x_n)$ converges to $x$ under $d_1$.
    The bound $\lvert \lvert x \rvert \rvert_2 \leq \lvert \lvert x \rvert \rvert_1$ from exercise 18 and a process almost identical to the above show that it converges under $d_2$ as well, and then of course under $d_{\infty}$.
    Lastly, the bound $\lvert \lvert x \rvert \rvert_1 \leq n \lvert \lvert x \rvert \rvert_{\infty}$ shows that convergence under $d_{\infty}$ implies convergence under $d_1$, thus completing the ``implication circle'' that indicates that all three metrics are equivalent.
\end{solution}

\begin{exercise}{45}
    We say that two norms on the same vector space $X$ are equivalent if the metrics they induce are equivalent.
    Show that $\lvert \lvert \cdot \rvert \rvert$ and $\lvert \lvert \lvert \cdot \rvert \rvert \rvert$ are equivalent on $X$ iff they generate the same sequences tending to 0; that is, $\lvert \lvert x_n \rvert \rvert \rightarrow 0$ iff $\lvert \lvert \lvert x_n \rvert \rvert \rvert \rightarrow 0$.
\end{exercise}

\begin{solution}
    \newcommand{\trvert}[1]{\lvert \lvert \lvert #1 \rvert \rvert \rvert}

    $\implies$: We call the induced metrics of $\lvert \lvert \cdot \rvert \rvert, \trvert{\cdot}, d_1$ and $d_2$ respectively. 
    Suppose $\lvert \lvert \cdot \rvert \rvert$ is equivalent to $\trvert{\cdot}$.
    Pick any $(x_n)$ such that it converges to 0 under $d_1$.
    This means that $d(x_n, 0) \rightarrow 0$, and thus also that $d_2(x_n, 0) \rightarrow 0$ (by the equivalency of the metrics).
    Obviously, the argument is symmetric for exchanging the roles of the two norms, which means that $\lvert \lvert x_n \rvert \rvert \rightarrow 0$ iff $\trvert{x_n} \rightarrow 0$.

    $\impliedby$: Now suppose that $\lvert \lvert x_n \rvert \rvert \rightarrow 0$ iff $\trvert{x_n} \rightarrow 0$, and let $(y_n)$ be a sequence that converges to $y$ under $d_1$.
    This means that $\lvert \lvert y_n - y \rvert \rvert \rightarrow 0$.
    In other words, the sequence $(y_n) - y$ tends 0 under $\lvert \lvert \cdot \rvert \rvert$, and thus by our hypothesis also under $\trvert{\cdot}$, which directly translates to $d_2(y_n, y) \rightarrow 0$, i.e., $(y_n)$ converges to $y$ under $d_2$.
    Once again, the argument is exactly symmetric for exchanging the roles of the two norms, and this leads us to conclude that the metrics they induce are equivalent, and thus that the norms are also equivalent.
\end{solution}

\newpage

\begin{exercise}{46}
    Given two metric spaces $(M, d)$ and $(N, \rho)$, we can define a metric on the product $M \times N$ in a variety of ways.
    Our only requirement is that a sequence of pairs $(a_n, x_n)$ in $M \times N$ should converge precisely when both coordinate sequences $(a_n), (x_n)$ converge in $(M, d), (N, \rho)$ respectively.
    Show that each of the following define metrics on $M \times n$ that enjoy this property and that all three are equivalent:

    $$d_1((a, x), (b, y)) = d(a, b) + \rho(x, y)$$
    $$d_2((a, x), (b, y)) = (d(a, b)^2 + \rho(x, y)^2)^{1/2}$$
    $$d_{\infty}((a, x), (b, y)) = \max\{d(a, b), \rho(x, y)\}$$

    Henceforth, any implict reference to ``the'' metric on $M \times N$, sometimes called the \textbf{product metric}, will mean one of $d_1, d_2, d_{\infty}$.
    Any one of them will serve equally well; use whichever looks most convenient for the argument at hand.
\end{exercise}

\begin{solution}
    
    We begin by showing that $d_1$ has the property mentioned.
    First, suppose $(x_n) \in M, (y_n) \in N$ converge to $x, y$ respectively under $d, \rho$.
    Pick $\epsilon > 0$ and $N, M > 0$ such that for $n \geq N, m \geq M, d(x_n, x) < \epsilon/2, \rho(y_n, y) < \epsilon/2$.
    Then, for the sequence $(x_n, y_n)$, for $n \geq \max\{N, M\}$ we have that:

    $$d_1((x_n, y_n), (x, y)) = d(x_n, x) + \rho(y_n, y) < \epsilon,$$

    which of course shows that $(x_n, y_n) \rightarrow (x, y)$ under $d_1$.

    Conversely, let's assume that $((x_n, y_n))$ is a sequence of ordered pairs in $M \times N$ that converges to $(x, y)$ under $d_1$.
    Then, for any given $\epsilon > 0$ we can find $N > 0$ such that for $n \geq N$, $d_1((x_n, y_n), (x, y)) < \epsilon \implies d(x_n, x) + \rho(y_n, y) < \epsilon$.
    Because a metric always yields non-negative values, each of $d(x_n, x) < \epsilon, \rho(y_n, y) < \epsilon$ must hold for $n \geq N$, which shows that $d(x_n, x) \rightarrow 0, \rho(y_n, y) \rightarrow 0$.

    For $d_2$, observe that not much will change in the proof: in one direction, we simply need to adjust the chosen $\epsilon$ for $x_n, y_n$ based on the squares and square roots, and in the other we will merely have to take squares of both sides of an inequality.

    For $d_{\infty}$, in one direction it suffices to make both $d(x_n, x), \rho(y_n, y)$ less than $\epsilon$, and then their max also satisfies this.
    In the other, for a ``target'' $\epsilon$, start from $d_{\infty}$ with $\epsilon$ and observe that the max enforces both $d(x_n, x), \rho(y_n, y)$ to satisfy this.

    Hence we only need to show the equivalency of the the three.
    Suppose $(x_n, y_n)$ converges to $(x, y)$  under any of them.
    Then we already showed that $x_n \rightarrow x, y_n \rightarrow y$ under $d, \rho$ respectively.
    But we also showed that for any of the other two product metrics, this is enough to guarantee that $(x_n, y_n)$ converges to $(x, y)$ under them, thus showing that they are in fact equivalent.

\end{solution}