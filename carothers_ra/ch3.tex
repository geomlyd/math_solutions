\chapter{Metrics and Norms}

\section{Metric Spaces}


\begin{exercise}{2}
    If $d$ is a metric on $M$, show that $\lvert d(x, z) - d(y, z) \rvert \leq d(x, y) $ for any $x, y, z \in M$.
\end{exercise}

\begin{solution}
    
    First, we apply the triangle inequality as follows:
    $$d(x, z) \leq d(x, y) + d(y ,z) \implies d(x, z) - d(y, z) \leq d(x, y)$$
    Another application of it yields:
    $$d(y, z) \leq d(y, x) + d(x, z) \implies -d(y, x) \leq d(x, z) - d(y, z) \implies -d(x, y) \leq d(x, z) - d(y,z),$$
    where we used the symmetry property of metric $d$.
    Putting the two inequalities together results in:
    $$\lvert d(x, z) - d(y, z) \rvert \leq d(x, y)$$
\end{solution}

\begin{exercise}{3}
    As it happens, some of our requirements for a metric are redundant.
    To see why this is so, let $M$ be a set and suppose $d: M \times M \rightarrow \mathbb{R}$ satisfies $d(x, y) = 0$ if and only if $x = y$, and $d(x, y) \leq d(x, z) + d(y, z)$ for all $x, y, z \in M$.
    Prove that $d$ is a metric; that is, show that $d(x, y) \geq 0$ and $d(x, y) = d(y, x)$ hold for all $x, y$.
\end{exercise}

\begin{solution}
    
    Pick any two $x, y \in M$.
    We know then that for any $z \in M$ it holds that $d(x, z) \leq d(x, y) + d(z, y)$.
    More specifically, this holds for $z = x$, in which case we obtain:
    $$d(x, x) \leq d(x, y) + d(x, y) \implies 0 \leq 2d(x, y),$$
    which means that for any two $x, y \in M, d(x, y) \geq 0$.

    For the symmetry propety, once again pick any two $x, y \in M$.
    Then $d(x, y) \leq d(x, z) + d(y, z)$ for any $z$.
    Pick then $z = x$, in which case we obtain $d(x, y) \leq d(x, x) + d(y, x) \implies d(x, y) \leq d(y, x)$.
    By exchanging the roles of $x, y$, we have that $d(y, x) \leq d(y, z) + d(x, z)$ for any $z$.
    Pick $z = y$ to observe that $d(y, x) \leq d(y, y) + d(x, y) \implies d(y, x) \leq d(x, y)$.
    But then $d(x, y) \leq d(y, x) \leq d(x, y)$, thus the only possibility is that $d(x, y) = d(y, x)$.
    Therefore $d$ is indeed a metric.
\end{solution}

\newpage
    
\begin{exercise}{7}
    Here is a generalization of exercises 5 and 6.
    Let $f: [0, \infty) \rightarrow [0, \infty)$ be increasing and satisfy $f(0) = 0$, and $f(x) > 0$ for all $x > 0$.
    If $f$ also satisfies $f(x + y) \leq f(x) + f(y)$ for all $x, y \geq 0$, then $f \circ d$ is a metric whenever $d$ is a metric.
    Show that each of the following conditions is sufficient to ensure that $f(x + y) \leq f(x) + f(y)$ for all $x, y \geq 0$:

    (a) $f$ has a second derivative satisfying $f'' \leq 0$

    (b) $f$ has a decreasing first derivative

    (c) $f(x)/x$ is decreasing for $x > 0$
    
    [Hint: First show that (a) $\implies$ (b) $\implies$ (c).]
\end{exercise}

\begin{solution}
    
    As indicated in the hint, we first show that (a) $\implies$ (b) $\implies$ (c).
    If, after that, we can show that (c) implies the triangle inequality for $f$, we know also that any of (a), (b) imply it as well (since they imply (c)).

    (a) $\implies$ (b): Since $f'' \leq 0$, $f$ more specifically has a first derivative on every point of $[0, \infty)$.
    From calculus 1, it's known also that this implies that $f'$ is decreasing (one would prove this via the Mean Value Theorem).

    (b) $\implies$ (c): (b) implies that $f$ is differentiable, thus $g(x) = f(x)/x, x > 0$ is also differentiable, with $g'(x) = \frac{f'(x)x - f(x)}{x^2}$.
    If we can show that the numerator here is non-positive for $x > 0$, we'll have shown that $g$ is decreasing.
    Pick any $x > 0$, and apply the Mean Value Theorem on $f$ in the interval $[0, x]$.
    This means that there exists $y \in (0, x)$ such that:
    $$f'(y) = \frac{f(x) - f(0)}{x - 0} = \frac{f(x)}{x},$$
    where we used the fact that $f(0) = 0$.
    Therefore, we have that:
    $$f'(x)x - f(x) = f'(x)x - xf'(y) = x(f'(x) - f'(y)) \leq 0,$$
    since $f'$ is decreasing and $y < x$.
    This completes the proof that (b) $\implies$ (c).

    We now need to show that the triangle inequality for $f$ follows from (c).
    Pick any two $x, y \geq 0$.
    If any of them, or both, are zero, the triangle inequality follows from the fact that $f(0) = 0$.
    If $x, y \neq 0$, then we know that $x \leq x + y, y \leq x + y$, and thus that $\frac{f(x)}{x} \geq \frac{f(x + y)}{x + y}, \frac{f(y)}{y} \geq \frac{f(x + y)}{x + y}$ (by the fact that $f(x)/x$ is decreasing). 
    We can rewrite these as:
    $$(x + y)f(x) \geq xf(x+ y), (x+ y)f(y) \geq yf(x + y),$$
    and now we can add them to obtain that:
    $$(x + y)(f(x) + f(y)) \geq f(x + y)(x + y) \implies f(x) + f(y) \geq f(x + y),$$
    which is of course what we wanted to show.
\end{solution}

\begin{exercise}{12}
    Check that $d(f, g) = \max_{a \leq t \leq b} \lvert f(t) - g(t) \rvert$ defines a metric on $C[a, b]$, the collection of all continuous, real valued functions defined on the closed interval $[a, b]$.
\end{exercise}

\begin{solution}
    
    Note that here continuity is important to ensure that the difference of any two functions is continuous, and as such achieves a maximum value in any closed interval.
    Using exercise 3, we need to check the following:
    \begin{itemize}
        \item One, that $d(f, g) = 0$ iff $f = g$.
        Suppose first that $d(f, g) = 0$, that is, $\max_{a \leq b} \lvert f(t) - g(t) \rvert = 0$.
        If $f, g$ were not equal, there would exist $x \in [a, b]$ such that $f(x) \neq g(x)$.
        Clearly then $\max_{a \leq t \leq b} \lvert f(t) - g(t) \rvert \geq \lvert f(x) - g(x) \rvert > 0$, which is a contradiction.
        Therefore $f = g$.
        In the other direction, if $f = g$, we equivalently have that $f - g$ is the zero function on $[a, b]$, and therefore $d(f, g) = \max_{a \leq t \leq b} \lvert f(t) - g(t) \rvert = 0$.
        \item Two, that for any three $f, g, h \in C[a, b], d(f, h) \leq d(f, g) + d(g, h)$.
        We have first that $d(f, h) = \max_{a \leq t \leq b} \lvert f(t) - h(t) \rvert$.
        Suppose that this function achieves its maximum value on $t_M \in [a, b]$ (if there are more than one, pick any).
        Then $d(f, h) = \lvert f(t_M) - h_(t_M) \rvert = \lvert f(t_M) - g(t_M) + g(t_M) - h(t_M) \rvert \leq \lvert f(t_M) - g(t_M) \rvert + \lvert g(t_M) - h(t_M) \rvert$.
        By definition, it holds that $\lvert f(t_M) - g(t_M) \rvert \leq \max_{a \leq t \leq b} \lvert f(t) - g(t) \rvert = d(f, g)$, and similarly $\lvert g(t_M) - h(t_M) \rvert \leq d(g, h)$.
        We therefore obtain that $d(f, h) \leq d(f, g) + d(g, h)$.
    \end{itemize}
    Therefore $d$ is indeed a metric on $C[a, b]$.
\end{solution}

\begin{exercise}{14}
    We say that a subset $A$ of a metric space is \textbf{bounded} if there is some $x_0 \in M$ and some constant $C < \infty$ such that $d(a, x_0) \leq C$ for all $a \in A$.
    Show that a finite union of bounded sets is again bounded.
\end{exercise}

\begin{solution}
    
    Suppose we have the finite union 
    $$U = A_1 \cup A_2 \cup \ldots \cup A_n$$
    of $n$ bounded subsets of a metric space $M$.
    Let then $x_1, x_2, \ldots, x_n, C_1, \ldots, C_n$ be such that $d(a, x_i) \leq C$ for $a \in A_i$.
    Form the set $\{d(x_1, x_1), d(x_1, x_2), \ldots, d(x_n, x_1)\}$, which, crucially, has a finite number of elements, and therefore also has a maximum element, $M$.
    By the same reasoning, there exists $C$ such that $C = \max\{C_1, \ldots, C_n\}$.
    Now pick any $a \in U$, which must belong in at least one $A_i$.
    Therefore:
    $$d(a, x_1) \leq d(a, x_i) + d(x_i, x_1) \leq C_i + M \leq C + M$$
    If we thus set $x_U = x_1$ and $C_U = C + M$, we have shown precisely that $U$ is bounded.
\end{solution}

\begin{exercise}{15}
    We define the \textbf{diameter} of a nonempty subset $A$ of $M$ by $\text{diam}(A) = \sup\{d(a, b) : a, b \in A\}$.
    Show that $A$ is bounded if and only if $\text{diam}(A)$ is finite.
\end{exercise}

\begin{solution}
    
    $\implies$: Suppose first that $A$ is bounded.
    Then there exists $x_0 \in M, C \in \mathbb{R}$ such that $d(a, x_0) \leq C$ for all $a \in A$.
    Pick any two $a, b \in A$.
    We then have that $d(a, b) \leq d(a, x_0) + d(x_0, b) \leq 2C$.
    Therefore the set $\{d(a, b): a, b \in A$\} has an upper bound, namely, $2C$, and therefore also a finite least upper bound, which means precisely that $\text{diam}(A)$ is finite.

    $\impliedby$: Now suppose $\text{diam}(A) = C \in \mathbb{R}$.
    Fix an $a \in A$ (which exists since $A$ is nonempty).
    Pick any $b \in A$, in which case we have that:
    $$d(a, b) \in \{d(x, y): x, y \in A\} \implies d(a, b) \leq \text{diam}(A) = C$$
    If thus set $x_0 = a$ and $C$ the respective constant, we see that the definition of $A$ being bounded is fulfilled.
\end{solution}