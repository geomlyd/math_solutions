\usepackage{comment} % enables the use of multi-line comments (\ifx \fi) 
\usepackage{lipsum} %This package just generates Lorem Ipsum filler text. 
\usepackage{fullpage} % changes the margin
\usepackage[a4paper, total={7in, 10in}]{geometry}

%\usepackage{tgadventor} % The font for the entire document can be changed here
%\usepackage{courier}
%\usepackage{charter}
\usepackage{tgcursor}

\usepackage{mathrsfs} % This allows for the use of \mathscr{P} which is the same font used for the powerset notation in Velleman's book.

\newtheorem{corollary}{Corollary}
\usepackage{graphicx}
\usepackage{tikz}
\usetikzlibrary{arrows}
\usepackage{verbatim}
\usepackage{float}
\usepackage{tikz}
    \usetikzlibrary{shapes,arrows}
    \usetikzlibrary{arrows,calc,positioning}

    \tikzset{
        block/.style = {draw, rectangle,
            minimum height=1cm,
            minimum width=1.5cm},
        input/.style = {coordinate,node distance=1cm},
        output/.style = {coordinate,node distance=4cm},
        arrow/.style={draw, -latex,node distance=2cm},
        pinstyle/.style = {pin edge={latex-, black,node distance=2cm}},
        sum/.style = {draw, circle, node distance=1cm},
    }
\usepackage{xcolor}
\usepackage{mdframed}
\usepackage[shortlabels]{enumitem}
\usepackage{indentfirst}
\usepackage{hyperref}
    
\renewcommand{\thesubsection}{\thesection.\alph{subsection}}

\newenvironment{exercise}[2][Exercise]
    { \begin{mdframed}[backgroundcolor=gray!20] \textbf{#1 #2} \\}
    {  \end{mdframed}}

% Define solution environment

%%%%%%%%%%%%%%%%%%%%%%%%%%%%%%%%%%%%%%%%%%%%%%%%%%%%%%%%%%%%%%%%%%%%%%%%%%%%%%%%%%%%%%%%%%%%%%%%%%%%%%%%%%%%%%%%%%%%%%%%%%%%%%%%%%%%%%%% Original packages, custom environments, and custom commands below
  \usepackage{amsmath,amsfonts,amsthm, amssymb}
  \usepackage{fullpage}
  \usepackage{array}
  \usepackage[all,textures]{xy}
  \usepackage{graphicx}
  \usepackage{alltt}
  \usepackage{listings}
  \usepackage{float}
  \usepackage{tabu}
  \usepackage{longtable}
  \usepackage{lipsum}
  
  \theoremstyle{plain}
  \newtheorem{innercustomgeneric}{\customgenericname}
\providecommand{\customgenericname}{}
\newcommand{\newcustomtheorem}[2]{%
  \newenvironment{#1}[1]
  {%
   \renewcommand\customgenericname{#2}%
   \renewcommand\theinnercustomgeneric{##1}%
   \innercustomgeneric
  }
  {\endinnercustomgeneric}
}

\newcustomtheorem{theorem}{Theorem}
\newcustomtheorem{definition}{Definition}
\newcustomtheorem{lemma}{Lemma}


  \setlength{\parindent}{0pt}
 
 \newcommand{\pow}{\mathscr{P}}
 \newcommand{\ffam}{\mathcal{F}}
 \newcommand{\family}{\mathcal{F}}
 \newcommand{\gfam}{\mathcal{G}}
 \newcommand{\Ran}{\text{Ran}}
 \newcommand{\Dom}{\text{Dom}}
\newcommand{\solution}{\textit{Solution.}}
 \newcommand{\real}{\mathbb{R}}
 \newcommand{\integer}{\mathbb{Z}}
 \newcommand{\rational}{\mathbb{Q}}
 \newcommand{\nat}{\mathbb{N}}
