\chapter{Functions on Euclidean Space}

\section{Norm and Inner Product}

\begin{exercise}{7}
    A linear transformation $T: \mathbb{R}^n \rightarrow \mathbb{R}^n$ is \textbf{norm preserving} if $\lvert T(x) \rvert = \lvert x \rvert$ and \textbf{inner product preserving} if $\langle Tx, Ty \rangle = \langle x, y \rangle$.

    a. Prove that $T$ is norm preserving if and only if $T$ is inner product preserving.

    b. Prove that such a linear transformation $T$ is 1-1 and $T^{-1}$ is of the same sort.
\end{exercise}

\begin{solution}

    a. Let $x, y \in \mathbb{R}^n$. Firstly, we observe the following relationship between the standard Euclidean norm and inner product:
    $$\langle x + y, x + y \rangle = \langle x, x \rangle + 2\langle x, y \rangle + \langle y, y \rangle \implies \lvert \lvert x+ y \rvert \rvert^2 = \lvert \lvert x \rvert \rvert^2 + \lvert \lvert y \rvert \rvert^2 + 2\langle x, y \rangle$$
    $$\implies \langle x, y \rangle = \frac{\lvert \lvert x + y \rvert \rvert^2 - \lvert \lvert x \rvert \rvert^2 - \lvert \lvert y \rvert \rvert^2}{2}$$

    Thus, we have that if $T$ is norm-preserving, then for any two $x, y \in \mathbb{R}^n$ it holds that
    $$\lvert \lvert T(x) \rvert \rvert = \lvert \lvert x \rvert \rvert, \lvert \lvert T(y) \rvert \rvert = \lvert \lvert y \rvert \rvert, \lvert \lvert T(x+y) \rvert \rvert = \lvert \lvert x + y\rvert \rvert$$
     Therefore:
    $$\langle T(x), T(y) \rangle  = \frac{\lvert \lvert T(x) + T(y) \rvert \rvert^2 - \lvert \lvert T(x) \rvert \rvert^2 - \lvert \lvert T(y) \rvert \rvert^2}{2} = $$
    $$ = \frac{\lvert \lvert T(x + y) \rvert \rvert^2 - \lvert \lvert x \rvert \rvert^2 - \lvert \lvert y \rvert \rvert^2}{2} = \frac{\lvert \lvert x + y \rvert \rvert^2 - \lvert \lvert x \rvert \rvert^2 - \lvert \lvert y \rvert \rvert^2}{2} = \langle x, y \rangle$$

    , therefore $T$ is also inner product preserving. For the other direction, if $T$ is inner product preserving, then more specifically for any $x \in \mathbb{R}^n$:
    $$\lvert \lvert T(x) \rvert \rvert^2 = \langle T(x), T(x) \rangle = \langle x, x \rangle = \lvert 
    \lvert x \rvert \rvert^2$$

    , which means $T$ is norm preserving.

    b. Firstly, recall that for linear operators it suffices to show that they are injective in order to show that they are invertible. Therefore, we only need to examine $T(x) = 0$. $T(x) = 0$ by the definition of the inner product is equivalent to $\lvert \lvert T(x) \rvert \rvert = 0$. But then because $T$ is norm preserving, we have that $\lvert \lvert x \rvert \rvert = \lvert \lvert T(x) \rvert \rvert = 0$, and by the same equivalency as before this means $x = 0$, therefore $T$ is injective. This means that it is also invertible, i.e.\ $T^{-1}$ is well defined. 

    Now, for any $x \in \mathbb{R}^n$, we have that $T^{-1}(x) = y$, such that $T(y) = x$. Since $T$ is norm preserving we have that:
    $$\lvert \lvert T(y) \rvert \rvert = \lvert \lvert y \rvert \rvert \implies \lvert \lvert x \rvert \rvert = \lvert \lvert T^{-1}(x) \rvert \rvert$$

    , therefore $T^{-1}$ is also norm preserving.
\end{solution}

\begin{exercise}{8}
    If $x, y \in \mathbb{R}^n$ are non-zero, the \textbf{angle} between $x$ and $y$, denoted $\angle (x, y)$, is defined as $\text{arccos}(\frac{\langle x, y \rangle}{\lvert \lvert x \rvert \rvert \cdot \lvert \lvert y \rvert \rvert})$, which makes sense by the Cauchy-Schwarz inequality. The linear transformation $T$ is \textbf{angle preserving} if $T$ is 1-1, and for $x, y \neq 0$ we have $\angle(Tx, Ty) = \angle(x, y)$.

    a. Prove that if $T$ is norm preserving, then $T$ is angle preserving.

    b. If there is a basis $x_1, \ldots, x_n \in \mathbb{R}^n$ and numbers $\lambda_1, \ldots \lambda_n$ such that $T(x_i) = \lambda_i x_i$, prove that if $T$ is angle preserving then all $\lvert \lambda_i \rvert$ are equal.

    c. What are all angle preserving $T: \mathbb{R}^n \rightarrow \mathbb{R}^n$?
\end{exercise}

\begin{solution}
    
    a. Suppose $T$ is norm preserving. Then, as we saw in exercise 7, it is also inner product preserving. Then, for $x, y \in \mathbb{R}^n$:
    $$\angle(T(x), T(y)) = \text{arccos}(\frac{\langle T(x), T(y) \rangle}{\lvert \lvert T(x) \rvert \rvert \cdot \lvert \lvert T(y) \rvert \rvert}) = \text{arccos}(\frac{\langle x, y \rangle}{\lvert \lvert x \rvert \rvert \cdot \lvert \lvert y \rvert \rvert}) = \angle(x, y)$$

    b. In order to simplify our computations, we will first convert the basis $x_1, \ldots, x_n$ to a basis $y_1, \ldots, y_n$ such that $y_i = \frac{x_i}{\lvert \lvert x_i \rvert \rvert}$, i.e.\ the new basis consists of unit length vectors. Now consider any two $y_i, y_i$ vectors of this basis. These are clearly non-zero, and also linearly independent, which means that $y_i+y_j, y_i-y_j$ are non-zero as well. Since $T$ is angle preserving, it must hold that:
    
    $$\angle(T(y_i+y_j), T(y_i-y_j)) = \angle(y_i+y_j, y_i-y_j)$$
    $$\implies \text{arccos}(\frac{\langle T(y_i+y_j), T(y_i-y_j) \rangle}{\lvert \lvert T(y_i-y_j) \rvert \rvert \cdot \lvert \lvert T(y_i+y_j) \rvert \rvert}) = \text{arccos}(\frac{\langle y_i+y_j, y_i-y_j \rangle}{\lvert \lvert y_i+y_j \rvert \rvert \cdot \lvert \lvert y_i-y_j \rvert \rvert}$$
    $$\implies \frac{\langle \lambda_i y_i + \lambda_j y_j, \lambda_i y_i - \lambda_j y_j\rangle}{\lvert \lvert \lambda_i y_i - \lambda_j y_j \rvert \rvert \cdot \lvert \lvert \lambda_i y_i + \lambda_j y_j \rvert \rvert} = \frac{1 - \langle y_i, y_j \rangle + \langle y_j, y_i \rangle - 1}{\lvert \lvert y_i+y_j \rvert \rvert \cdot \lvert \lvert y_i - y_j \rvert \rvert}$$
    $$\implies \langle \lambda_i y_i + \lambda_j y_j, \lambda_i y_i - \lambda_j y_j \rangle = 0 \implies \lambda_i^2 - \lambda_i \lambda_j \langle y_i, y_j \rangle + \lambda_i \lambda_j \langle y_i, y_j \rangle - \lambda_j^2 = 0$$
    $$\implies \lambda_i^2 = \lambda_j^2 \implies \lvert \lambda_i \rvert = \lvert \lambda_j \rvert$$

    , where throughout the derivation we used the fact that $\lvert \lvert y_i \rvert \rvert = \lvert \lvert y_j \rvert \rvert = 1$.

    c. By the Polar Decomposition, we know that $T=S\sqrt{T^*T}$ for some isometry $S$. If $T$ is angle preserving, then for any two non-zero $x, y$ it must be the case that:
    $$\frac{\langle S\sqrt{T^*T}(x), S\sqrt{T^*T}(y) \rangle}{\lvert \lvert S\sqrt{T^*T}(x) \rvert \rvert \cdot \lvert \lvert S\sqrt{T^*T}(y) \rvert \rvert} = \frac{\langle x, y \rangle}{\lvert \lvert x \rvert \rvert \cdot \lvert \lvert y \rvert \rvert} \implies \frac{\langle \sqrt{T^*T}(x), \sqrt{T^*T}(y) \rangle}{\lvert \lvert \sqrt{T^*T}(x) \rvert \rvert \cdot \lvert \lvert \sqrt{T^*T}(y) \rvert \rvert} = \frac{\langle x, y \rangle}{\lvert \lvert x \rvert \rvert \cdot \lvert \lvert y \rvert \rvert}$$

    , where we omitted the arccos for brevity and used the properties of isometries. Observe that this means that $\sqrt{T^*T}$ is angle preserving. Additionally, this operator is positive and self-adjoint, which means that by the Real Spectral Theorem there exists an orthonormal basis of $\mathbb{R}^n$ with respect to which its matrix is diagonal. Every entry on the diagonal of this matrix must be positive, since $\sqrt{T^*T}$ is positive. Since these entries are precisely the eigenvalues of $\sqrt{T^*T}$, and since it is angle preserving, by part (b) we obtain that they must all be equal in absolute values, and consequently equal (since they are positive). Therefore, $T = SaI = aS$, for some isometry $S$ and some positive number $a$.

    Conversely, if $T = aS$ for some isometry $S$ and a positive number $a$, the properties of isometries and the definition of angle preserving operators yields that $T$ is indeed angle preserving (we have but to observe that $a^2$ will appear both on the numerator and denominator of the argument of arccos).

    \end{solution}

    \begin{exercise}{9}
        If $0 \leq \theta \leq \pi$, let $T \in \mathbb{R}^2 \rightarrow \mathbb{R}^2$ have the matrix $\begin{pmatrix}
            \text{cos} \theta & \text{sin} \theta \\ -\text{sin} \theta & \text{cos} \theta
        \end{pmatrix}$. Show that $T$ is angle preserving and if $x \neq 0$, then $\angle(x, Tx) = \theta$.
    \end{exercise}

    \begin{solution}

        From Linear Algebra we know that this matrix corresponds to an isometry in $\mathbb{R}^2$. By part (c) of exercise 8 above, $T$ is angle preserving, since it equals an isometry times (a multiple of) the identity. For a non-zero $x = (x_1, x_2)$ we have that:
        $$\angle(x, Tx) = \text{arccos}(\frac{\langle x, Tx \rangle}{\lvert \lvert x \rvert \rvert \cdot \lvert \lvert Tx \rvert \rvert}) = \text{arccos}(\frac{\langle (x_1, x_2), (\text{cos} \theta x_1 + \text{sin} \theta x_2, -\text{sin} \theta x_1 + \text{cos} \theta x_2) \rangle}{\lvert \lvert x \rvert \rvert \cdot \lvert \lvert x \rvert \rvert})$$
        $$= \text{arccos}(\frac{\text{cos} \theta x_1^2 + \text{sin} \theta x_1x_2 - \text{sin} \theta x_1x_2 + \text{cos} \theta x_2^2}{\lvert \lvert x \rvert \rvert^2}) = \text{arccos}(\frac{\text{cos} \theta \lvert \lvert x \rvert \rvert^2}{\lvert \lvert x \rvert \rvert^2}) = \text{arccos}(\text{cos}\theta) = \theta$$
    \end{solution}

    \begin{exercise}{10}
        If $T: \mathbb{R}^m \rightarrow \mathbb{R}^n$ is a linear transformation, show that there is a number $M$ such that $\lvert \lvert T(h) \rvert \rvert \leq M \lvert \lvert h \rvert \rvert$ for $h \in \mathbb{R}^m$. \textit{Hint:} Estimate $\lvert \lvert T(h) \rvert \rvert$ in terms of $\lvert \lvert h \rvert \rvert$ and the entries in the matrix of $T$.
    \end{exercise}

    \begin{solution}

        We have seen in Hubbard and Hubbard that the following analogue of the Cauchy-Schwarz inequality holds for any matrix $A$ and any vector $b$:
        $$\lvert \lvert A b \rvert \rvert \leq \lvert \lvert A \rvert \rvert \cdot \lvert \lvert b \rvert \rvert$$

        , where $\lvert \lvert A \rvert \rvert$ indicates the Frobenius norm of $A$, i.e.\ $\lvert \lvert A \rvert \rvert = \sqrt{\sum_{i, j} A_{i, j}^2}$. Additionally, we know that for any linear transformation $T$, $T(h) = \mathcal{M}(T)\mathcal{M}(h)$, where $\mathcal{M}$ indicates the matrices with respect to the standard bases of $\mathbb{R}^m, \mathbb{R}^n$. Therefore, we have that:
        $$\lvert \lvert T(h) \rvert \rvert = \lvert \lvert \mathcal{M}(T) \mathcal{M}(h) \rvert \rvert \leq \lvert \lvert \mathcal{M}(T) \rvert \rvert \cdot \lvert \lvert h \rvert \rvert$$.

        If we set $M = \lvert \lvert \mathcal{M}(T) \rvert \rvert$ we obtain the inequality requested by the exercise.
    \end{solution}

    \begin{exercise}{13}
        If $x, y \in \mathbb{R}^n$, then $x$ and $y$ are called perpendicular (or orthogonal) if $\langle x, y \rangle = 0$. If $x, y$ are perpendicular, prove that $\lvert \lvert x + y \rvert \rvert^2 = \lvert \lvert x \rvert \rvert^2 + \lvert \lvert y \rvert \rvert^2$.
    \end{exercise}

    \begin{solution}

        By the properties of the inner product and the definition of the associated norm we have that:
        $$\lvert \lvert x + y \rvert \rvert^2 = \langle x + y, x + y \rangle = \langle x, x \rangle + \langle x, y \rangle + \langle y, x \rangle + \langle y, y \rangle = \lvert \lvert x \rvert \rvert^2 + 0 + 0 + \lvert \lvert y \rvert \rvert^2 = \lvert \lvert x \rvert \rvert^2 + \lvert \lvert y \rvert \rvert^2$$
    \end{solution}

    \begin{exercise}{26}
        Let $A = \{(x, y) \in \mathbb{R}^2 : x > 0, 0 < y < x^2\}$.

        (a) Show that every straight line through (0, 0) contains an interval around (0, 0) which is in $\mathbb{R}^2 - A$.

        (b) Define $f: \mathbb{R}^2 \rightarrow \mathbb{R}$ by $f(x) = 0$ if $x \notin A$ and $f(x) = 1$ if $x \in A$. For $h \in \mathbb{R}^2$ define $g_h : \mathbb{R} \rightarrow \mathbb{R}$ by $g_h(t) = f(th)$. Show that each $g_h$ is continuous at 0, but $f$ is not continuous at (0, 0).
    \end{exercise}

    \begin{solution}

        (a) Let us first consider the case of non-vertical lines, i.e.\ lines of the form $y = ax$. The points lying on this line are of the form $(x, ax)$. We are thus interested in examining when is it the case that $(x, ax) \notin A$. This happens precisely whenever $x \leq 0$ or $0 \geq ax$ or $ax \geq x^2$. There are two cases:
        \begin{itemize}
            \item $a \leq 0$: then the second inequality holds for all non-negative $x$. The first inequality holds trivially for all negative $x$. Thus it suffices to pick e.g. the line segment from $(-a, -a^2)$ to $(a, a^2)$, and this will lie entirely in $\mathbb{R}^2 - A$. Geometrically, this line has a negative or zero slope, and as such it never intersects the area between the parabola $y = x^2$ and the horizontal axis.
            \item $a > 0$: Then the first inequality again holds trivially for all non-positive $x$.  For positive $x$, we divide both sides of the third inequality with $x$ to obtain $a \geq x$. Thus for all $ 0 < x \leq a$, the points $(x, ax)$ lie in $\mathbb{R}^2 - A$. Pick, therefore, the line segment from $(-a, -a^2)$ to $(a, a^2)$. Geometrically, we have found the intersection of the line with the parabola $y=x^2$ and, for aesthetic purposes, made the line segment symmetric around $(0, 0)$.
        \end{itemize}
        In either case, the line segment from $(-a, -a^2)$ to $(a, a^2)$ lies in $\mathbb{R}^2 - A$.

        Now we examine the case of the vertical line $x = 0$. Here the line segment $(0, -1)$ to $(0, 1)$ works trivially: indeed, any point with $x = 0$ satisfies the first of the inequalities, $x \leq 0$. We have thus found an interval of the desired form for all lines through the origin.

        (b) The value of $f$ at $(0, 0)$ is $f((0,0)) = 0$ since $(0,0) \notin A$. Consider the sequence of points $i \rightarrow p_i = (\frac{1}{i}, \frac{1}{i^2} - \frac{1}{2i^3})$. As $i \rightarrow \infty$, both the $x$- and the $y$-coordinate here tend to 0. Therefore, this sequence of points in $\mathbb{R}^2$ tends to $(0, 0)$. For $f$ to be continuous, it must be the case that $\text{lim}_{i \rightarrow \infty} f(p_i) = f((0,0)) = 0$. Each $p_i$ clearly has an $x$ coordinate $x_i$ which satisfies $x_i > 0$. Furthermore, its $y$-coordinate is $y_i = \frac{1}{i^2} - \frac{1}{2i^3} = \frac{2i - 1}{2i^3}$. For any integer $i \geq 1$, this is clearly positive. Additionally, $y_i - x_i^2 = -\frac{1}{2i^3} < 0 \implies y_i < x_i^2$. This means that all $p_i$ are in $A$, and thus $f(p_i) = 1$. Obviously, the limit of this constant sequence of $f(p_i)$ as $i \rightarrow \infty$ is $1 \neq 0$, therefore $f$ is not continuous at the origin.

        Now consider any $g_h$ such that $g_h(t) = f(th), h \in \mathbb{R}^2$. $th$ defines a line in $\mathbb{R}^2$ that passes through the origin. From (a) we know that there exists a line segment around which lies entirely in $\mathbb{R}^2 - A$, and thus $f$ evaluated on any point $p$ of it is $f(p) = 0$. Let $p_1, p_2$ be the endpoints of this line segment. Then $p_1 = t_1h, p_2 = t_2h$ and WLOG $t_1 < t_2$ (we proved that this is a non-degenerate line segment, thus $t_1$ cannot equal $t_2$). But then for all $t \in [t_1, t_2]$, $g_h(t) = f(th) = 0$. Clearly, this implies that $g_h$ is continuous at 0, since it is constant on an interval around it (and on 0 itself).
    \end{solution}