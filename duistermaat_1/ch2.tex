\chapter{Differentiation}

\begin{exercise}{1}
    We study some properties of the operation of taking the trace; this will provide some background for the Euclidean norm. All the results obtained below apply, mutatis mutandis, to End$(\mathbb{R}^n)$ as well.

    (i) Verify that $\text{tr}: \text{Mat}(n, \mathbb{R}) \rightarrow \mathbb{R}$ is a linear mapping, and that, for $A, B \in \text{Mat}(n, \mathbb{R})$,

    $$\text{tr} A = \text{tr} A^T, \text{tr} AB = \text{tr} BA, \text{tr} BAB^{-1} = \text{tr} A$$

    if $B \in \textbf{GL}(n, \mathbb{R})$. Deduce that the mapping tr vanishes on \textit{commutators} in $\text{Mat}(n, \mathbb{R})$, that is, for $A, B \in \text{Mat}(n, \mathbb{R})$,

    $$\text{tr}([A, B]) = 0, \text{where    } [A, B] = AB-BA.$$

    Define a bilinear functional $\langle \cdot, \cdot, \rangle : \text{Mat}(n, \mathbb{R}) \times \text{Mat}(n, \mathbb{R}) \rightarrow \mathbb{R}$ by $\langle A, B \rangle = \text{tr}(A^TB)$.

    (ii) Show that $\langle A, B \rangle = \text{tr}(B^TA) = \text{tr}(AB^T) = \text{tr}(BA^T)$.

    (iii) Let $a_j$ be the $j$-th column vector of $A \in \text{Mat}(n, \mathbb{R})$, thus $a_j = Ae_j$. Verify, for $A, B \in \text{Mat}(n, \mathbb{R})$:
    $$A^TB = (\langle a_i, b_j \rangle)_{1 \leq i, j \leq n}, \langle A, B \rangle = \sum_{1 \leq j \leq n} \langle a_j, b_j \rangle$$.

    (iv) Prove that $\langle \cdot, \cdot \rangle$ defines an inner product on $\text{Mat}(n, \mathbb{R})$, and that the corresponding norm equals the Euclidean norm on $\text{Mat}(n, \mathbb{R}$

    (v) Using part (ii), show that the decomposition from Lemma 2.1.4 is an orthogonal direct sum, in other words, $\langle A, B \rangle = 0$ if $A \in \text{End}^+(\mathbb{R}^n)$ and $B \in \text{End}^-(\mathbb{R}^n)$. Verify that $\text{dim End}^{\pm}(\mathbb{R}^n) = \frac{1}{2}(n\pm1)$.

    Define the bilinear functional $B : \text{Mat}(n, \mathbb{R}) \times \text{Mat}(n, \mathbb{R}) \rightarrow \mathbb{R}$ by $B(A, A') = \text{tr}(AA')$.
    
    (vi) Prove that $B$ satisfies $B(A_{\pm}, A_{\pm}) \gtrless 0$, for $0 \neq A_{\pm} \in \text{End}^{\pm}(\mathbb{R}^n)$.

    Finally, we show that the trace is completely determined by some of the properties listed in part (i).

    (vii) Let $\tau : \text{Mat}(n, \mathbb{R}) \rightarrow \mathbb{R}$ be a linear mapping satisfying $\tau(AB) = \tau(BA)$, for all $A, B \in \text{Mat}(n, \mathbb{R})$. Prove
    $$\tau = \frac{1}{n}\tau(I)tr$$

    \textbf{Hint:} Note that $\tau$ vanishes on commutators. Let $E_{ij} \in \text{Mat}(n, \mathbb{R})$ be given by having a single 1 at position $(i, j)$ and zeros elsewhere. Then the statement is immediate from $[E_{ij}, E_{jj}] = E_{ij}$ for $i \neq j$, and $[E_{ij}, E_{ji}] = E_{ii} - E_{jj}$.
    
\end{exercise}

\begin{solution}
    
    (i) We know the following from Linear Algebra (the proofs are simple, and based on the definition of the trace of a matrix):
    \begin{itemize}
        \item For $A, B$ square matrices, it holds that $\text{tr}(A+B) = \text{tr}(A) + \text{tr}(B)$.
        \item For $A$ square matrix and $\lambda \in \mathbb{R}$, it holds that $\text{tr}(\lambda A) = \lambda \text{tr}(A)$.
    \end{itemize}
    Therefore, tr is indeed a linear mapping. For the properties requested, we have that:
    \begin{itemize}
        \item Taking the transpose of a matrix leaves the diagonal unchanged, therefore the sum of the diagonal elements is also unchanged. This equals the trace of the transpose, which means that $\text{tr}(A^T) = \text{tr}(A)$.
        \item This is something that we have also already seen in Linear Algebra, and the proof once more follows from the definition. We have that $\text{tr}(AB) = \sum_i (AB)_{i,i} = \sum_i \sum_j A_{ij}B_{ji}, \text{tr}(BA) = \sum_i (BA)_{i, i} = \sum_i \sum_j B_{ij}A_{ji}$, and these are of course equal, since they sum the exact same products.
        \item Using the above property, for an invertible matrix $B$ we obtain that $\text{tr}(BAB^{-1}) = \text{tr}((BA)B^{-1}) = \text{tr}(B^{-1}(BA)) = \text{tr}(IA) = \text{tr}(A)$.
    \end{itemize}
    We additionally have that $\text{tr}([A, B]) = \text{tr}(AB - BA) = \text{tr}(AB) - \text{tr}(BA) = \text{tr}(AB) - \text{tr}(AB) = 0$.

    (ii) Using the definition and the properties we proved in (i), we have that:
    \begin{itemize}
        \item $\text{tr}(B^TA) = \text{tr}((B^TA)^T) = \text{tr}(A^TB) = \langle A, B \rangle$
        \item $\text{tr}(AB^T) = \text{tr}(B^TA) = \langle A, B \rangle$
        \item $\text{tr}(BA^T) = \text{tr}(A^TB) = \langle A, B \rangle$
    \end{itemize}

    (iii) The $(i, j)$-th element of $A^TB$ equals the following (by the definition of matrix multiplication and the transpose):
    
    $$(A^TB)_{i, j} = \sum_{k} A^T_{i, k} B_{k, j} = \sum_{k} A_{k, i} B_{k, j}$$

    Each term of the last sum is nothing but the standard Euclidean inner product of $a_i, b_j$. Therefore, $(A^TB)_{i, j} = \langle a_i, b_j \rangle$.

    We now also have that $\langle A, B \rangle = \text{tr}(A^TB) = \sum_{1 \leq j \leq n} (A^TB)_{j, j} = \sum_{1 \leq j \leq n} \langle a_j, b_j \rangle$.

    (iv) To show that $\langle \cdot, \cdot \rangle$ defines an inner product for square matrices, we have that (by using results previously proved in the exercise):
    \begin{itemize}
        \item \textbf{Additivity in the first slot}: We have that $\langle A_1 + A_2, B \rangle = \text{tr}((A_1+A_2)^TB) = \text{tr}((A_1^T+A_2^T)B) = \text{tr}(A_1^TB + A_2^TB) = \text{tr}(A_1^TB) + \text{tr}(A_2^TB) = \langle A_1, B \rangle + \langle A_2, B \rangle$.
        \item \textbf{Homogeneity in the first slot}: We have that $\langle \lambda A, B \rangle = \text{tr}((\lambda A)^TB) = \text{tr}(\lambda A^TB) = \text{tr}(\lambda(A^TB)) = \lambda \text{tr}(A^TB)$.
        \item \textbf{Symmetry:} Note that in $\mathbb{R}^n$ we care about symmetry instead of conjugate symmetry, thus: $\langle B, A \rangle = \text{tr}(B^TA) = \langle A, B \rangle$ by part (ii).

        \item \textbf{Positivity}: We have that $\langle A, A \rangle =  \sum_{1 \leq j \leq n} \langle a_j, a_j \rangle \geq 0$, by the positivity of the Euclidean inner product for vectors.

        \item  \textbf{Definiteness}: By the immediately preceding bullet, we have that $\langle A, A \rangle$ is a sum of non-negative quantities, each of which is zero iff the $j$-th column of $A$ is zero (definiteness of the Euclidean inner product). This leads us to the conclusion that $\langle A, A \rangle$ is zero iff all of its columns are zero, which means that $A$ is the zero matrix, which is the zero element for matrices, thus proving definiteness of $\langle \cdot, \cdot \rangle$.
    \end{itemize}

    Therefore, $\langle \cdot, \cdot \rangle$ does indeed define an inner product for square matrices. The associated norm is $\lvert \lvert A \rvert \rvert = \sqrt{\langle A, A \rangle} = \sqrt{\sum_{1 \leq j \leq n} \langle a_j, a_j \rangle} = \sqrt{\sum_{i, j} A_{i, j} A_{i, j}}$ which indeed equals the Euclidean norm for the matrix $A$.\bigskip
    
    (v) Lemma 2.1.4 states that the vector space $\text{End}(\mathbb{R}^n)$ of operators in $\mathbb{R}^n$ can be written as a direct sum of the subspace $\text{End}^+(\mathbb{R}^n)$ of self-adjoint operators and the subspace $\text{End}^-(\mathbb{R}^n$ of anti-adjoint operators (that is, operators for which $A^* = -A$). In particular:
    $$\text{End}(\mathbb{R}^n) = \text{End}^+\mathbb{R}^n \oplus \text{End}^-(\mathbb{R}^n)$$

    , since any operator $T$ can be written as $T = \frac{1}{2}(T+T^*) + \frac{1}{2}(T-T^*)$.

    Consider then $A \in \text{End}^+(\mathbb{R}^n), B \in \text{End}^-(\mathbb{R}^n)$. Using the defining properties of these operators and part (ii), we obtain that:

    $$\langle A, B \rangle = \text{tr}(B^*A) = \text{tr}(-BA) = -\text{tr}(BA) = -\text{tr}(AB) = -\text{tr}(A^*B) = -\langle A, B \rangle$$

    Clearly, this implies that $\langle A, B \rangle$. Now let $x, y$ be the dimensions of $\text{End}^+(\mathbb{R}^n), \text{End}^-(\mathbb{R}^n)$ respectively. Due to the direct sum and the fact that $\text{dim End}(\mathbb{R}^n) = n^2$, we have that $x+y = n^2$. Recall also the isomorphism between operators and square matrices,and the fact that in $\mathbb{R}^n$ self-adjoint operators are precisely those whose matrix wrt. the standard basis equals its transpose. Such a matrix is fully defined by choosing the elements on and above the diagonal. In total, these are $n + \frac{n^2 - n}{2} = \frac{n(n+1)}{2}$. 
    
    In particular, such a matrix can be written uniquely as a sum of a linearly independent list of matrices consisting of either a one on the diagonal and zeros everywhere else, or a one in positions $(i, j), (j, i), i \neq j$, and zeros everywhere else. These are again in total $\frac{n(n+1)}{2}$. We conclude that the dimension $x$ equals $\frac{n(n+1)}{2}$, and thus that $y = n^2 - \frac{n(n+1)}{2} = \frac{n(n-1)}{2}$

    (vi) Consider first a self-adjoint $A$. Then, we have that:
    $$B(A, A) = \text{tr}(AA) = \text{tr}(A^*A) = \langle A, A \rangle$$

    This is clearly greater than zero if $A$ is not zero, by the properties of the inner product. For an anti-adjoint $A$ we have that:
    $$B(A, A) = \text{tr}(AA) = \text{tr}((-A^*)A) = -\text{tr}(A^*A) = \langle A, A \rangle$$

    By the exact same reasoning as before, for a non-zero $A$ this is clearly less than zero.

    (vii) Since $\tau$ is a linear mapping, it is uniquely determined by its values on a basis of $\text{Mat}(n, \mathbb{R})$. Observe that the $E_{ij}$ given in the hint form such a basis. Consider the value of $\tau$ on the commutator $[A, B] = AB - BA$. By its linearity and its given defining property, it must hold that:
    $$\tau([A, B]) = \tau(AB-BA) = \tau(AB) - \tau(BA) = \tau(AB) - \tau(AB) = 0$$

    For $i \neq j$, it holds that $[E_{ij}, E_{jj}] = E_{ij}E_{jj} - E_{jj}E_{ij} = E_{ij} - 0 = E_{ij}$ (since $E_{ij}$ maps everything to zero except for mapping $e_i$ to $e_j$ and $E_{jj}$ maps everything to zero except for mapping $e_j$ to $e_j$. By the above observation regarding commutators, we conclude that:
    $$\tau(E_{ij}) = 0$$

    For $i = j$, it holds that $[E_{ij}, E_{ji}] = E_{ii} - E_{jj}$. Again, we conclude that:
    $$\tau(E_{ii} - E_{jj}) = 0 \implies \tau(E_{ii}) = \tau(E_{jj})$$

    This means that $\tau$ maps every vector $E_{ij}, i \neq j$ to zero, and maps \textit{all} vectors $E_{ii}$ to the same value, let it be called $a$. By its linearity, observe that:
    $$\tau(I) = \tau(E_{11}+E_{22} + \ldots + E_{nn}) = na \implies a = \frac{1}{n}\tau(I)$$

    Finally, observe that the trace is also a linear function, maps $E_{ij}, i \neq j$ to zero and maps each $E_{ii}$ to 1 (since the trace equals the sum of the diagonal). But then it becomes clear that for all vectors $E_{ij}$ it holds that 
    $$\tau(E_{ij}) = a \text{tr}(E_{ij}) = \frac{1}{n}\tau(I)\text{tr}(E_{ij})$$

    , and since these are linear mappings, this also holds for any matrix, i.e.\ $\tau = \frac{1}{n}\tau(I)\text{tr}$.
\end{solution}
\newpage
\begin{exercise}{4}
    Suppose $A \in \text{End}(\mathbb{R}^n)$ is an orthogonal transformation, that is, $\lvert \lvert Ax \rvert \rvert = \lvert \lvert x \rvert \rvert$, for all $x \in \mathbb{R}^n$.

    (i) Show that $A \in \text{Aut}(\mathbb{R}^n)$ and that $A^{-1}$ is orthogonal.

    (ii) Deduce from the polarization identity in Lemma 1.1.5.(iii) that
    $$\langle A^*Ax, y \rangle = \langle Ax, Ay \rangle = \langle x, y \rangle, x, y \in \mathbb{R}^n$$

    (iii) Prove $A^*A = I$ and deduce that $\text{det} A = \pm 1$. Furthermore, using (i) show that $A^*$ is orthogonal, thus $AA* = I$; and also obtain $\langle Ae_i, Ae_j \rangle = \langle A^*e_i, A^*e_j \rangle = \langle e_i, e_j \rangle = \delta_{i, j}$, where $e_1, \ldots, e_n$ is the standard basis for $\mathbb{R}^n$. Conclude that the column and row vectors, respectively, in the corresponding matrix of $A$ form an orthonormal basis for $\mathbb{R}^n$.

    (iv) Deduce from (iii) that the corresponding matrix $A = (a_{ij}) \in \textbf{GL}(n, \mathbb{R})$ is \textit{orthogonal} with coefficients in $\mathbb{R}$ and belongs to the \textit{orthogonal group} $\textbf{O}(n, \mathbb{R})$, which means that it satisfies $A^TA = I$; in addition, deduce
    $$\sum_{1 \leq k \leq n} a_{ki}a_{kj} = \sum_{1 \leq k \leq n}a_{ik}a_{jk} = \delta_{ij}, 1 \leq i, j \leq n$$
\end{exercise}

\begin{solution}

    (i) We need to show that $A$ is a bijection. Since $A$ is an operator, it suffices to show that it is injective. Consider then an $x$ such that $Ax = 0$. We then have that $\lvert \lvert Ax \rvert \rvert = 0$, which, since $A$ is an isometry, implies $\lvert \lvert x \rvert \rvert = 0$, which means $x = 0$, therefore $A$ is injective and thus bijective. Therefore, $A^{-1}$ is well defined in $\mathbb{R}^n$.

    Now consider any $y \in \mathbb{R}^n$. It holds that $A^{-1}y = x$ such that $Ax = y$. Since $A$ is an isometry, we have that:
    $$\lvert \lvert Ax \rvert \rvert = \lvert \lvert x \rvert \rvert \implies \lvert \lvert y \rvert \rvert = \lvert \lvert A^{-1}y \rvert \rvert$$

    This means that $A^{-1}$ is also an isometry (in the terminology of this book, an orthogonal transformation).

    (ii) From the polarization identity we obtain that:
    $$\langle Ax, Ay \rangle = \frac{1}{4}(\lvert \lvert Ax + Ay \rvert \rvert^2 - \lvert \lvert Ax - Ay \rvert \rvert^2) = \frac{1}{4}(\lvert \lvert A(x+y) \rvert \rvert^2 - \lvert \lvert A(x-y) \rvert \rvert^2)$$
    $$\implies \langle Ax, Ay \rangle = \frac{1}{4}(\lvert \lvert x + y \rvert \rvert^2 - \lvert \lvert x - y \rvert \rvert^2) = \langle x, y \rangle$$

    By the definition of the adjoint, we also have that:
    $$\langle A^*Ax, y \rangle = \langle Ax, (A^*)^*y \rangle = \langle Ax, Ay \rangle$$

    (iii) Let $x$ be any element of $\mathbb{R}^n$. Using the properties proved in (ii), we have that:
    $$\langle A^*Ax - x, A^*Ax - x \rangle = \langle A^*Ax, A^*Ax \rangle + \langle x, x \rangle - 2\langle A^*Ax, x \rangle = $$
    $$\langle x, A^*Ax \rangle + \langle x, x \rangle - 2\langle x, x \rangle = \langle x, x \rangle + \langle x, x \rangle - 2 \langle x, x \rangle = 0$$

    By the properties of the inner product, the only way this can be true is if $A^*Ax - x = 0 \implies A^*Ax = x$. Since $x$ was selected arbitrarily, this means that $A^*A = I$. Because in $\mathbb{R}$ $A$ and $A^*$ have the same eigenvalues with the same multiplicities, and thus equal determinants, this also implies that:
    $$\text{det}(A^*A) = \text{det}(I) \implies \text{det}(A^*)\text{det}(A) = 1 \implies \text{det}(A)^2 = 1 \implies \text{det}(A) = \pm 1$$
    
    Since $A^*A = I$, $A^*$ is a left inverse of $A$, and from Linear Algebra we know that this means that $A^*$ is the inverse of $A$. Therefore, $A^* = A^{-1}$, and from (i) we know that $A^{-1}$ is an isometry/orthogonal.

    For any two $e_i, e_j$ of the standard basis of $\mathbb{R}^n$, we have that:
    $$\langle Ae_i, Ae_j \rangle = \langle e_i, e_j \rangle = \delta_{ij}$$
    , since the standard basis is orthonormal (and thus $\langle e_i, e_j \rangle = 0, i \neq j, \langle e_i, e_j \rangle = 1, i = j$). Also:
    $$\langle A^*e_i, A^*e_j \rangle = \langle e_i, (A^*)^*A^*e_j \rangle = \langle e_i, AA^*e_j \rangle = \langle e_i, e_j \rangle = \delta_{ij}$$

    From the first of those equations, we deduce that the columns of $A$ are orthonormal vectors. Furthermore, since they are $n$ in total, they must form a basis for $\mathbb{R}^n$. With respect to the standard basis, we know that the matrix of $A^*$ equals the transpose of the matrix of $A$. From the second of those equations, we then have that the columns of this matrix are also orthonormal and $n$ in total. Thus, the columns of the transpose of $\mathcal{M}(A)$, i.e.\ the rows of $\mathcal{M}(A)$, also form an orthonormal basis for $\mathbb{R}^n$.

    (iv) This is immediately obvious from the definitions of the general linear group and orthogonal group of matrices. The given sum $\sum_{1 \leq k \leq n} a_{ki}a_{kj}$ is nothing but the Euclidean inner product of the $i$-th and $j$-th columns of $\mathcal{M}(A)$, which from (iii) equals $\delta_{ij}$. Similarly, the given sum $\sum_{1 \leq k \leq n} a_{ik}a_{jk}$ is nothing but the Euclidean inner product of the $i$-th and $j$-th rows of $\mathcal{M}(A)$, which again from (iii) equals $\delta_{ij}$.
\end{solution}

\begin{exercise}{5}
    Let $\textbf{SO}(3, \mathbb{R}$ be the special orthogonal group in $\mathbb{R}^3$, consisting of the orthogonal matrices in $\text{Mat}(3, \mathbb{R})$ with determinant equal to 1. One then has, for every $R \in \textbf{SO}(3, \mathbb{R})$, see Exercise 2.4,
    $$R \in \textbf{GL}(3, \mathbb{R}), R^TR = I, \text{det} R = 1$$
    Prove that for every $R \in \textbf{SO}(3, \mathbb{R})$ there exist $a \in \mathbb{R}, 0 \leq a \leq \pi$ and $v \in \mathbb{R}^3$ with $\lvert \lvert v \rvert \rvert = 1$ with the following properties: $R$ fixes $v$ and maps the linear subspace $N_v$ orthogonal to $v$ into itself; in $N_v$ the action of $R$ is that of rotation by the angle $a$ such that, for $0 < a < \pi$ and for all $y \in N_v$, one has $\text{det}(v \ y \ Ry) > 0$.

    We write $R=R_{a, v}$, which is referred to as the counterclockwise rotation in $\mathbb{R}^3$ by the angle $a$ around the axis of rotation $\mathbb{R}v$.
\end{exercise}

\begin{solution}

    We begin by observing that since $R$ is an orthogonal matrix, the corresponding linear transformation $T$ must be an isometry. Indeed, for any $v \in \mathbb{R}^3$ we have that:
    $$\lvert \lvert T v \rvert \rvert^2 = \langle Tv, Tv \rangle = \langle v, T*Tv \rangle = \langle v, R^TRv \rangle = \langle v, Iv \rangle = \lvert \lvert v \rvert \rvert^2$$
    
    Since $\mathbb{R}^3$ has odd dimension, $T$ must have a real eigenvalue $\lambda$. Furthermore, because $\lvert \lvert T v \rvert \rvert = \lvert \lvert v \rvert \rvert$, $\lambda$ can only be 1 or -1. We claim that it cannot be the case that the only real eigenvalue is -1, because then the determinant of $T$ could not be 1. We prove this by contradiction. Recall that the determinant of $T$ equals the product of the eigenvalues of $T_c$ each raised to the corresponding multiplicity. Besides $\lambda$, $T_c$ can only have at most two more eigenvalues, $\lambda_1, \lambda_2$. If this is the case, because $T_c$ is the complexification of $T$, it must hold that $\lambda_2 = \overline{\lambda_1}$. But then if $\lambda = -1, \lambda \cdot \lambda_1 \cdot \lambda_2 = -1\cdot \lvert \lambda_1 \rvert^2$, and this cannot be equal to 1. The only other possibility is for $\lambda_1, \lambda_2$ to both be real. But then they would have to be either 1 or -1. If $\lambda = \lambda_1 = \lambda_2 = -1$ their product is -1, contradiction.
    
    Therefore, 1 is an eigenvalue of $T$ and there must exist a non-zero corresponding eigenvector $v$ with norm 1. Since $T$ is an isometry, it is normal. Again from Linear Algebra, we know then that the orthogonal complement of $v$, $N_v$ is invariant under $T$. Note also that $N_v$ must have dimension 2, since $\mathbb{R}^3 = \text{span}\{v\} \oplus N_v$. Since $T$ is an isometry, it must be the case that $T|_{N_v}$ is also an isometry. In a two-dimensional real vector space, we know that we can always find an orthonormal basis with respect to which the matrix of the isometry is either of the form:
    $$\begin{pmatrix}
        \text{cos} a & -\text{sin} a \\
        \text{sin} a & \text{cos} a
    \end{pmatrix}$$

    , with $a \in [0, \pi]$, or of the form $\begin{pmatrix}
        1 & 0 \\ 0 & -1
    \end{pmatrix}$ or $\begin{pmatrix}
        -1 & 0 \\ 0 & 1
    \end{pmatrix}$. In our case, these last two can be ruled out because they would lead to $T$ having a determinant of -1. Therefore, since $N_v$ has dimension 2 we can choose an orthonormal basis $v_1, v_2$ for $N_v$ with respect to which $\mathcal{M}(T)$ has the first of the three forms described above. Putting this together with the facts that $\mathbb{R}^3 = \text{span}\{v\} \oplus N_v$, $Tv = v$ and $N_v$ is the orthogonal complement of $\text{span}\{v\}$ we obtain that the matrix of $T$ with respect to the orthonormal basis $v, v_1, v_2$ must be of the form:
    $$\begin{pmatrix}
        1 & 0 & 0 \\
        0 & \text{cos} a & -\text{sin} a \\
        0 & \text{sin} a & \text{cos} a
    \end{pmatrix}$$
    It becomes clear that for any vector in $N_v$, $T$ rotates the vector by the angle $a$, and that this is counterclockwise, by the geometric definition of measuring angles between vectors. Also, from the definition of invariant subspaces, $T$ maps $N_v$ into itself. Therefore, we only need to show that $\text{det}(v \ y \ Ry)$ for all $y \in N_v$. We have that (where everything is with respect to our basis $v, v_1, v_2$):
    
    $$(v \ y \ Ry) = \begin{pmatrix}
        1 & y_1 & y_1 \\ 0 & y_2 & \text{cos} a y_2 - \text{sin} a y_3 \\ 0 & y_3 & \text{sin} a y_2 + \text{cos} a y_3
    \end{pmatrix}$$

    Observe that for this matrix, the terms of the determinant sum that take the first column element from the second or third row are zero. Therefore, the determinant is:
    $$\text{det}(v \ y \ Ry) = \text{sign}(1, 2, 3)1\cdot y_2 \cdot (\text{sin}a y_2 + \text{cos} a y_3) + \text{sign}(1, 3, 2)1 \cdot y_3 \cdot (\text{cos}a y_2 - \text{sin} a y_3)$$
    $$= {y_2}^2 \text{sin}a + y_2 y_3 \text{cos}a - y_2 y_3 \text{cos} a + {y_3}^2 \text{sin} a = \text{sin}a({y_2}^2 + {y_3}^2)$$

    Because $a \in [0, \pi]$, this determinant is always non-negative, and is zero whenever $y_2, y_3 = 0$, which means that $y$ is a multiple of $v$. Since $y \in N_v$ this can only happen if $y = 0$ (which also makes intuitive sense, because then $Ry = 0$).
\end{solution}

\begin{exercise}{6}
    Assume a differentiable function $f: \mathbb{R} \rightarrow \mathbb{R}$ has a zero, so $f(x) = 0$ for some $x \in \mathbb{R}$.
    Suppose $x_0 \in \mathbb{R}$ is a first approximation to this zero and consider the tangent line to the graph $\{(x, f(x)) \lvert x \in \text{dom}(F)\}$ of $f$ at $(x_0, f(x_0)$; this is the set
    $$\{(x, y) \in \mathbb{R}^2 \lvert y -f(x_0) = f'(x_0)(x - x_0)\}$$
    Then determine the intercept $x_1$ of that line with the $x$-axis, in other words $x_1 \in \mathbb{R}$ for which $-f(x_0) = f'(x_0)(x_1 - x_0)$, or, under the assumption that $A^{-1} = f'(x_0) \neq 0$,
    
    $$x_1 = F(x_0) \text{ where } F(x) = x - Af(x)$$
    
    It seems plausible that $x_1$ is nearer the required zero $x$ than $x_0$, and that iteration of this procedure will get us nearer still.
    In other words, we hope that the sequence $(x_k)_{k \in \mathbb{N}_0}$ with $x_{k+1} = F(x_k)$ converges to the required zero $x$, for which then $F(x) = x$.
    Now we formalize this heuristic argument.

    Let $x_0 \in \mathbb{R}^n, \delta > 0$, and let $V = V(x_0; \delta)$ be the closed ball in $\mathbb{R}^n$ of center $x_0$ and radius $\delta$; furthermore, let $f: V \rightarrow \mathbb{R}^n$.
    Assume $A \in \text{Aut}(\mathbb{R}^n)$ and a number $\epsilon$ with $0 \leq \epsilon < 1$ to exist such that:

    (i) the mapping $F: V \rightarrow \mathbb{R}^n$ with $F(x) = x - A(f(x))$ is a contraction with contraction factor $\leq \epsilon$;

    (ii) $\lvert \lvert A f(x_0) \rvert \rvert \leq (1 - \epsilon) \delta$

    Prove that there exists a unique $x \in V$ with
    $$f(x) = 0; \text{ and also } \lvert \lvert x - x_0 \rvert \rvert \leq 
    \frac{1}{1 - \epsilon}\lvert \lvert A f(x_0) \rvert \rvert$$
\end{exercise}

\begin{solution}

    Since $V$ is closed and $F$ is a contraction, we would like to use the contraction lemma. 
    However, this requires $F$ to map $V$ to itself.
    For this we should therefore show that for any $x \in V$ it also holds that $F(x) \in V$,
    which is equivalent to showing that $\lvert \lvert F(x) - x_0 \rvert \rvert \leq \delta$.
    We know that $F$ is a contraction, which means that:
    $$\lvert \lvert F(x) - F(x_0) \rvert \rvert \leq \epsilon \lvert \lvert x - x_0 \rvert \rvert \implies \lvert \lvert F(x) - x_0 + A(f(x_0)) \rvert \rvert \leq \epsilon \lvert \lvert x - x_0 \rvert \rvert$$
    $$\implies \Bigl \lvert \lvert \lvert F(x) - x_0 \rvert \rvert - \lvert \lvert A(f(x_0)) \rvert \rvert \Bigr \rvert \leq \epsilon \lvert \lvert x - x_0 \rvert \rvert \implies \lvert \lvert F(x) - x_0 \rvert \rvert - \lvert \lvert A(f(x_0)) \rvert \rvert \leq \epsilon \lvert \lvert x - x_0 \rvert \rvert$$
    $$\implies \lvert \lvert F(x) - x_0 \rvert \rvert \leq \epsilon\delta + (1 - \epsilon)\delta \implies \lvert \lvert F(x) - x_0 \rvert \rvert \leq \delta,$$
    which is precisely the condition we require for the contraction lemma.
    By applying it we have that there exists a unique $x \in V$ such that:
    $$F(x) = x \implies x - A(f(x)) = x \implies A(f(x)) = 0$$
    Because $A$ is invertible, the only way this can hold is if $f(x) = 0$.
    Furthermore, again from the contraction lemma we know that:
    $$\lvert \lvert x - x_0 \rvert \rvert \leq \frac{1}{1 - \epsilon}\lvert \lvert F(x_0) - x_0 \rvert \rvert \implies \lvert \lvert x - x_0 \rvert \rvert \leq \frac{1}{1 - \epsilon} \lvert \lvert A f(x_0) \rvert \rvert$$
\end{solution}

\begin{exercise}{8}
    Let $f: \mathbb{R}^n \rightarrow \mathbb{R}$ be differentiable.

    (i) Prove that $f$ is constant if $D f = 0$.

    \textbf{Hint}: Recall that the result is known if $n = 1$ and use directional derivatives.

    Let $f: \mathbb{R}^n \rightarrow \mathbb{R}^p$ be differentiable.

    (ii) Prove that $f$ is constant if $D_f = 0$.

    (iii) Let $L \in L(\mathbb{R}^n, \mathbb{R}^p)$, and suppose $Df(x) = L$, for every $x \in \mathbb{R}^n$. Prove the existence of $c \in \mathbb{R}^p$ with $f(x) = Lx + c$, for every $x \in \mathbb{R}^n$.
\end{exercise}

\begin{solution}

    (i) Suppose $f$ is not constant. Then there exist $x, y, x \neq y$ such that $f(x) \neq f(y)$. This means also that $y - x \neq 0$. Then let $g: \mathbb{R} \rightarrow \mathbb{R}$ be $g(t) = f(t(y - x) + x)$, in which case $g(0) = f(x), g(1) = f(y)$. Because $f$ is differentiable, so is $g$ (composition of differentiable functions). Then, apply the Mean Value Theorem in the interval $[0, 1]$ to obtain that there exists $c \in (0, 1)$ such that:
    $$g'(c) = \frac{g(1) -g(0)}{1 - 0} = f(y) - f(x) \neq 0$$

    But then observe that:
    $$g'(c) = \lim_{h \rightarrow 0} \frac{g(c + h) - g(c)}{h} = \lim_{h \rightarrow 0} \frac{f((c+h)(y - x) +x) - f(c(y - x) + x)}{h}$$
    $$ = \lim_{h \rightarrow 0}  \frac{f(h(y -x) + c(y - x) + x) - f(c(y - x) + x)}{h}$$

    , which equals precisely the directional derivative of $f$ along the vector $y - x$ at $c(y - x) + x$. Because this equals $[D f(c(y -x) +x)](y - x)$, and $D f = 0$, it must be zero, which is a contradiction.

    (ii) Again, suppose $f$ is not constant. Then there exist $x = (x_1, \ldots, x_n), y = (y_1, \ldots y_n), x \neq y$ such that $f(x) = (z_1, \ldots, z_p) \neq (w_1, \ldots, w_p) = f(y)$. As a consequence, $z_j \neq w_j$ for at least one $j$. Then consider the function $g: \mathbb{R}^n \rightarrow \mathbb{R}, g(x_1, \ldots, x_n) = f_j (x_1, \ldots, x_n)$. It is the case that:
    $$[D g] (x)= [D f] (x)_{j, .}$$

    , that is, the derivative of $g$ at any $x$ equals the $j$-th row of the derivative of $f$ at $x$. Therefore, $D_g$ is also zero (a row of zeros). But then by part (i), $g$ is constant, and thus $g(z) = f_j(z)$ must be equal to $g(w) = f_j(w)$, a contradiction. Therefore $f$ is constant.

    (iii) Consider the function $g: \mathbb{R}^n \rightarrow \mathbb{R}^p, g(x) = f(x) - Lx$. At any given $x$, we know that the derivative of $Lx$ is $L$ (clearly, the ``best linear approximation'' of a linear approximation is itself). Furthermore, we are given that $D f = L$. By the addition rule for derivatives, we have firstly that $g$ is differentiable and secondly that:
    $$[D g](x) = [D f](x) - L = L - L = 0$$

    But then by part (ii), $g$ is constant, therefore there exists $c \in \mathbb{R}^p$ such that $g(x) = c$ for all $x \in \mathbb{R}^n$. This means of course that $f(x) - Lx = c \implies f(x) = Lx + c$.
\end{solution}

\begin{exercise}{9}
    Let $f: \mathbb{R}^n \rightarrow \mathbb{R}$ be homogeneous of degree 1, in the sense that $f(tx) = tf(x)$ for all $x \in \mathbb{R}^n$ and $t \in \mathbb{R}$. Show that $f$ has directional derivatives at 0 in all directions. Prove that $f$ is differentiable at 0 if and only if $f$ is linear, which is the case if and only if $f$ is additive.
\end{exercise}

\begin{solution}

    Observe first that $f(\mathbf{0}) = f(0 \cdot \mathbf{0}) = 0f(\mathbf{0}) = 0$. Now, pick any $v \in \mathbb{R}^n$ and examine the definition of the directional derivative of $f$ along $v$ at 0:
    $$\lim_{h \rightarrow 0} \frac{f(0 + hv) - f(0)}{h} = \lim_{h\rightarrow 0}\frac{f(hv) - 0}{h} = \lim_{h \rightarrow 0} \frac{hf(v)}{h} = f(v)$$

    , where we used the fact that $f$ is homogeneous. 
    We observe then that this limit always evaluates to some number, namely, to the corresponding $f(v)$. 
    This means that all directional derivatives of $f$ at 0 exist. 
    Now we examine the second claim of the exercise. 
    We observe first that if $f$ is linear, it is clearly additive, whereas if it is additive, since we are given that it is homogeneous, it is then also linear. 
    This means that it suffices to prove that ``$f$ is differentiable at 0 if and only if it is additive''
    We have that:

    $\implies$: Suppose first that $f$ is differentiable at 0.
    Then $[D f(0]$ is a unique linear transformation, with the property that the directional derivative of $f$ along any $v$ at 0 equals $[D f(0)](v)$.
    Pick any two $a, b \in \mathbb{R}^n$.
    We then have that:
    $$[D f(0)](a+b) = f(a+b)$$
    , since the directional derivative of $f$ at 0 along any $v$ equals $f(v)$. 
    Additionally, $[D f(0)]$ is linear, and therefore:
    $$[D f(0)](a+b) = [D f(0)](a) + [D f(0)](b) = f(a) + f(b)$$
    , again by using the same property for the directional derivatives of $f$ at 0.
    But then we have shown that $f(a+b) = f(a) + f(b)$, which means precisely that $f$ is additive.

    $\impliedby$: Now suppose $f$ is additive. 
    We know from the first part of the exercise that all directional derivatives of $f$ at 0 exist. 
    If $e_1, \ldots, e_n$ are the vectors of the standard basis of $\mathbb{R}^n$, then we know that the corresponding directional derivatives of $f$ at 0 are equal to $f(e_1), \ldots, f(e_n)$.
    Let then $L$ be the linear transformation such that $L(e_i) = f(e_i)$ (linear transformations are completely defined by their values on a basis).
    Consider any vector $h = \sum_{i=1}^{n} a_i e_i$ which tends to zero.
    Then:
    $$\lim_{h \rightarrow 0} \frac{1}{\lvert \lvert h \rvert \rvert}( f(0+ h) - f(0) - L(h) ) = \lim_{h \rightarrow 0} \frac{1}{\lvert \lvert h \rvert \rvert} ( f(\sum_{i=1}^{n} a_i e_i) - f(0) - L(\sum_{i=1}^{n} a_i e_i) )$$
    $$= \lim_{h \rightarrow 0} \frac{1}{\lvert \lvert h \rvert \rvert}( \sum_{i=1}^{n} a_i f(e_i) - 0 - \sum_{i=1}^{n} a_i L(e_i) ) = 0$$

    , where we used the linearity of both $f$ and $L$.
    This limit being equal to 0 means precisely that $f$ is differentiable at 0.
    
\end{solution}

\newpage

\begin{exercise}{10}
    Let $g$ be as in Example 1.3.11, that is:
    $$g(x) = \begin{cases}
        \frac{x_1 x_2^2}{x_1^2 + x_2^4},\ x \neq 0 \\
        0,\ x = 0
    \end{cases}$$ 
    Then we define $f: \mathbb{R}^2 \rightarrow \mathbb{R}$ by
    $$f(x) = x_2 g(x) = \begin{cases}
        \frac{x_1 x_2^3}{x_1^2 + x_2^4},\ x \neq 0 \\
        0,\ x = 0
    \end{cases}$$

    Show that $D_v f(0) = 0$ for all $v \in \mathbb{R}^2$; and deduce that $D f(0) = 0$, if $f$ were differentiable at 0. Prove that $D_v f(x)$ is well-defined for all $v, x \in \mathbb{R}^2$, and that $v \mapsto D_v f(x)$ belongs to $\mathcal{L}(\mathbb{R}^2, \mathbb{R})$, for all $x \in \mathbb{R}^2$.
    Nevertheless, verify that $f$ is not differentiable at 0 by showing
    $$\lim_{x_2 \rightarrow 0} \frac{\lvert f(x_2^2, x_2) \rvert}{\lvert \lvert (x_2^2, x_2) \rvert \rvert} = \frac{1}{2}$$
\end{exercise}

\begin{solution}

    Pick any $v \in \mathbb{R}^2, v \neq 0$ and examine the definition of the directional derivative of $f$ along $v$ at 0:
    $$\lim_{h \rightarrow 0} \frac{f(0 + hv) - f(0)}{h} = \lim_{h \rightarrow 0} \frac{\frac{(hv_1)(hv_2)^3}{(hv_1)^2 + (hv_2)^4} - 0}{h} = \lim_{h \rightarrow 0} \frac{h^4 v_1 v_2^3}{h^3v_1^2 + h^4 v_2^4} = \lim_{h \rightarrow 0} \frac{h v_1 v_2^3}{v_1^2 + hv_2^4}$$

    If $v_1 = 0$, it has to be the case that $v_2 \neq 0$, in which case the limit clearly evaluates to 0. 
    If $v_2  = 0$, it has to be the case that $v_1 \neq 0$, in which case again the limit evaluates to 0.
    If both $v_1, v_2 \neq 0$, then the denominator tends to $v_1^2$ and the numerator tends to 0, which again means that the limit evaluates to 0.
    Therefore, the limit can indeed be show to equal 0 for all $v \in \mathbb{R}^2 \setminus {0}$.
    Thus, \textit{if} $f$ were differentiable at 0, it would have to be the case that $D_f(0) = 0$, since it would have to equal a linear transformation which maps all vectors to zero.
    
    We have already shown that $D_v f(x)$ is zero whenever $x = 0$. For $x \neq 0$, observe that $f$ is defined as a rational function with a non-zero numerator. 
    This is then clearly a differentiable function, which means $D_f (x)$ exists, and then we know that the directional derivative along $v$ at $x$ is nothing but $[D_f(x)](v)$, so obviously the function $v \mapsto D_v f(x)$ is linear.

    However, we will now show that the zero matrix/function does not fulfill the definition of being the derivative of $f$ at 0, that is, we will show that:

    $$\lim_{h \rightarrow 0}\frac{1}{\lvert \lvert h \rvert \rvert}( (f(0 + h) - f(0)) - Lh ) \neq 0$$

    Consider approaching $0$ with a sequence of points $i \rightarrow (\frac{1}{i^2}, \frac{1}{i}), i > 0$. This corresponds to the limit:

    $$\lim_{x_2 \rightarrow 0} \frac{1}{\lvert \lvert (x_2^2, x_2) \rvert \rvert}(f(x_2^2, x_2) - 0 - 0h) = \lim_{x_2 \rightarrow 0}\frac{1}{\sqrt{x_2^4 + x_2^2}}(\frac{x_2^2x_2^3}{x_2^4 + x_2^4}) = \lim_{x_2 \rightarrow 0} \frac{x_2}{2x_2\sqrt{1 + x_2^2}} = \frac{1}{2}$$

    Since this is not zero, the limit as $h \rightarrow 0$ cannot be zero either, thus showing that $f$ is not differentiable at 0 (since we showed that if it were, it would have to hold that $D_f(0) = 0$).
    
\end{solution}

\newpage

\begin{exercise}{11}
    Define $f: \mathbb{R}^2 \rightarrow \mathbb{R}$ by
    $$f(x) = \begin{cases}
        \frac{x_1^3}{\lvert \lvert x \rvert \rvert^2}, & x \neq 0\\
        0, & x = 0
    \end{cases}$$
    Prove that $f$ is continuous on $\mathbb{R}^2$. 
    Show that directional derivatives of $f$ at 0 in all directions do exist. Verify, however, that $f$ is not differentiable at 0. 
    Compute
    $$D_1 f(x) = 1 + x_2^2\frac{x_1^2 - x_2^2}{\lvert \lvert x \rvert \rvert^4}, D_2 f(x) = -\frac{2x_1^3x_2}{\lvert \lvert x \rvert \rvert^4} \ (x \in \mathbb{R}^2 \setminus \{0\})$$
    In particular, both partial derivatives of $f$ are well-defined on all of $\mathbb{R}^2$. Deduce that at least one of these partial derivatives has to be discontinuous at 0. To see this explicitly, note that
    $$D_1 f(tx) = D_1 f(x), D_2 f(tx) = D_2 f(x) (t \in \mathbb{R} \setminus \{0\}, x \in \mathbb{R}^2)$$
    This means that the partial derivatives assume constant values along every line through the origin under omission of the origin. More precisely, in every neighborhood of 0 the function $D_1 f$ assumes every value in $[0, \frac{9}{8}]$ while $D_2 f$ assumes every value in $[-\frac{3}{8}\sqrt{3}, \frac{3}{8}\sqrt{3}]$. Accordingly, in this case both partial derivatives are discontinuous at 0.
\end{exercise}

\begin{solution}

    For $x \neq 0$, $f$ is continuous as a fraction of two continuous functions. In order for $f$ to be continuous at 0 it has to be the case that $\lim_{x \rightarrow 0} f(x) = f(0) = 0$. We have that:
    $$\Biggl \lvert \frac{x_1^3}{\lvert \lvert x \rvert \rvert^2} \Biggr \rvert = \Biggl \lvert \frac{x_1^3}{x_1^2 + x_2^2} \Biggr \rvert = \Biggl \lvert \frac{x_1}{1 + (\frac{x_2}{x_1})^2} \Biggr \rvert \leq \lvert x_1 \rvert$$

    , since we are dividing $\lvert x_1 \rvert$ with a quantity that is always at least 1. Now, as $x \rightarrow 0$, $\lvert x_1 \rvert \rightarrow 0$, which means that the LHS is upper-bounded by a quantity that tends to zero, which means that it must itself tend to zero. This means precisely that $\lim_{x \rightarrow 0} f(x) = f(0) = 0$. Therefore $f$ is continuous on $\mathbb{R}^2$.

    Now, for computing the directional derivative of $f$ at 0 along $v$ we set 
    $$g(h) = f(0 + hv) = f(hv) = \frac{h^3v_1^3}{h^2v_1^2 + h^2v_2^2} = \frac{h v_1^3}{v_1^2 + v_2^2}$$
    and we observe that $D_v f(0) = g'(0) = \frac{v_1^3}{v_1^2 + v_2^2}$. 
    If $f$ were differentiable at 0 it would have to be the case that $D f(0) = \begin{pmatrix}
        1 & 0
    \end{pmatrix}$, which is obtained by substituting $v = (1, 0), v = (0, 1)$ in the formula above. 
    It would also have to hold that:
    $$\lim_{h \rightarrow 0} \frac{1}{\lvert \lvert h \rvert \rvert}( f(0 + h) - f(0) - D f(0) h) = 0 \implies \lim_{h \rightarrow 0} \frac{1}{\lvert \lvert h \rvert \rvert}( \frac{h_1^3}{h_1^2 + h_2^2} - 0 - \begin{pmatrix}
        1 & 0
    \end{pmatrix} \begin{pmatrix}
        h_1 \\ h_2
    \end{pmatrix}) = 0$$
    $$\implies \lim_{h \rightarrow 0} \frac{1}{\lvert \lvert h \rvert \rvert}( \frac{h_1^3}{h_1^2 + h_2^2} - h_1) = 0 \implies \lim_{h \rightarrow 0} \frac{1}{\lvert \lvert h \rvert \rvert}( \frac{h_1^3 - h_1h_2^2}{h_1^2 + h_2^2}) = 0$$

    Consider what happens when $h$ approaches 0 with $h_2 = 0$. Then the quantity inside the limit evaluates to:
    $$\frac{h_1^3 - 0}{\lvert h_1 \rvert \cdot h_1^2} = \frac{h_1}{\lvert h_1 \rvert},$$
    which tends to 1 for $h \rightarrow 0^+$ and to -1 for $h \rightarrow 0^-$. This shows that the limit above does not exist, and therefore $f$ cannot be differentiable at 0.

    Now let us compute $D_1 f(x), D_2 f(x)$ for $x \neq 0$. By standard differentiation rules:
    $$D_1 f(x) = \frac{\partial}{\partial x_1}( \frac{x_1^3}{x_1^2 + x_2^2}) = \frac{3x_1^2(x_1^2 + x_2^2) - 2x_1^4}{(x_1^2 + x_2^2)^2} = \frac{x_1^4 + 3x_1^2 x_2^2}{\lvert \lvert x \rvert \rvert^4} = \frac{x_1^4 + \lvert \lvert x \rvert \rvert^4 - \lvert \lvert x \rvert \rvert^4 + 3x_1^2 x_2^2}{\lvert \lvert x \rvert \rvert^4}$$
    $$= 1 + \frac{x_1^4 - (x_1^2 + x_2^2)^2 + 3x_1^2 x_2^2}{\lvert \lvert x \rvert \rvert^4} = 1 + \frac{-2x_1^2 x_2^2 - x_2^4 + 3x_1^2 x_2^2}{\lvert \lvert x \rvert \rvert^4} = 1 + \frac{x_2^2(x_1^2 - x_2^2)}{\lvert \lvert x \rvert \rvert^4}$$
    One can similarly obtain that $D_2 f(x) = -\frac{2x_1^3x_2}{\lvert \lvert x \rvert \rvert^4}$ (calculations omitted for brevity).

    For both of these formulas one can notice that if we substitute $(x_1, x_2)$ by $(tx_1, tx_2)$, the powers are arranged in such a way that all $t$'s cancel out, the result being that $D_1 f(tx) = D_1 f(x), D_2 f(tx) = D_2 f(x)$.

    Now consider a neighborhood around 0 containing an open ball $B_\epsilon(0)$. Observe that if we select a point $v$ inside this ball such that $v = (\sqrt{3}x, x)$ we have that:
    $$D_1 f(v) = 1 + v^2 \frac{3v^2 - v^2}{(3v^2 + v^2)^2} = 1 + \frac{2v^4}{16v^4} = \frac{9}{8}$$
    But recall that $D_1 f(0) = 1$, which means that if $D_1 f$ were continuous at 0, its limit as $x \rightarrow 0$ would have to be 0 also. 
    However, the existence of a point $v$ with the above property for \textit{any} $\epsilon$ such that $D_1 f(v) = \frac{9}{8}$ implies that the limit cannot be zero. Therefore, $D_1 f$ is discontinuous at 0.

    Similarly, if $D_2 f$ were continuous at 0, its limit would have to equal $D_2 f(0) = 0$. 
    By the same argument as above, observe that for any $\epsilon > 0$ we can find $v \in B_\epsilon (0), v = (\sqrt{3}x, x)$, and then:
    $$D_2 f(v) = -2 \frac{(\sqrt{3}x)^3x}{(3x^2 + x^2)^2} = - \frac{2\sqrt{3}\cdot3x^4}{16x^4} = -\frac{3\sqrt{3}}{8}$$
    The existence of this point shows that the limit of $D_2 f$ as $x \rightarrow 0$ cannot be zero, and thus $D_2 f$ is not continuous at 0.
    
\end{solution}

\begin{exercise}{12}
    Theorem 2.3.4 states that:

    Let $U \subset \mathbb{R}^n$ be an open set, let $a \in U$ and $f: U \rightarrow \mathbb{R}^p$. Then $f$ is differentiable at $a$ if $f$ is partially differentiable in a neighborhood of $a$ and all its partial derivatives are continuous at $a$.

    
    Let the notation be as in the above theorem, but with $n \geq 2$. 
    Show that the conclusion of the theorem remains valid under the weaker assumption that $n - 1$ partial derivatives of $f$ exist in a neighborhood of $a$ and are continuous at $a$ while the remaining partial derivative merely exists at $a$.

    \textbf{Hint:} Write
    $$f(a + h) - f(a) = \sum_{n \geq j \geq 2} (f(a + h^{(j)}) - f(a + h^{(j - 1)}) + f(a + h^{(1)}) - f(a)$$

    Apply the method of the theorem to the sum over $j$ and the definition of derivative to the remaining difference.

    \textbf{Background}: As a consequence of this result we see that at least two different partial derivatives of $f$ must be discontinuous at $a$ if $f$ fails to be differentiable at $a$, compare with Example 2.3.5.
\end{exercise}

\begin{solution}

    WLOG we will assume that the partial derivative of $f$ with respect to the first variable merely exists at $a$, and all others exist in a neighborhood of $a$ and are continuous at $a$ (this is because in all other cases everything below is the same up to a ``permutation'' of variables).
    
    We begin by restating the definition of $h^{(j)}$ that was used in the proof of theorem 2.3.4:
    $$h^{(j)} = \sum_{1 \leq k \leq j} h_k e_k \in \mathbb{R}^n, n \geq j \geq 0$$
    We also recall that $h^{(j)} = h^{(j-1)} + h_j e_j$.
    Essentially, $h^{(j)}$ is a vector that lies in the span of the first $j$ vectors of the standard basis of $\mathbb{R}^n$.
    The letter $h$ serves to remind us of the fact that it will be used as an infinitesimal ``perturbation'' vector.
    Now, as given in the hint, observe that:
    $$f(a + h) - f(a) = \sum_{n \geq j \geq 2} (f(a + h^{(j)}) - f(a + h^{(j-1)}) + f(a + h^{(1)}) - f(a),$$
    where we see that almost all terms of the sum lead to cancellations of these ``partially perturbed'' $f(a + h^{(j)})$. 
    Now, as in theorem 2.3.4, we define $g_j: \mathbb{R} \rightarrow \mathbb{R}$ as $g_j(t) = f(a + h^{(j-1)} + te^j)$.
    We can thus rewrite:
    $$f(a + h) - f(a) = \sum_{n \geq j \geq 2} (g_j(h_j) - g_j(0)) + g_1(h_1) - g_1(0)$$

    For $j=2, \ldots, n$, the $j$-th partial derivative of $f$ exists in a neighborhood of $a$, which means that there exists an open interval in $\mathbb{R}$ containing 0 such that $g_j$ is differentiable on it.
    Consider a sufficiently small $h$ such that each $h_j$ is contained in the corresponding interval for $g_j$.
    As in the proof of 2.3.4, we can apply then the mean value theorem to each $g_j$ to obtain that there exist $\tau_j \in [0, h_j]$  such that:
    $$D_j f(a + h^{(j - 1)} + \tau_j e_j) = g_j'(\tau_j) = \frac{f(a + h^{(j-1)} + h_je^j) - f(a + h^{(j-1)})}{h_j}$$
    One can thus write that:
    $$f(a + h) - f(a) = \sum_{n \geq j \geq 2} [D_j f(a + h^{(j-1)} + \tau_j e_j)] h_j + (g_1(h_1) - g_1(0))$$
    We recall now that the remaining partial derivative at $a$ $(D_1 f(a))$ exists.
    Equivalently, $g_1$ is differentiable at 0, which, from Hadamard's lemma means that there exists $\phi_1 \in \mathbb{R} \rightarrow \mathcal{L}(\mathbb{R}, \mathbb{R})$ such that $g_1(t) - g_1(0) = \phi_1(t)t$.
    By a slight abuse of notation, $\phi_1(t)$ can be identified with a real number also denoted as $\phi_1(t)$.
    Then let $\phi: \mathbb{R}^n \rightarrow \mathcal{L}(\mathbb{R}^n, \mathbb{R})$ be:
    $$\phi(a + h) = \begin{pmatrix}
        \phi_1(h_1) & D_2 f(a + h^{(1)} + \tau_2 e_2) & \ldots & D_n f(a + h^{(n-1)} + \tau_n e_n)
    \end{pmatrix}$$
    By the equations above one can see that $f(a+h) - f(a) = \phi(a + h)(h)$, and because $D_2 f, \ldots D_n f$ are all continuous at the indicated points and $\phi_1$ is continuous at 0, $\phi$ is continuous at $a$, which is precisely condition (ii) of the Hadamard lemma, which we know is equivalent to $f$ being differentiable at $a$.
    
\end{solution}

\begin{exercise}{15}
    Let $f: \mathbb{R}^n \rightarrow \mathbb{R}$ be a $C^1$ function and $k > 0$, and assume that $\lvert D_jf(x) \rvert \leq k$ for $1 \leq j \leq n$ and $x \in \mathbb{R}^n$.

    (i) Prove that $f$ is Lipschitz continuous with $\sqrt{n}k$ as a Lipschitz constant.

    Next, let $g: \mathbb{R}^n \setminus \{0\} \rightarrow \mathbb{R}$ be a $C^1$ function and $k > 0$, and assume that $\lvert D_j g(x) \rvert \leq k$, for $1 \leq j \leq n$ and $x \in \mathbb{R}^n \setminus \{0\}$.

    (ii) Show that if $n \geq 2$, then $g$ can be extended to a continuous function defined on all of $\mathbb{R}^n$.
    Show that this is false if $n = 1$ by giving a counterexample.
\end{exercise}

\begin{solution}
    
    (i) Suppose $x = (x_1, x_2, \ldots, x_n), y = (y_1, y_2, \ldots, y_n), x, y \in \mathbb{R}^n$.
    We need to show that $\lvert f(y) - f(x) \rvert \leq \sqrt{n}k \lvert \lvert y - x \rvert \rvert$.
    Let $h_i = y_i - x_i$, in which case we have that:

    $$\lvert f(y_1, \ldots, y_n) - f(x_1, \ldots, x_n) \rvert = \lvert f(x_1 + h_1, \ldots, x_n + h_n) - f(x_1, \ldots, x_n) \rvert$$
    $$= \lvert f(x_1 + h_1, \ldots, x_n + h_n) - f(x_1, x_2 + h_2, \ldots, x_n + h_n) + f(x_1, x_2 + h_2, \ldots, x_n + h_n) - \ldots$$
    $$ + f(x_1, \ldots, x_n + h_n) - f(x_1, \ldots, x_n) \rvert $$

    By writing out like this, we observe that we now have a sum of differences, each of which only differs in one variable of $f$.
    Because $f$ is $C^1$, by the mean value theorem we can write:

    $$f(x_1, \ldots, x_{i-1}, x_i + h_i, \ldots, x_n + h_n) - f(x_1, \ldots, x_{i-1}, x_i, \ldots x_n + h_n) = $$
    $$h_i D_i (f)(x_1, \ldots, x_{i-1}, b_i, x_{i+1} + h_{i+1}, \ldots, x_n + h_n),$$

    for some $b_i \in (x_i, x_i + h_i)$.
    If we call $c_i$ the point at which $D_i(f)$ is evaluated, we can rewrite the difference we started with as:

    $$\lvert f(y) - f(x) \rvert = \lvert \sum_{i=1}^{n} h_i D_i (f)(c_i) \rvert \leq k \lvert \sum_{i=1}^{n} h_i \rvert \leq k \sqrt{n} \lvert \lvert h \rvert \rvert = k \sqrt{n} \lvert \lvert y - x \rvert \rvert,$$

    where we used the inequality given for $D_i(f)$ and the bound for the $l$-1 norm that we have seen in e.g. exercise 18, section 3.2 of Carothers.

    (ii) If we can show that for any two sequences $(x_n), (y_n) \rightarrow 0$ the corresponding $(g(x_n)), (g(y_n))$ converge to same real number $L$, we will have shown that a continuous extension of $g$ to all of $\mathbb{R}^n$ exists (by defining its value at 0 as $L$).
    We can restrict our attention to the unit ball $B_1(0)$.
    We would like to be able to use the mean value theorem, but we observe that because $0 \in B_1(0)$, it may not always be the case that for $a, b \in B_1(0) \setminus \{0\}$ the segment $[a, b]$ lies entirely within the domain of $g$.

    Motivated by this, consider partitioning $B_1(0) \setminus \{0\}$ into $2^n$ ``open quadrants'', for all possible combinations of the signs of the coordinates.
    For example,  $Q_1 = \{(x_1, \ldots, x_n) \in \mathbb{R}^n \lvert x_1 > 0, x_2 > 0\ldots x_n > 0\}, Q_2 = \{(x_1, \ldots, x_n) \in \mathbb{R}^n \lvert x_1 < 0, x_2 > 0\ldots x_n > 0\}$.
    Notice that these are disjoint, and that if we take the union of all $Q_i$ and of all the planes where at least coordinate equals zero we get $B_1(0) \setminus \{0\}$.
    Notice also that each $Q_i$ is open, and that since all partial derivatives of $g$ are bounded, the proof of (i) applies on each $Q_i$ (instead of on all of $\mathbb{R}^n$.

    Therefore, for any two sequences $x_k, y_k \rightarrow 0$ inside the same quadrant, we have that $\lvert g(y_k) - g(x_k) \rvert \leq k \sqrt{n} \lvert \lvert y_k - x_k \rvert \rvert$.
    Firstly, this shows that $g$'s values are bounded inside each quadrant (since the RHS is also bounded).
    Secondly, the convergence of the two sequences to zero shows that as $k \rightarrow \infty$, $\lvert f(x_k) - f(y_k) \rvert \rightarrow 0$.
    Putting together these two facts leads to the conclusion that $(f(x_k)), (f(y_k))$ both converge to the same real number.

    Now assume that $(x_k) \subset Q_i, x_k \rightarrow 0$ and that $y_k \rightarrow 0$ is a sequence that lies on the subset plane which is such that at least one coordinate is zero, and that all non-zero coordinates have the same signs as those of $Q_i$ (intuitively, one of the ``borders'' of $Q_i$.
    Crucially for the remainder of the proof, as long as $n \geq 2$ this subset never degenerates to a single point, which is exactly what happens when $n = 1$.
    Notice now that the segment formed by the points $x_k, y_k$ lies entirely inside the domain of $g$ (in fact, excluding $y_k$ it lies entirely inside $Q_i$).
    Thus, the mean value theorem again applies, and we obtain a similar bound on $\lvert g(y_k) - g(x_k) \rvert$, showing that in fact $g(y_k) \rightarrow L_i$ as well, where $L_i$ is the unique limit obtained previously for $Q_i$.
    The final part of the proof is the observation that given any two quadrants, we can find a ``path'' between them that consists of subsets of planes as described above, and potentially other quadrants.
    This allows us to transitively reason that all $L_i$ must in fact be equal, thus completing the proof.

    Lastly, a concrete counterexample for $n = 1$ is the function $g(x) = -1, x < 0, g(x) = 1, x > 0$, which has zero partial derivatives everywhere, but cannot be extended to a continuous function in all of $\mathbb{R}$.
\end{solution}

\begin{exercise}{18}
    Let $U = \mathbb{R}^2 \setminus \{(0, x_2) \lvert x_2 \geq 0\}$ and define $f: U \rightarrow \mathbb{R}$ by
    $$f(x) = \begin{cases}
        x_2^2, x_1 > 0 \text{ and } x_2 \geq 0 \\
        0, x_1 < 0 \text{ and } x_2 < 0
    \end{cases}$$

    (i) Show that $D_1 f = 0$ on all of $U$ but that $f$ is not independent of $x_1$.

    Now suppose that $U \subset \mathbb{R}^2$ is an open set having the property that for each $x_2 \in \mathbb{R}$ the set $\{ x_1 \in \mathbb{R} \lvert (x_1, x_2) \in \mathbb{U}\}$ is an interval.

    (ii) Prove that $D_1 f = 0$ on all of $U$ implies that $f$ is independent of $x_1$.
\end{exercise}

\begin{solution}

    (i) Let $x \in \mathbb{R}^2$.
    If $x_1 < 0$, then $f(x) = 0$ and thus $D_1 f(x) = 0$.
    If $x_1 > 0$, then $f(x) = x_2^2$, and thus again $D_1 f(x) = 0$.
    However, observe that $f(-1, 1) = 0$ and $f(1, 1) = 1$, which means that when varying only $x_1$, the value of $f$ varies, so it cannot be independent of $x_1$. A one-dimensional analogue of this would be a function that is stepwise constant (e.g. $f :\mathbb{R} \setminus \{0\}, f(x) = 1, x < 1, f(x) = 2, x > 0$), and as such has derivative 0 in its entire domain but cannot be said to be independent (in this case, constant) of $x$.

    (ii) We need to show that for any three $x_1, y_1, c \in \mathbb{R}$ such that $(x_1, c), (y_1, c) \in U$ it holds that $f(x_1, c) = f(y_1, c)$. 
    In other words, that when $x_2$ is kept constant, $f$'s value cannot change.
    We know that for a given $c, \{x_1 \in \mathbb{R} \lvert (x_1, c) \in U\}$ is an interval.
    Let it be called $I$, with endpoints $a, b$ which it may or may not include.
    Define $g: I \rightarrow \mathbb{R}, g(x) = f(x, c)$. 
    Then it is the case that $g$ is differentiable, and that $g'(x) = D_1 f(x, c) = 0$.
    In other words, $g$ has a derivative of 0 and its domain is an interval.
    From calculus 1, we know that this means that $g$ is constant (one can prove this by contradiction and the mean value theorem), in other words that $g(x) = d \implies f(x, c) = d$ for all $x \in I$, which is exactly what we wanted to show.
\end{solution}

\begin{exercise}{22}
    Let $U \subset \mathbb{R}^n$ and $V \subset \mathbb{R}^p$ be open and let $f: U \rightarrow V$ be a $C^1$ mapping.
    Define $T f : U \times \mathbb{R}^n \rightarrow \mathbb{R}^p \times \mathbb{R}^p$, the \textit{tangent mapping } of $f$ to be the mapping given by
    $$T f(x, h) = (f(x), Df(x) h)$$
    Let $V \subset \mathbb{R}^p$ be open and let $V \rightarrow \mathbb{R}^q$ be a $C^1$ mapping.
    Show that the chain rule takes the natural form
    $$T(g \circ f) = Tg \circ T f: U \times \mathbb{R}^n \rightarrow \mathbb{R}^q \times \mathbb{R}^q$$
\end{exercise}

\begin{solution}

    The chain rule states that for $x \in U$:
    $$[D(g \circ f)](x)(h) = [D(g)](f(x))Df(x)(h)$$
    By using the tangent mapping definition given in the exercise, we have that for $x \in U, h \in \mathbb{R}^n$:
    $$T(g \circ f)(x, h) = ((g \circ f)(x), [D(g \circ f)](x)(h)) = (g(f(x)), [D(g)](f(x))Df(x)(h))$$
    Consider now what the RHS of the given equality describes: we take the result of $Tf$ on $(x, h)$, which belongs in $V \times \mathbb{R}^p$ (a point in $V$ and a direction vector), and pass it as an argument to $Tg$, which indeed accepts arguments belonging in $V \times \mathbb{R}^p$ (a point in $V$ and a direction vector).
    This is therefore well-defined.
    Furthermore:
    $$(Tg \circ Tf)(x, h) = Tg(Tf(x, h)) = Tg(f(x), Df(x)h) = (g(f(x)), [D(g)](f(x))Df(x)(h)),$$
    which yields that $T(g \circ f) = Tg \circ Tf$.
\end{solution}

\begin{exercise}{23}
    This exercise serves as another proof of the mean value theorem.
    As a reminder for notation, the theorem states that:

    Let $U$ be a convex open subset of $\mathbb{R}^n$ and let $f: U \rightarrow \mathbb{R}^p$ be a differentiable mapping.
    Suppose that the derivative $Df: U \rightarrow \mathcal{L}(\mathbb{R}^n, \mathbb{R}^p)$ is bounded on $U$, that is, there exists $k > 0$ with $\lvert \lvert Df(\xi)h \rvert \rvert \rvert \rvert \leq k \lvert \lvert h \rvert \rvert$, for all $\xi \in U$ and $h \in \mathbb{R}^n$, which is the case if $\lvert \lvert Df(\xi) \rvert \rvert_{\text{eucl}} \leq k$.
    Then $f$ is Lipschitz continuous on $U$ with Lipschitz constant $k$, in other words

    $$\lvert \lvert f(x) - f(x') \rvert \rvert \leq k \lvert \lvert x - x' \rvert \rvert, x, x' \in U$$

    Let $x', x \in U$ be arbitrary and consider $x_t, x_s \in L(x', x)$, i.e., on the segment connecting $x', x$.
    Suppose $m > k$.

    (i) By means of the definition of differentiability verify, for $t > s$ and $t - s$ sufficiently small,

    $$\lvert \lvert f(x_t) - f(x_s) - (t - x)Df(x_s)(x - x') \rvert \rvert \leq (m - k)(t - s)\lvert \lvert x - x' \rvert \rvert$$

    Applying the reverse triangle inequality, deduce
    
    $$\lvert \lvert f(x_t) - f(x_s) \rvert \rvert \leq m(t - s)\lvert \lvert x - x' \rvert \rvert$$

\end{exercise}

\begin{solution}
    
    (i) Let $\epsilon = m - k > 0$.
    By the definition of differentiability of $f$ at $s$, we have that there exists $\delta > 0$ such that for any $h$ with $\lvert \lvert h \rvert \rvert < \delta$ it holds that:

    $$\frac{\lvert \lvert f(x_s + h) - f(x_s) - Df(x_s)(h) \rvert \rvert}{\lvert \lvert h \rvert \rvert} < \epsilon$$

    More specifically then, for sufficiently small $t - s > 0$, such that $(t - s)\lvert \lvert x - x' \rvert \rvert < \delta$, the above holds for $h = (t - s) (x - x') = x_t - x_s$.
    Thus:

    $$\frac{\lvert \lvert f(x_s + (x_t - x_s)) - f(x_s) - Df(x_s)((t - s)(x - x'))\rvert \rvert}{(t - s)\lvert \lvert x - x' \rvert \rvert} < \epsilon \implies$$
    $$\lvert \lvert f(x_t) - f(x_s) - (t - s)Df(x_s)(x - x') \rvert \rvert <(m - k)(t - s)\lvert \lvert x - x' \rvert \rvert$$

    As pointed out in the hint, by applying the reverse triangle inequality we can also obtain:

    $$\Bigl \lvert \lvert \lvert f(x_t) - f(x_s) \rvert \rvert - (t - s)\lvert \lvert Df(x_s)(x - x') \rvert \rvert \Bigr \rvert < (m - k)(t - s) \lvert \lvert x - x' \rvert \rvert \implies $$
    $$\lvert \lvert f(x_t) - f(x_s) \rvert \rvert < (m - k)(t - s)\lvert \lvert x - x' \rvert \rvert + (t - s)\lvert \lvert Df(x_s)(x - x') \rvert \rvert \implies$$
    $$\lvert \lvert f(x_t) - f(x_s) \rvert \rvert < (m - k)(t - s) \lvert \lvert x - x' \rvert \rvert + (t - s)\lvert \lvert Df(x_s) \rvert \rvert \cdot \lvert \lvert x - x' \rvert \rvert \implies $$
    $$\lvert \lvert f(x_t) - f(x_s) \rvert \rvert < (m - k)(t - s)\lvert \lvert x - x' \rvert \rvert + (t -s)k \lvert \lvert x - x' \rvert \rvert \implies \lvert \lvert f(x_t) - f(x_s) \rvert \rvert < m(t -s)\lvert \lvert x - x' \rvert \rvert,$$

    where we have used the analogue of the Cauchy-Schwarz inequality for linear operators, and the fact that the derivative of $f$ is bounded.

\end{solution}

\begin{exercise}{24}
    Let $U$ be a convex open subset of $\mathbb{R}^n$ and let $f: U \rightarrow \mathbb{R}^p$ be a differentiable mapping.
    Prove that the following assertions are equivalent.

    (i) The mapping $f$ is Lipschitz continuous on $U$ with Lipschitz constant $k$.

    (ii) $\lvert \lvert D f(x) \rvert \rvert \leq k$ for all $x \in U$, where $\lvert \lvert \cdot \rvert \rvert$ denotes the operator norm.
\end{exercise}

\begin{solution}
    
    (i) $\implies$ (ii): Suppose first that $f$ is Lipschitz continuous.
    Recall the definition of differentiability at a particular $x$:

    $$\lim_{h \rightarrow 0} \frac{f(x + h) - f(x) - D f(x)(h)}{\lvert \lvert h \rvert \rvert} = 0,$$

    with $Df(x)$ being the (unique) linear operator that satisfies this.
    Take any $\epsilon > 0$.
    Then there exists $\delta > 0$ such that whenever $\lvert \lvert h \rvert \rvert < \delta$, it holds that:

    $$\Bigl\lvert \Bigl\lvert \frac{f(x + h) - f(x) - Df(x)(h)}{\lvert \lvert h \rvert \rvert} - 0 \Bigr \rvert \Bigr \rvert \leq \epsilon \implies \lvert \lvert f(x + h) - f(x) - Df(x)(h) \rvert \rvert \leq \epsilon \lvert \lvert h \rvert \rvert$$

    By the triangle inequality we have that:

    $$\Bigl \lvert - \lvert \lvert f(x+h) - f(x) \rvert \rvert + \lvert \lvert Df(x)(h) \rvert \rvert \Bigr \rvert \leq \epsilon \lvert \lvert h \rvert \rvert \implies \lvert \lvert Df(x)(h) \rvert \rvert < \epsilon \lvert \lvert h \rvert \rvert + \lvert \lvert f(x+h) - f(x) \rvert \rvert$$

    Now, Lipschitz continuity guarantees that $\lvert \lvert f(x+h) - f(x) \rvert \rvert \leq k \lvert \lvert x + h - x \rvert \rvert = k \lvert \lvert h \rvert \rvert$.
    We therefore have that $\lvert \lvert Df(x)(h) \rvert \rvert < \epsilon \lvert \lvert h \rvert \rvert + k \lvert \lvert h \rvert \rvert$.
    Note now that for computing the operator norm of $Df(x)$ we are interested in $h$ that have unit norm.
    Pick then any such $h$, and observe that for any $\epsilon > 0$ there exists $\delta > 0$ such that $h' = \delta h$ satisfies the above inequality, which means:

    $$\lvert \lvert Df(x)(\delta h) \rvert \rvert < \epsilon \lvert \lvert \delta h \rvert \rvert + k \lvert \lvert \delta h \rvert \rvert \implies \lvert \lvert Df(x)(h) \rvert \rvert < (\epsilon + k)$$
    Because this holds for \textit{any} $\epsilon > 0$, we must necessarily have that $\lvert \lvert Df(x)(h) \rvert \rvert \leq k$, and since this is true for any unit-length vector, we can conclude that $\lvert \lvert Df(x) \rvert \rvert \leq k$ as well.

    (ii) $\implies$ (i): First, pick any vector $h \in \mathbb{R}^n$ and observe that:

    $$\lvert \lvert Df(x) \rvert \rvert \leq k \implies \lvert \lvert Df(x)(h/ \lvert \lvert h \rvert \rvert) \rvert \rvert \leq k \implies \lvert \lvert Df(x)(h) \rvert \rvert \leq k \lvert \lvert h \rvert \rvert$$ 

    Given that $U$ is convex and open, we can now simply apply theorem 2.5.3 to conclude that $f$ is Lipschitz continuous on $U$ with Lipschitz constant $k$.
\end{solution}

\begin{exercise}{25}
    Let $U \subset \mathbb{R}^n$ be open and $a \in U$, and let $f: U \setminus \{a\} \rightarrow \mathbb{R}^p$ be differentiable.
    Suppose there exists $L \in \mathcal{L}(\mathbb{R}^n, \mathbb{R}^p)$ with $\lim_{x \rightarrow a} Df(x) = L$.
    Prove that $f$ is differentiable at $a$ with $Df(a) = L$.

    \textbf{Hint}: Apply the Mean Value Theorem to $x \mapsto f(x) - Lx$.
\end{exercise}

\begin{solution}

    We first note that for the exercise to make sense it has to be the case that $n > 1$, otherwise easy counterexamples can be found, and we cannot safely draw the needed intermediate conclusion (by exercise 15 part 2, and by observing that $\lim_{x \rightarrow a} Df(x) = L$ implies that $D_jf(x)$ are bounded for all $j$) that $f$ is continuous at $a$ and thus $f(a)$ is well-defined.
    
    Consider any open half-ball $H_{\delta}(a) \subset U \setminus\{a\}$, and any two $x, y$ inside it.
    This set is convex, and as such we can apply the mean value theorem on $f$ to obtain the existence of $c$ on the segment connecting $x, y$ such that:

    $$f(y) - f(x) -L(y - x)= Df(c)(y - x) - L(y - x) \implies \lvert \lvert f(y) - f(x) - L(y - x)\rvert \rvert = \lvert \lvert Df(c)(y - x ) - L(y - x)\rvert \rvert$$
    $$\implies \lvert \lvert f(y) - f(x) - L(y - x)\rvert \rvert \leq \lvert \lvert Df(c) - L \rvert \rvert \cdot \lvert \lvert y - x \rvert \rvert,$$

    where we use the Frobenius norm for the linear function $Df(c) - L$.

    Now we note that because $\lim_{x \rightarrow a} Df(x) = L$, for any given $\epsilon > 0$, we can find $\delta$ such that for $\lvert \lvert x -a \rvert \rvert < \delta$ it holds that $\lvert \lvert Df(c) - L \rvert \rvert < \epsilon/2$.
    For an arbitrary $\epsilon$, pick then this corresponding $\delta$, and consider any two $x, y$ with $\lvert \lvert y - x \rvert \rvert < \delta$.
    If these lie inside \textit{some} half-ball $H_{\delta}(a)$, then by applying the inequality above we draw the conclusion that $\lvert \lvert f(y) - f(x) - L(y - x) \rvert \rvert < \frac{\epsilon}{2} \cdot \lvert \lvert y - x \rvert \rvert$.
    If they don't, this means that they are colinear.
    Crucially, because $n \geq 1$, the ``perpendicular bisector'' (used loosely here, since it may well be more than one-dimensional) of the segment connecting them is not empty.
    More specifically, there exists $z$ such that $\lvert \lvert z - x \rvert \rvert = \lvert \lvert z - y \rvert \rvert < \min\{\delta_1, \delta_2\}$, where $\delta_1, \delta_2$ are obtained by using the limit definition above for $\epsilon/4$ on half-balls containing $y, z$ (for $\delta_1$) and $z, x$ (for $\delta_2$).
    This means that:

    $$\lvert \lvert f(y) - f(x) - L(y - x) \rvert \rvert \leq \lvert \lvert f(y) - f(z) + f(z) - f(x) - L(y - z + z - x) \rvert \rvert$$
    $$\leq \lvert \lvert f(y) - f(z) - L(y - z) \rvert \rvert + \lvert \lvert f(z) - f(x) - L(z - x) \rvert \rvert$$
    $$\leq \lvert \lvert Df(c_1) - L \rvert \rvert \cdot \lvert \lvert z - y \rvert \rvert + \lvert \lvert Df(c_2) - L \rvert \rvert \cdot \lvert \lvert z - x \rvert \rvert < \frac{\epsilon}{4}\lvert \lvert y - x \rvert \rvert + \frac{\epsilon}{4} \lvert \lvert y -x \rvert \rvert = \frac{\epsilon}{2} \lvert \lvert y - x \rvert \rvert$$

    Notice then that if we let $y \rightarrow a$, and by recalling that limits preserve non-strict inequalities, we have that conclusion that for any $\epsilon > 0$, there exists $\delta > 0$ such that for $\lvert \lvert x - a \rvert \rvert < \delta, \lvert \lvert f(a+ h) - f(a) - L(h) \rvert \rvert \leq \frac{\epsilon}{2} \lvert \lvert h \rvert \rvert < \epsilon \lvert \lvert h \rvert \rvert$, which is precisely the definition of $Df(a) = L$.
\end{solution}

\begin{exercise}{32}
    Let $f: \mathbb{R}^n \setminus \{0\} \rightarrow \mathbb{R}$ be a differentiable function.

    (i) Let $x \in \mathbb{R}^n \setminus \{0\}$ be a fixed vector and define $g: \mathbb{R}_{+} \rightarrow \mathbb{R}$ by $g(t) = f(tx)$.
    Show that $g'(t) = D f(tx)(x)$.

    Assume $f$ to be positively homogeneous of degree $d \in \mathbb{R}$, that is, $f(tx) = t^df(x)$, for all $x \neq 0, t \in \mathbb{R}_{+}$.

    (ii) Prove the following, known as Euler's identity:
    $$Df(x)(x) = \langle x, \nabla f(x) \rangle = d f(x), x \in \mathbb{R}^n \setminus \{0\}$$

    (iii) Conversely, prove that $f$ is positively homogeneous of degree $d$ if we have $Df(x)(x) = df(x)$, for every $x \in \mathbb{R}^n \setminus \{0\}$. \textbf{Hint}: Calculate the derivative of $t \mapsto t^{-d}g(t), t \in \mathbb{R}_{+}$.
\end{exercise}

\begin{solution}

    (i) Set $h: \mathbb{R}^{+} \rightarrow \mathbb{R}^n, h(t) = tx$.
    Then $h$ is differentiable, and it is the case that $Dh(t)(s) = x$.
    Also, $g(t) = f(h(t))$, so as a composition of differentiable functions $g$ is also differentiable, and:
    $$Dg(t)(s) = ([D(f)]h(t)))([Dh(t)](s)) = D(f)(tx)(x)$$

    (ii) We know that for any differentiable $f: \mathbb{R}^n \rightarrow \mathbb{R}$:
    $$\langle v, \nabla f(x) \rangle = [D f(x)](v)$$
    Therefore $Df(x)(x) = \langle x, \nabla f(x) \rangle$.
    By setting $g(t) = f(tx) = t^df(x)$, we have that $g'(t) = dt^{d-1}f(x)$.
    Therefore, $g'(1) = df(x)$, while from part (i) we have that $g'(1) =  D(f)(x)(x)$.
    This implies that $Df(x)(x) = d f(x) = \langle x, \nabla f(x) \rangle$.

    (iii) First, we have that by the linearity of the derivative, $df(tx) = Df(tx)(tx) = t Df(tx)(x)$.
    Then, by using the hint, we define $h(t) = t^{-d}g(t)$ with $g$ as in (i) and we observe that this is a differentiable function with:
    $$h'(t) = g'(t)t^{-d} + (-d)t^{-d-1}g(t) = t^{-d}Df(tx)(x) + (-d)t^{-d-1}f(tx)$$
    $$ = t^{-d}\frac{df(tx)}{t} - dt^{-d-1}f(tx) = 0$$
    But then this means that $h$ is a constant function, therefore that $c = t^{-d}g(t) = t^{-d}f(tx)$ for some $c \in \mathbb{R}$. 
    Also, by setting $t = 1$ we have that $c = f(x)$, which means that $t^{-d}f(tx) = f(x) \implies f(tx) = t^{d}f(x)$, i.e.\ that $f$ is positively homogeneous of degree $d$.
    
\end{solution}

\newpage

\begin{exercise}{44}
    Let $A \in \mathbb{R}^{n \times n}$. We write $A = (a_1 \cdots a_n)$ if $a_j \in \mathbb{R}^n$ is the $j$-th column vector of $A, 1 \leq j \leq n$.
    This means that we identify $\mathbb{R}^{n \times n}$ with $\mathbb{R}^n \times \ldots \mathbb{R}^n$.
    Furthermore, we denote by $A^{\#}$ the complementary matrix as in Cramer's rule (so $A^{\#} = (a_{ij}^{\#} = ((-1)^{i - j}\det A_{ji}))$, where $A_{ij}$ is the matrix obtained by deleting the $i$-th row and $j$-th column).

    (i) Consider the function $\det: \mathbb{R}^{n \times n} \rightarrow \mathbb{R}$ as an element of $\mathcal{L}^{n}(\mathbb{R}^n, \mathbb{R})$ via the identification above.
    Now prove that its derivative $D (\det)(A) \in \mathcal{L}(\mathbb{R}^n, \mathbb{R})$ is given by
    $$D(\det)(A) = \tr \circ A^{\#},$$
    where $\tr$ denotes the trace of a matrix.
    More explicitly:
    $$D(\det)(A) H = \tr(A^{\#}H) = \langle (A^{\#})^T, H \rangle, H \in \mathbb{R}^{n \times n}$$
    Here we use the result from exercise 2.1 (ii).
    In particular, verify that $\nabla \det A = (A^{\#})^T$, the cofactor matrix, and
    $$(D \det)(I) = \tr \ \ (*), (D \det)(A) = \det A (\tr \circ A^{-1}) \ \ (**),$$
    for $A$ invertible, by means of Cramer's rule.

    (ii) Let $X \in \mathbb{R}^{n \times n}$ and define $e^X \in \mathbb{R}^{n \times n}$ as in Example 2.4.10 (i.e. $e^X = I +  X + \frac{1}{2}X^2 + \frac{1}{3!}X^3 + \ldots$).
    Show that for $t \in \mathbb{R}$:
    $$\frac{d}{dt} \det(e^{tX}) = \frac{d}{ds} \Biggr \rvert_{s=0} \det(e^{(s+ t)X}) = \frac{d}{ds} \Biggr \rvert_{s = 0} \det(e^{sX}) \det(e^{tX}) = \tr X \det(e^{tX})$$

    Deduce by solving this differential equation that (compare also with Formula 5.33)
    $$\det(e^{tX}) = e^{t \cdot \tr X}$$

    (iii) Assume that $A$ is an invertible $n \times n$ matrix.
    Using $\det(A + H) = \det(A) \det(I + A^{-1}H)$, deduce (**) from (*) in (i).
    Now also derive $(\det A)A^{-1} = A^{\#}$ from (i).

    (iv) Assume that $A: \mathbb{R} \rightarrow \mathbf{GL}(n, \mathbb{R})$ (i.e.\ $A$ always yields invertible matrices) is a differentiable mapping.
    Derive from part (i) the following formula for the derivative of $\det \circ A: \mathbb{R} \rightarrow \mathbb{R}$:
    $$\frac{1}{\det \circ A}(\det \circ A)' = \tr(A^{-1} \circ DA),$$
    in other words (compare with part (ii))
    $$\frac{d}{dt}(\log \det A)(t) = \tr(A^{-1} \frac{dA}{dt})(t), t \in \mathbb{R}$$

    (v) Suppose that the mapping $A: \mathbb{R} \rightarrow \mathbb{R}^{n \times n}$ is differentiable and that $F: \mathbb{R} \rightarrow \mathbb{R}^{n \times n}$ is continuous, while we have the differential equation
    $$\frac{dA}{dt}(t) = F(t)A(t), t \in \mathbb{R}$$
    In this context, the \textit{Wronskian} $w \in C^{1}(\mathbb{R})$ of $A$ is defined as $w(t) = \det A(t)$.
    Use part (i), exercise 2.1 (i) and Cramer's rule to show that for $t \in \mathbb{R}$
    $$\frac{dw}{dt}(t) = \tr F(t) w(t), \text{ and deduce } w(t) = w(0)e^{\int_{0}^{t} \tr F(\tau) d\tau}$$
\end{exercise}

\begin{solution}

    (i) We have shown in exercise 2.15 of Spivak that:
    $$D(\det)(a_1, \ldots, a_n)(h_1, \ldots, h_n) = \sum_{i=1}^{n} \det \begin{pmatrix}
        a_1 \\ \vdots \\ h_i \\ \vdots \\ a_n
    \end{pmatrix}$$    
    We therefore have but to show that the RHS of the equation given in (i) equals the RHS of this equation.
    We have that $(\tr \circ A^{\#})(H) = \tr(A^{\#} H)$.
    Due to the trace we are only interested in the diagonal elements of the argument.
    In particular, we have that:
    $$\tr(A^{\#}H) = \sum_{i=1}^{n} \sum_{j=1}^{n} (-1)^{i - j} \det A_{ji} h_{ji}$$
    We now manipulate the expression we have from Spivak using the property of row linearity:
    $$D(\det)(a_1, \ldots, a_n)(h_1, \ldots, h_n) = \sum_{i=1}^{n} \sum_{j=1}^{n} h_{ij} \det \begin{pmatrix}
        a_1 \\ \vdots \\ a_{i-1} \\ e_j \\ \vdots \\ a_n
    \end{pmatrix}$$
    Now we observe that we can apply row-wise Laplace expansion on the $i$-th row of each of the summands for which the outer index equals $i$ to rewrite it as:
    $$\det \begin{pmatrix}
        a_1 \\ \vdots \\ a_{i-1} \\ e_j \\ \vdots \\ a_n
    \end{pmatrix} = (-1)^{i +j} \det A_{ij},$$
    where we used the fact that this row contains exactly one non-zero element (at column $j$), and that deleting this row and column yields the same result as deleting the same row and column from $A$.
    Note that $(-1)^{i + j} = (-1)^{i - j}$, and thus from this derivation we obtain that:
    $$D(\det)(a_1, \ldots, a_n) (h_1, \ldots, h_n) = \tr(H A^{\#}) = \tr(A^{\#}H),$$
    by using the commutative property of the trace.
    This concludes the proof,
    and now we can of course write $D(\det)(A)H = \langle (A^{\#})^{T}, H \rangle$ by the definition of inner product for matrices we saw in exercise 2.1.
    From this we additionally conclude that $\nabla \det A = (A^{\#})^T$: the gradient is precisely the vector whose inner product with each ``direction vector'' (here, the matrix $H$) yields the directional derivative along $H$, i.e.\ $D(\det) (A) H$.

    Now if we were to evaluate at $A = I$, we first have that $A^{\#} = I$.
    To see why this is the case, consider erasing column $i$ and row $i$: we are simply left with a smaller identity matrix, which of course has determinant 1.
    On the other hand, erasing row $i$, column $j, i \neq j$ will result in an either upper or lower triangular matrix that has at least one zero on the diagonal, and such a matrix will have determinant zero.

    Thus:
    $$D(\det)(I)(H) = \tr (I H) = \tr(H)$$
    For $A$ invertible, we know by Cramer's rule that $\det A \cdot I = A A^{\#} = A^{\#} A$, and since $A^{-1}$ is well defined, $(\det A) A^{-1} = A^{\#}$.
    Substituting into the trace expression from before:
    $$D(\det)(A)(H) = \tr(A^{\#}H) = \tr((\det A)A^{-1} H) = \det A \tr(A^{-1}H)$$

    (ii) Call $f(t) = e^{tX}, g(A) = \det A $, in which case $f: \mathbb{R} \rightarrow \mathbb{R}^{n \times n}$ and $g: \mathbb{R}^{n \times n} \rightarrow \mathbb{R}$, and in which case we are interested in $(g \circ f)'$.
    By the chain rule, we have that:
    $$ D (g \circ f) (t) (h) = [D (g) (f(t))]( [D (f) (t)] (h))$$
    We know from Example 2.4.10 that $[D(f)(t)](h) = h e^{tX} X$, and by also using the expression we obtained in part (i) for $A$ invertible (as is $e^{tX}$) we get:
    $$ D (g \circ f) (t) (h) = [D (g) (f(t))](he^{tX}X) = \det(e^{tX}) \tr((e^{tX})^{-1}he^{tX}X) = \det(e^{tX}) \tr(hX) = \det{e^{tX}} \tr(X)h,$$
    which is equivalent to saying that $\frac{d}{dt} \det(e^{tX}) = \tr(X) \det(e^{tX})$.

    Now notice that the function $g \circ f$ maps from $\mathbb{R}$ to $\mathbb{R}$, and therefore this differential equation is of the general form:
    $$\frac{d}{dt} (g \circ f)(t) = a (g \circ f)(t),$$
    whose solution is well known to be $(g \circ f)(t) = Ce^{at}$, which implies that $\det(e^{tX}) = Ce^{\tr(X) t}$. 
    To determine $C$, we set $t = 0$, in which case $e^{tX}$ is the identity matrix, and thus the LHS is 1 and the RHS is $C$, meaning that $C = 1$.

    (iii) We begin by writing out the limit that corresponds to (*), i.e.\ to $(D \det)(I) = tr$:
    $$\lim_{H \rightarrow 0}\frac{1}{\lvert \lvert H \rvert \rvert}( \det(I + H) - \det(I) - \tr(H)) = 0 \implies \lim_{H \rightarrow 0}\frac{1}{\lvert \lvert H \rvert \rvert}(\det(I+H) - 1 -\tr(H)) = 0$$
    Now, for the derivative of $\det$ at an invertible $A$, we are interested in finding the linear transformation $\mathcal{L}$ such that:
    $$\lim_{H \rightarrow 0}\frac{1}{\lvert \lvert H \rvert \rvert}(\det(A + H) - \det(A) - \mathcal{L}(H)) = 0$$
    Using the provided hint:
    $$\lim_{H \rightarrow 0}\frac{1}{\lvert \lvert H \rvert \rvert}(\det(A + H) - \det(A) - \mathcal{L}(H)) = \lim_{H \rightarrow 0}\frac{1}{\lvert \lvert H \rvert \rvert}(\det(A) \det(I + A^{-1}H) - \det(A) - \mathcal{L}(H))$$
    This then motivates us to set $U = A^{-1}H$, which means $AU = H$, and also that since $A^{-1}$ is kept constant as $H \rightarrow 0$, it is also the case that $U \rightarrow 0$. Therefore, the above limit can be rewritten as:
    $$\lim_{U \rightarrow 0}\frac{1}{\lvert \lvert A U \rvert \rvert}(\det(A) \det(I + U) - \det(A) - \mathcal{L}(AU)) = \lim_{U \rightarrow 0}(\frac{1}{\lvert \lvert AU \rvert \rvert}\det(A)(\det(I + U) - 1) - \frac{1}{\lvert \lvert AU \rvert \rvert}\mathcal{L}(AU))$$
    Now we have that $\lvert \lvert U \rvert \rvert = \lvert \lvert A^{-1}A U \rvert \rvert \leq \lvert \lvert A^{-1} \rvert \rvert \cdot \lvert \lvert A U \rvert \rvert$, we observe that for the first term of the sum:
    $$\frac{1}{\lvert \lvert AU \rvert \rvert}\lvert \lvert \det(A)(\det(I+U) - 1) \rvert \rvert \leq \frac{1}{\lvert \lvert A^{-1} \rvert \rvert \cdot \lvert \lvert U \rvert \rvert}\lvert \lvert \det(A)(\det(I+U) - 1) \rvert \rvert $$
    But now the right side is known to tend to zero if one were to subtract $\det(A) \tr(U)$ from the quantity inside the magnitude of the numerator.
    This therefore motivates us to have $\mathcal{L}(AU) = \det(A) \tr(U)$, which, substituting $U$, yields $\mathcal{L}(H) = \det(A) \tr(A^{-1}H)$, and since \textit{a} linear transformation that makes the limit zero is \textit{the} derivative of $\det$ at $A$, we have indeed arrived at (**).

    (iv) We have shown in (i) that for $A$ invertible:
    $$D(\det)(A)(H) = \det(A) tr(A^{-1}H)$$
    Let then $f(t) = \det(A(t)), f: \mathbb{R} \rightarrow \mathbb{R}$.
    By using the chain rule:
    $$D(f)(t)(h) = D(\det)(A(t))(D(A)(t)(h)) \implies D(f)(t)(h) = \det(A(t)) \tr((A(t)^{-1}D(A)(t)(h)))$$
    Because $A(t)$ is always invertible, we can divide both sides by it, and by simplifying the notation to match the exercise we obtain:
    $$\frac{1}{\det A}(\det \circ A)' = \tr(A^{-1} \circ DA)$$
    Now, by recalling the derivative of the (natural) logarithm, we can also rewrite this as:
    $$\frac{d}{dt}(\log \det A)(t) = \tr(A^{-1}\frac{dA}{dt})(t)$$

    (v) We begin by differentiating $w(t)$ and using part (i):

    $$D(w)(t)(h) = D(\det)(A(t))(D(A)(t)(h)) \implies D(w)(t)(h) = \tr(A(t)^{\#} D(A)(t)(h))$$

    Note now that by simplifying the notation and substituting from the given differential equation we obtain that:
    
    $$\frac{dw}{dt}(t) = \tr(A(t)^{\#} F(t)A(t)) \implies \frac{dw}{dt}(t)= \tr(A{t}^{\#}A(t)F(t)) \implies \frac{dw}{dt}(t) = \tr(\det A(t) I F(t)) $$
    $$\implies \frac{dw}{dt}(t) = \det A(t) \tr(F(t)) = w(t) \tr(F(t)),$$
    
    where we have used properties of the trace (shown in exercise 2.1) and Cramer's rule.
    This is a differential equation somewhat similar to (ii), except that $w(t)$ is multiplied with a function of $t$ instead of a constant.
    Because $F$ is continuous, the solution to this differential equation is of the form:
    $$w(t) = c e^{\int_{0}^{t} \tr F(\tau) d \tau},$$
    and by substituting $t = 0$ we have $w(0) = c$, where $w(0)$ depends on $A(0)$ (an ``initial condition'' for the differential equation).
\end{solution}